The security of the web improved greatly throughout the last couple of years.
A large majority of the web is now served encrypted as part of HTTPS, and
web browsers accordingly moved from positive to negative security indicators
that warn the user if a connection is insecure.  A secure connection requires
that the server presents a valid certificate that binds the domain name in
question to a public key.  A certificate used to be valid if signed by a trusted
Certificate Authority (CA), but web browsers like Google Chrome and
Apple's Safari have additionally started to mandate Certificate Transparency (CT)
logging to overcome the weakest-link security of the CA ecosystem.  Tor and the
Firefox-based Tor Browser have yet to enforce CT.

\hspace{12pt}
In this paper, we present privacy-preserving and incrementally-deployable
designs that add support for CT in Tor. Our designs go beyond the currently
deployed CT enforcements that are based on blind trust:
	if a user that uses Tor Browser is man-in-the-middled over HTTPS,
	we probabilistically detect and disclose cryptographic evidence of CA and/or
	CT log misbehavior.
The first design increment allows Tor to play a vital role in the overall goal
of CT:
	detect mis-issued certificates and hold CAs accountable.
We achieve this by randomly cross-logging a subset of certificates into other CT
logs.  The final increments hold misbehaving CT logs accountable, initially
assuming that some logs are benign and then without any such assumption.
Given that the current CT deployment lacks strong mechanisms to verify if log
operators play by the rules, exposing misbehavior is important for the web in
general and not just Tor.  The full design turns Tor into a system for
maintaining a probabilistically-verified view of the CT log ecosystem available
from Tor's consensus. 
