\section{CT Trust Anchors} \label{app:ct-trust-anchors}
The standardized CT protocol suggests that a log's trust anchors should
``usefully be the union of root certificates trusted by major browser
vendors''~\cite{ct,ct/bis}.  Apple further claims that a log in their CT program
``must trust all root CA certificates included in Apple's trust
store''~\cite{apple-log-policy}.  This bodes well for CTor:
	we assumed that the existence of independent log operators implies the
	ability to add certificate chains (Section~\ref{sec:base}) or complete SFOs
	(Section~\ref{sec:log}) into logs that the attacker does not control.
The preprint of Korzhitskii and Carlsson show that the overlap between root
certificates that are trusted by major browser vendors and CT logs increased
over time, but, \emph{no log} accepts all roots to
date~\cite{ct-root-landscape}.  Let's Encrypt is likely a commonly excluded CA,
following from the large volume of issued certificates which might deter some
logs (or operators) from including them as a trust anchor.  All is not lost,
however.  A few notes:
\begin{itemize}
	\item Even if there are no independent logs available for some CA, there is
		still significant value in practise to (re)add the presented certificate
		chain and SCTs to the same logs that supposedly logged them.  For
		example, the attacker might have access to signing keys but not the
		actual infrastructure.  In such a scenario the mis-issued certificate
		makes it into the public.  Given SCTs, the log operators also learn that
		their signing keys were compromised at some point.
	\item Most log operators only exclude a small fraction of widely accepted
		root certificates: 1--5\%~\cite{ct-root-landscape}.  This narrows down
		the possible CAs that the attacker must control by 1--2 orders of
		magnitude.  In other words, to be entirely sure that CTor would (re)add
		a mis-issued SFO to the attacker-controlled CT logs, this smaller group
		of CAs must issue the underlying certificate.  It is arguably harder to
		take control of Let's Encrypt than, say, a smaller CA that may be
		coerced to misbehave by national law enforcement.
	\item Given that the coverage of CT logs increased over time, such a
		trend would likely continue if the ecosystems around CT benefit from it
		(e.g., CTor).
	\item The two designs that rely on independent CT logs also permit
		community-driven CT logs that are not recognized by any major browser
		vendor.  Such logs, if included in the Tor consensus, can be viewed as
		parties facilitating auditing of the CT landscape.  Such interest
		exist already, e.g., in the form of emerging CT monitoring
		services~\cite{lwm,ct-monitors}.  It could also be an alternative for
		CAs that, somehow, need to stay relevant in light of today's
		\emph{freely available} certificate issuance as provided by Let's
		Encrypt~\cite{le}.
\end{itemize}

\section{Trusted Auditor} \label{app:auditor}
An trusted auditor is expected to accept SFOs at a dedicated endpoint,
endeavoring to validate the inclusion status of each SCT with regards to the
first STH in the Tor consensus that elapsed the log's MMD.  If a log appears to
function correctly except for some evidence that cannot be
resolved despite multiple attempts that span a given time period, it is
paramount that the auditor software alerts its operator who can investigate
the issue further before submitting a full report to the CT-policy mailing list.
Cases of misbehavior, in detail:
\begin{itemize}
	\item If the log fails to provide a valid inclusion proof for an SCT with
		regards to the first applicable STH in the Tor consensus
		(certificate omission).
	\item If the log fails to provide a valid consistency proof between two
		STHs in the Tor consensus
		(split-view).
	\item If a published STH is future-dated or backdated more than an MMD
		(general log misbehavior).
	\item If a log ignores some CTRs (uptime misbehavior).
\end{itemize}

An unresponsive log can be suspected by observing the watchdog report frequency,
and possibly confirmed by querying the log in question from different vantage
points.  Among other extra-info metrics that go beyond received and deleted
SFO-bytes that the announced CT auditors should check, it could be valuable to
publish the extent to which CTR-to-log interactions fail.

While not within our threat model, we do encourage that the announced auditors
verify whether STHs in the Tor consensus are consistent with external views of
the CT landscape.  For example, operate an STH pollination
endpoint~\cite{nordberg} and fetch STHs actively from many diverse vantage
points using Tor, VPN services, DoH resolvers, and RIPE Atlas (to mention a few
options).  This goes back to the point of general ecosystem value.

\section{Flushing a Single CTR} \label{app:flush}
Let $n$ be the number of SFOs that a CTR can store in its buffer.  The
probability to sample a target SFO is thus $\frac{1}{n}$, and the probability to
not sample a target SFO is $q = 1 - \frac{1}{n}$.  The probability to not sample
a target SFO after $k$ submissions is $q^k$.  Thus, the probability to sample
the relevant buffer index at least once is $p = 1 - q^k$.  Solving for $k$ we
get: $k = \frac{\log(1 - p)}{\log(q)}$.  Substituting $q$ for $1 - \frac{1}{n}$
yields Equation~\ref{eq:flush}, which was to be shown.