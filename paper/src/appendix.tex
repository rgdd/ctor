\section{Log Operators \& Trust Anchors} \label{app:ct-trust-anchors}
The standardized CT protocol suggests that a log's trust anchors should
``usefully be the union of root certificates trusted by major browser
vendors''~\cite{ct,ct/bis}.  Apple further claims that a log in their CT program
``must trust all root CA certificates included in Apple's trust
store''~\cite{apple-log-policy}.  This bodes well for CTor:
	we assumed that the existence of independent log operators implies the
	ability to at least add certificate chains (Section~\ref{sec:base}) and
	possibly complete SFOs (Section~\ref{sec:log}) into logs that the attacker
	does not control.
Google's CT policy currently qualifies 31 logs that are hosted by
	Cloudflare,
	DigiCert,
	Google,
	Let's Encrypt, and
	Sectigo~\cite{google-log-policy}.
No log accepts all roots, but the overlap between root certificates that are
trusted by major browser vendors and CT logs increased over
time~\cite{ct-root-landscape}.  Despite relatively few independent log
operators and an incomplete root coverage, the basic and extended
cross-logging in CTor still provides value:
\begin{itemize}
	\item Even if there are no independent logs available for a certificate
		issued by some CA, adding it again \emph{to the same logs} come with
		practical security gains.  For example, if the attacker gained access to
		the secret signing keys but not the actual log infrastructures the
		mis-issued certificate trivially makes it into the public.  If the full
		SFO is added, the log operators could also notice that they were
		compromised.
	\item Most log operators only exclude a small fraction of widely accepted
		root certificates: 1--5\%~\cite{ct-root-landscape}.  This narrows down
		the possible CAs that the attacker must control by 1--2 orders of
		magnitude.  In other words, to be entirely sure that CTor would (re)add
		a mis-issued SFO to the attacker-controlled CT logs, this smaller group
		of CAs must issue the underlying certificate.  It is likely harder to
		take control of Let's Encrypt which some logs and operators exclude due
		to the sheer volume of issued certificates than, say, a smaller CA that
		law enforcement may coerce.
\end{itemize}

Browser-qualified or not, the availability of independent logs that accept the
commonly accepted root certificates provides significant ecosystem value.
Log misbehavior is mostly reported through the CT policy mailing list.  Thus, it
requires manual intervention.  Wide support of certificate chain and SCT
cross-logging allows anyone to \emph{casually} disclose suspected log
misbehavior on-the-fly.

\section{Trusted Auditor} \label{app:auditor}
An trusted auditor is expected to accept SFOs at a dedicated endpoint,
endeavoring to validate the inclusion status of each SCT with regards to the
first STH in the Tor consensus that elapsed the log's MMD.  If a log appears to
function correctly except for some evidence that cannot be
resolved despite multiple attempts that span a given time period, it is
paramount that the auditor software alerts its operator who can investigate
the issue further before submitting a full report to the CT-policy mailing list.
Cases of misbehavior, in detail:
\begin{itemize}
	\item If the log fails to provide a valid inclusion proof for an SCT with
		regards to the first applicable STH in the Tor consensus
		(certificate omission).
	\item If the log fails to provide a valid consistency proof between two
		STHs in the Tor consensus
		(split-view).
	\item If a published STH is future-dated or backdated more than an MMD
		(general log misbehavior).
	\item If a log ignores some CTRs (uptime misbehavior).
\end{itemize}

An unresponsive log can be suspected by observing the watchdog report frequency,
and possibly confirmed by querying the log in question from different vantage
points.  Among other extra-info metrics that go beyond received and deleted
SFO-bytes that the announced CT auditors should check, it could be valuable to
publish the extent to which CTR-to-log interactions fail.

While not within our threat model, we do encourage that the announced auditors
verify whether STHs in the Tor consensus are consistent with external views of
the CT landscape.  For example, operate an STH pollination
endpoint~\cite{nordberg} and fetch STHs actively from many diverse vantage
points using Tor, VPN services, DoH resolvers, and RIPE Atlas (to mention a few
options).  This goes back to the point of general ecosystem value.

\section{Flushing a Single CTR} \label{app:flush}
Let $n$ be the number of SFOs that a CTR can store in its buffer.  The
probability to sample a target SFO is thus $\frac{1}{n}$, and the probability to
not sample a target SFO is $q = 1 - \frac{1}{n}$.  The probability to not sample
a target SFO after $k$ submissions is $q^k$.  Thus, the probability to sample
the relevant buffer index at least once is $p = 1 - q^k$.  Solving for $k$ we
get: $k = \frac{\log(1 - p)}{\log(q)}$.  Substituting $q$ for $1 - \frac{1}{n}$
yields Equation~\ref{eq:flush}, which was to be shown.
