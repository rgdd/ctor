\section{Auditor Extension} \label{sec:auditor}
Until now we have not \emph{verified} whether the certificate chain of an SFO is
publicly logged as promised by the issued SCTs.  As such, it is unlikely that a
misbehaving CT log will be detected.
Our base design can be extended to follow-up
on an SFO's inclusion status rather than adding the underlying certificate chain
to another CT log.  This has the benefit of relaxing our initial trust
assumption, namely that some CT logs are honest, as well as making it possible
to disclose those CT logs that are dishonest.  The downside is added
complexity, which introduces additional threats that must be considered.

\subsection{Design Sketch} \label{sec:auditor:design}
Figure~\ref{fig:auditor} provides an overview of the extended design.  Tor
Browser submits presented SFOs probabilistically to CTRs that are selected at
random, and CTRs store the submitted SFOs before any auditing takes place.
Here, auditing refers to inclusion verification rather than adding certificate
chains.  The moment before auditing, the SFO in question is shared with a CTR
that acts as a \emph{watchdog}.  Unless the auditing CTR receives a timely
inclusion proof and acknowledges it to its watchdog, the (now suspicious) SFO is
reported to a CT auditor.  Phase~1 remains unchanged, some changes are needed
in phase~2, and major changes are required in phase~3 as well as the Tor
consensus.  The extra-info document also includes two new metrics that are
related to flooding.  Another prerequisite is the existence of CT auditors.

\begin{figure*}
    \centering
    \includegraphics[width=0.85\textwidth]{img/design-auditor}
	\caption{The auditor extension to CTor, where cryptographic evidence of log
	omission can be collected without modification to CT logs. The extension
	changes the consensus to include the latest STHs from CT logs and makes
	phase 3 significantly more complex. CTRs in phase 3 now challenge logs to
	prove inclusion of certificates from SFOs, using other CTRs as
	``watchdogs'', ensuring that SFOs that are not provably correct are reported
	to trusted auditors.}
	\label{fig:auditor}
\end{figure*}

\subsubsection{Tor Consensus} \label{sec:auditor:design:consensus}
Tor's consensus should capture a fixed view of the CT landscape by publishing
STHs from all recognized logs.  A CT log is recognized if a majority of directory
authorities proposed a \texttt{ct-log-info} item, which contains a log's ID,
public key, base URL, MMD, and most recent STH.  Note that each directory
authority proposes its own STH, and agrees to use the most recent STH as
determined by timestamp.  Since CTRs verify inclusion statuses of SCTs
that Tor Browser accepts, the CT logs recognized by Tor Browser must be in
Tor's consensus.

Tor's directory authorities also majority-vote on \texttt{ct-auditor} items,
which pin base URLs and public keys of CT auditors that watchdogs contact in
case that any log misbehavior is suspected.  A watchdog triggers if the time
specified by \texttt{ct-watchdog-timeout} elapses without receiving any
acknowledgment.  The following auditor submission is governed by a
\texttt{ct-auditor-timeout}, which, if triggered, results in a resubmission
later on.

\subsubsection{Phase~2---Storage} \label{sec:auditor:design:phase2}
Other than updating the criteria of what it means that an SFO can be audited
in phase~3, the computation of $l$ and $t$ changes for a buffer's entry.  With
regards to some CT circuit, process an incoming SFO $s$ as follows:
\begin{enumerate}
	\item\label{enm:ext:storage:close} Close the current circuit to enforce
		one-time usage.
	\item\label{enm:ext:storage:unrecognized} Discard unrecognized SCTs in $s$
		whose logs have no corresponding \texttt{ct-log-info} items listed in
		the Tor consensus.  Stop if there are no remaining SCTs in~$s$.
	\item\label{enm:ext:storage:cached}
		Stop if $s$ is cached or pending to be audited already.
	\item\label{enm:ext:storage:fix-log} Sample a CT log $l$ that issued a
		remaining SCT in~$s$.
	\item\label{enm:storage:audit-after} Compute an \texttt{audit\_after}
		time~$t$, see Figure~\ref{fig:audit-after}.
	\item\label{enm:storage:store} Add $(l,t,s)$ to a buffer of pending SFOs.
\end{enumerate}

Recall from Section~\ref{sec:background:ct} that an inclusion proof is fetched
with regards to an STH.  As such, we discard SCTs that cannot be verified due to
lack of \texttt{ct-log-info} items in the Tor consensus.  The sampled CT log $l$
now refers to an entity that issued an SCT in the submitted SFO, and it will be
challenged to prove inclusion in phase~3 sometime after the
\texttt{audit\_after} timestamp $t$ elapsed.  Figure~\ref{fig:audit-after} shows
that $t$ takes the log's MMD into account.  This is one of two parts that
prevent \emph{early signals} to the issuing CT logs that an SFO is being
audited.  For example, if an SFO is audited before the MMD elapsed, the issuing
CT log could simply merge the underlying certificate chain to avoid an MMD
violation.  This would not yield any improvement with regards to the base
design.

\begin{figure}
	\centering
	\pseudocode[linenumbering, syntaxhighlight=auto]{%
		\textrm{t} \gets \mathsf{now}() +
			\mathsf{MMD} +
			\mathsf{random}(\texttt{ct-delay-dist}) \\
		\pcif \textrm{SCT.timestamp} + \textrm{MMD} <
				\mathsf{now}():\\
			\pcind\textrm{t} \gets \mathsf{now}() +
				\mathsf{random}(\texttt{ct-delay-dist})
	}
	\caption{%
		Algorithm that computes an \texttt{audit\_after} timestamp $t$.
	}
	\label{fig:audit-after}
\end{figure}

\subsubsection{Phase~3---Auditing} \label{sec:auditor:design:phase3}
In addition to maintaining a single Tor circuit that is used to interact with
CT logs that have \texttt{ct-log-items}, a distinct Tor circuit is maintained
and rotated that ends at a random watchdog CTR.  Given a known CT log
$l$:
\begin{enumerate}
	\item\label{enm:ext:auditing:backoff} Sample a delay $d \gets
		\mathsf{random}(\texttt{ct-backoff-dist})$.
	\item\label{enm:ext:auditing:sleep} Schedule a timer, waiting until $d$
		time units elapsed.
	\item\label{enm:auditing:loop} For each pending buffer entry $(l',s,t)$:
		\begin{enumerate}
			\item\label{enm:ext:auditing:log-check}
				Continue the loop if $l\ne l'$.
			\item\label{enm:ext:auditing:timestamp-check} Continue the loop if
				$t > \mathsf{now}()$.
			\item\label{enm:ext:auditing:sth-check} Continue the loop if $t >
				\textrm{STH}.\mathsf{timestamp}$.
			\item\label{enm:ext:auditing:watchdog} Share $s$ with the current
				watchdog.
			\item\label{enm:ext:auditing:challenge} Use \texttt{ct-log-timeout}
				and $\textrm{STH}.\mathsf{treesize}$ to set a timer and
				challenge the log to prove inclusion.
				\begin{itemize}
					\item\label{enm:ext:auditing:challenge:success} On valid
						proof: send an acknowledgment to the watchdog, then
						cache $s$ and discard it.
					\item\label{enm:ext:auditing:challenge:fail} On any other
						outcome: discard $s$ and break.
				\end{itemize}
		\end{enumerate}
	\item\label{enm:ext:auditing:restart} Go back to
		step~\ref{enm:ext:auditing:backoff}.
\end{enumerate}

An SFO is not audited for inclusion until the log's MMD elapsed \emph{and} there
is an STH in the Tor consensus that captures it.  As such, a CT log that intends
to omit a certificate chain despite promising to merge it within its MMD will
not get an early signal that a CTR will audit it.
Before auditing, the SFO in question is shared with a watchdog that takes on
the responsibility of reporting suspicious SFOs to the pinned CT auditors.
An SFO is considered suspicious if it is not acknowledged by the log-challenging
CTR as verified within the time specified by \texttt{ct-watchdog-timeout}.  As
argued in Section~\ref{sec:auditor:analysis:phase3}, it is motivated to use a
watchdog:
	the attacker learns which CTR holds the problematic SFO at the time of
		auditing, and
	during the \texttt{ct-log-timeout} actions could then be taken to ensure
		that the SFO does not reach a CT auditor.
A watchdog that receives a suspicious SFO reports it to a random CT auditor,
and resubmits it later on if the \texttt{ct-auditor-timeout} happens to trigger.

\subsubsection{Extra-Info Document} \label{sec:auditor:extra-info}
Following from the \texttt{audit\_after} timestamp algorithm in
Figure~\ref{fig:audit-after}, an SFO may be stored throughout an entire MMD.
This results in a relatively large time window during which the attacker can
attempt to flood all CTRs in hope that they delete the omitted SFO at random
before it is audited.  We discuss the threat of flooding further in
Section~\ref{sec:auditor:analysis:phase2}, noting that it can be detected if
CTRs publish two new metrics in the extra-info document:
	\texttt{ct-receive-bytes} and
	\texttt{ct-delete-bytes}.
These metrics indicate how many SFO bytes were received and deleted throughout
different time intervals, which is similar to other extra-info metrics such
as \texttt{read-history}.

\subsubsection{CT Auditor} \label{sec:auditor:auditor}
An announced auditor is expected to accept SFOs at a dedicated endpoint,
endeavoring to validate the inclusion status of each SCT with regards to the
first STH in the Tor consensus that elapsed the log's MMD.  If a log appears to
function correctly except for some evidence that cannot be
resolved despite multiple attempts that span a given time period, it is
paramount that the auditor software alerts its operator who can investigate
the issue further before submitting a full report to the CT-policy mailing list.
Cases of misbehavior, in detail:
\begin{itemize}
	\item If the log fails to provide a valid inclusion proof for an SCT with
		regards to the first applicable STH in the Tor consensus
		(certificate omission).
	\item If the log fails to provide a valid consistency proof between two
		STHs in the Tor consensus
		(split-view).
	\item If a published STH is future-dated or backdated more than an MMD
		(general log misbehavior).
	\item If a log ignores some CTRs (uptime misbehavior).
\end{itemize}

An unresponsive log can be suspected by observing the watchdog report frequency,
and possibly confirmed by querying the log in question from different vantage
points.  Among other extra-info metrics that go beyond received and deleted
SFO-bytes that the announced CT auditors should check, it could be valuable to
publish the extent to which CTR-to-log interactions fail.

%While not within our threat model, we do encourage the announced auditors to
%verify that STHs in the Tor consensus are consistent with external views of the
%CT landscape.  For example, operate an STH pollination~\cite{nordberg} endpoint
%and fetch STHs actively from many diverse vantage points using Tor, VPN
%services, DoH resolvers, and RIPE Atlas (to mention a few low-cost options).
%This goes back to the point of general ecosystem value.
% => this is now a "forward point"

%TODO: all flushing and tagging details here
\subsection{Security Analysis} \label{sec:auditor:analysis}
Idea: go through what is \emph{different} with regards to the base analysis.

\subsubsection{Phase~2---Storage} \label{sec:auditor:analysis:phase2}
Main difference: longer storage time to avoid early signals.  Bound by 2*MMD +
C, where C is the longest time to get an STH into the Tor consensus.  I.e.,
attacker can maximize the window by producing an STH after MMD-1s, then wait
another MMD before making next STH which we are waiting for to make it into the
Tor consensus.  MMD usually 24h, and a consensus document is valid for 3h.  As
such, attack window is at most 51 hours without any formal log misbehavior.

Flooding can be modelled as...see equation.  Assuming 1GiB memory and a
confidence level of 0.90, 433.2k queries and roughly 2GiB bytes must be
transmitted to flush a single CTR.  Note: some relays have bandwidth in the
order of 1--5 MiB/s, which if all is used to flood, flooding still takes
takes between 3--35 minutes regardless of the attacker's available bandwidth.

The attacker does not know which CTR to flood.  As such, the entire network must
be targeted.  This results in 1.73B SFO submissions and roughly 1TiB bytes.  To
flush the network in 10m, 20m, 1h, and 24h, it requires a flooding bandwidth of
at least 142 Gbps, 71.1 Gbps, 23.7 Gbps, and 0.99 Gbps.  The announced bandwidth
of the Tor network is 450 Gbps, and the announced exit bandwidth is roughly 110
Gbps.

TODO: it is certainly DoS within minutes if we do Tor circuit, coming out from
an exit to a CTR.  And assuming CTRs checked that submission circuits come from
an exit relay.  Can also possibly draw parallel to historical attacks, e.g.,
attack on root DNS servers was around 35 Gbps in 2015 (before amplification).

TODO: wrap-up, attacker can flush the entire network without DoS:ing it.  It
would be detected by the extra-info document, but hard to attribute to the
actual attacker.  Implies minor impact scenario.  

TODO: double check numbers in more detail, extracted quickly from our github
page.

\subsubsection{Phase~3---Auditing} \label{sec:auditor:analysis:phase3}
Now we are talking to the attacker.  Attacker learns that a mis-issued SFO
is being audited.  Auditing happens over a shared Tor circuit, and despite the
anonymity that it provides, the attacker can overcome it by tagging.  Briefly
explain tagging.

If the CTR submitted to a CT auditor, it would leave a ~seconds window where
the attacker could simply DoS an identifiable CTR.  Therefore, we delegated
reporting to a random watchdog \emph{before} doing the inclusion query.
The attacker does not learn the watchdog identity unless we sample the
attacker's CTR.  We do not consider the threat of fully taking over the Tor
relay within seconds to identify the watchdog; too strong for Tor's threat
model.

% MISC notes
% - Network-wide flush, detectable but hard to attribute
% - Requires new reliable auditor software
% - Bit more bandwidth due to watchdog.  The overhead, when compared to log a
% log extension, is sending an SCT hash and receiving a proof (2-3KiB).
% - A rational attacker is the issuer of all SCTs in the omitted SFO, so it is
% sufficient to verify inclusion with regards to one SCT and then leave it to
% the auditors to bust all involved CT logs.

