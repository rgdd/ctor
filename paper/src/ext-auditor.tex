\section{Auditor Extension} \label{sec:auditor}
Until now we have not verified whether a certificate chain is logged as promised
by any issued SCT.  Our base design can be extended to follow-up on inclusion
statuses rather than opting for cross-logging.  In terms of
the number of necessary changes, this extension is more significant when
compared to Section~\ref{sec:log}.  However, it does not require any log
modifications, and it entails additional ecosystem value by providing a
well-audited view of the CT landscape which is captured by the Tor consensus.
It is possible to detect internal inconsistency attacks within the Tor
network and omission attacks.  Trust in some logs is also shifted towards CT
auditors.

\subsection{Design Sketch}
Figure~\ref{fig:auditor} provides an overview of the extended design.  Tor
Browser submits presented SFOs probabilistically to CTRs that are selected at
random, and CTRs mix the submitted SFOs before any auditing takes place.  Here,
auditing refers to inclusion verification rather than cross-logging.  The moment
before an SFO is audited, it is shared with a watchdog-CTR that is reused during
an entire auditing instance.  Unless an SFO is acknowledged as verified within a
timely manner by the log-challenging CTR, it is submitted by the watchdog to a
CT auditor.  Phase~1 remains unchanged, phase~2 needs minor changes, and phase~3
major changes.  Flushing statistics are also required in the extra-info
document, as well as additional parameters in the Tor consensus.  A final
prerequisite is that there are CT auditors available.

\begin{figure*}
    \centering
    \includegraphics[width=0.85\textwidth]{img/design-auditor}
    \caption{todo}
	\label{fig:auditor}
\end{figure*}

\subsubsection{Tor Consensus}
Tor's consensus should capture a fixed view of the CT landscape by publishing
STHs from all recognized logs.  A log is recognized if a majority of directory
authorities proposed a \texttt{ct-log-info} item, which contains a log's ID,
public key, base URL, MMD, and most recent STH.  Note that each directory
authority proposes its own STH, and agrees to use the most recent STH as
determined by timestamp.  Since CTRs verify inclusion statuses of SCTs
that Tor Browser accepts, the CT logs recognized by Tor Browser must be in
the consensus.

Tor's directory authorities also majority-vote on \texttt{ct-auditor} items,
which pin base URLs and public keys of CT auditors that watchdogs contact in
case that any log misbehavior is suspected.  A watchdog triggers if the
\texttt{ct-watchdog-timeout} elapses without acknowledgment.  The following
auditor submission times-out if \texttt{ct-auditor-timeout} elapses.  These
items are determined similar to the \texttt{ct-query-timeout} in
Section~\ref{sec:base:consensus:params}.

\subsubsection{Phase~2: Storage}
A CTR cannot verify an SCT from an unrecognized log.
Step~\ref{enm:ctr-api:unrecognized} should therefore be updated so that
unrecognized SCTs are discarded, stopping if no SCTs remain in the resulting
SFO.  If an SFO is neither cached nor pending, sample an SCT and note down the
outcome and a corresponding \texttt{audit\_after} timestamp using
Equation~\ref{eq:audit-after}.  The combination of SFO, sampled SCT outcome, and
\texttt{audit\_after} timestamp is stored in the pending buffer.

\subsubsection{Phase~3: Auditing}
The detailed auditing steps change to a large extent.  As such, we give a
detailed description from scratch.

\begin{enumerate}
	\item\label{enm:ctr:audit:backoff} Sample a uniform delay from
			$[0, \texttt{ct-backoff}]$,
		then schedule a timer and wait until that time elapsed.
	\item\label{enm:ctr:audit:log-circuit} Establish a new circuit and use it
		for all subsequent log connections.  Connect to the logs when needed.
	\item\label{enm:ctr:audit:watchdog-circuit} Establish a new three-hop
		circuit that ends at a randomly selected watchdog, i.e., another CTR.
	\item\label{enm:ctr:audit:loop} Enumerate the buffer of pending SFOs with
		regards to the sampled SCT and its associated STH:
		\begin{enumerate}
			\item\label{enm:ctr:audit:too-soon} Continue the loop if
				$\textrm{SFO}.\mathsf{audit\_after} > \mathsf{now}()$.
			\item\label{enm:ctr:audit:too-soon2} Continue the loop if
				$\textrm{STH}.\mathsf{timestamp} > \mathsf{now}()$.
			\item\label{enm:ctr:audit:watchdog} Submit $s$ to the
				watchdog selected in step~\ref{enm:ctr:audit:watchdog-circuit}.
			\item\label{enm:ctr:audit:challenge} Use \texttt{ct-query-timeout}
				and $\textrm{STH}.\mathsf{treesize}$ to set a timer and
				challenge the log to prove inclusion.
				\begin{itemize}
					\item On valid proof: send an acknowledgment to the
						watchdog, then cache $s$ and discard it.
					\item On any other outcome: discard $s$ and break.
				\end{itemize}
		\end{enumerate}
	\item\label{enm:ctr:audit:teardown} Close all opened circuits and go back to
		step~\ref{enm:ctr:audit:backoff}.
\end{enumerate}

Each auditing instance uses a single watchdog, which takes on the responsibility
of reporting suspected log misbehavior on any other outcome that valid inclusion
proof.  It is motivated to use a watchdog because the attacker learns which
CTR audits an SFO, and could likely take actions to intervene while the CTR
waits for \texttt{ct-submit-timeout} to trigger (see
Section~\ref{sec:ext-auditor:analysis}).  A watchdog's behavior is to listen for
SFO submissions on a dedicated endpoint, and submit them to a random CT auditor
that is pinned in the Tor consensus if no timely acknowledgment is received.  If
an auditor is unavailable as indicated by further timeouts, resubmit later on.

An SFO's sampled SCT is not audited before the log's MMD elapsed \emph{and}
there is an STH in the Tor consensus that captured it.  Note that it is
sufficient for CTRs to verify the inclusion status of a single SCT:
	the attacker has to omit logging of all SCTs regardless, and
	if log misbehavior is suspected the full SFO is submitted to CT auditors
		that can follow-up on all SCTs.

\subsubsection{Auditor} \label{sec:ext-auditor:auditor}
An announced auditor is expected to accept SFOs at a dedicated endpoint,
endeavoring to validate the inclusion status of each SCT with regards to the
first STH in the Tor consensus that elapsed the log's MMD.  If a log appears to
function correctly except for some evidence that cannot be
resolved despite multiple attempts that span a given time period, it is
paramount that the auditor software alerts its operator who can investigate
the issue further before submitting a full report to the CT-policy mailing list.
Cases of misbehavior in detail:
\begin{itemize}
	\item If the log fails to provide a valid inclusion proof for an SCT with
		regards to the first applicable STH in the Tor consensus
		(certificate omission).
	\item If the log fails to provide a valid consistency proof between two
		STHs in the Tor consensus
		(split-view).
	\item If a published STH is future-dated or backdated more than an MMD
		(general log misbehavior).
	\item If a log ignores some CTRs (uptime misbehavior).
\end{itemize}

An unresponsive log can be suspected by observing the report frequency, and
possibly confirmed by querying the log in question over a diverse set of Tor
circuits.  Among other \emph{good to have} extra-info that go beyond received
and deleted SFO-bytes, it would be valuable to publish CTR failure rates while
challenging the logs.

While not within our threat model, we do encourage the announced auditors to
verify that STHs in the Tor consensus are consistent with external views of the
CT landscape.  For example, operate an STH pollination~\cite{nordberg} endpoint
and fetch STHs actively from many diverse vantage points using Tor, VPN
services, DoH resolvers, and RIPE Atlas (to mention a few low-cost options).

\subsection{Security Sketch} \label{sec:ext-auditor:analysis}
% MISC notes
% - Network-wide flush, detectable but hard to attribute
% - Requires new reliable auditor software
% - Bit more bandwidth due to watchdog.  The overhead, when compared to log a
% log extension, is sending an SCT hash and receiving a proof (2-3KiB).
