\section{Privacy} \label{sec:privacy}
There is an inherent privacy problem in the setting due to how CT is designed
and deployed. A browser, like Tor Browser, that wishes to validate that SFOs presented to
it are \emph{consistent} and \emph{included} in CT logs must directly or
indirectly interact with CT logs wrt. its observed SFOs. Without protections
like Private Information Retrieval~\cite{PIR} that require server-side support
or introduction of additional parties and trust
assumptions~\cite{lueks-and-goldberg,kales}, exposing SFOs to any party risks
leaking (partial) information about the browsing activities of the user.

Given the constraints of the existing CT ecosystem, CTor is made
privacy-preserving thanks to the distributed nature of Tor with its anonymity
properties and high-uptime relays that make up the Tor network. First, all
communication between Tor Browser, CTRs, CT logs, and auditors are made over full
Tor-circuits. This is a significant privacy-gain, not available, e.g., to
browsers like Chrome that in their communications would reveal their public
IPv4-address (among a number of other potentially identifying metadata).
Secondly, the use of CTRs as intermediaries probabilistically delays the
interaction with the CT logs---making correlating Tor Browser user browsing with CT log
interaction harder for attackers---and safely maintains a dynamic cache of the
most commonly already verified SFOs. While regular browsers like Chrome could
maintain a cache, Tor Browser's security and privacy goals (see
Section~\ref{sec:background:tor}) prohibit such shared (persisted) dynamic
state.

CTRs are also essential for security in the setting of the auditor extension. A
rational attacker will produce SFOs with SCT timestamps that are too new to have
to be part of CT logs at the time of the attack. Tor Browser cannot store SFOs for any
extended time period without violating its security and privacy goals, not to
mention that presumably most Tor Browser sessions are not maintained for
days (as required, see analysis in Section~\ref{sec:auditor:analysis}) 
{\bf \color{red} What analysis is this pointing at?}\\
but rather minutes~\cite{DBLP:conf/pam/AmannS16}. Further, buffering
at Tor Browser is not an option, since an attacker could compromise Tor Browser shortly
after presenting the forged SFO.

The main limitation of CTor in terms of privacy is that CTor continuously leaks
to CT logs---and to a \emph{lesser extent} auditors (depending on design)--a
fraction of certificates of websites visited using Tor Browser to those that operate CT
logs. This provides to a CT log a partial list of websites visited via the Tor
network over a period of time (determined by \texttt{ct-delay-dist}), together
with some indication of distribution based on the number of active CTRs. It does
not, however, provide even pseudonymously any information about which sites
individual users visit, much less with which patterns or timing. As such it
leaks significantly less information than does OCSP validation by Tor Browser or DNS
resolution at exit-relays~\cite{TorDNS}, both of which indicate visit activity
in real time to a comparably small number of entities.

Another significant privacy limitation is that relays with the CTR flag learn
real-time browser behavior of Tor users. Relays without the Exit flag primarily
only transport encrypted Tor-traffic between clients and other relays, never to
destinations. If such relays are given the CTR flag---as we stated in the full
design, see Section~\ref{sec:base:consensus:ctr-flag}---then this might
discourage some from running Tor relays unless it is possible to opt out.
Another option is to give the CTR flag only to exit relays, but this \emph{might
be} undesirable for overall network performance despite the modest overhead of
CTor (Section~\ref{sec:performance}). Depending on the health of the network
and the exact incremental deployment of CTor, there are different trade-offs.
For example, fewer CTRs with higher bandwidth and less memory combined make it
easier to perform flooding attacks that flush the entire network.
