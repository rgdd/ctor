\section{Privacy} \label{sec:privacy}

\subsection{Privacy and CTRs}
There is an inherent privacy problem in the setting due to how CT is designed. A
browser, like TB, that wishes to validate that SFOs presented to it are
\emph{consistent} and \emph{included} in CT logs must directly or indirectly
interact with CT logs wrt. its observed SFOs. Without protections like Private
Information Retrieval (PIR), exposing shared SFOs to any party leaks partial
information about the browsing activities of the user.

Given the constraints of the existing CT ecosystem, CTor is made
privacy-preserving mainly in two ways. First, all communication between TB,
CTRs, CT logs, and auditors are made over full Tor-circuits. This is a
significant privacy-gain, not available, e.g., to browsers like Chrome that in
their communications would reveal their public IPv4-address (among a number of
other potentially identifying metadata). Secondly, the use of CTRs as
intermediaries probabilistically delays the interaction with the CT
logs---making correlation harder---and can safely maintain a dynamic cache of
the most commonly already verified SFOs. While regular browsers like Chrome
could maintain a cache, TB and its security and privacy goals (see
Section~\ref{sec:background:tor}) prohibit such persisted dynamic state.

CTRs are also essential for security in the setting of the auditor extension. A
rational attacker will produce SFOs with SCT timestamps that are too new to have
to be part of CT logs at the time of the attack. TB cannot store SFOs for any
extended period of time (a couple of days, see
Section~\ref{sec:discussion:flooding}) without violating its security and
privacy goals, not to mention that presumably most TB sessions are not
maintained for days. Further, buffering at TB is not an option, since an
attacker could compromise TB shortly after presenting the forged SFO.

In gist, the distributed nature of Tor with its anonymity properties and
high-uptime relays that make up the Tor network are essential for
privacy-preserving auditing of the current CT ecosystem. The auditor extension
turns Tor into a system for maintaining a probabilistically-audited
cryptographically-verifiable view of the entire CT log ecosystem available from
Tor’s consensus. In a sense, this system is a consensus mechanism for CT logs.

The main limitation of CTor in terms of privacy is that CTor continuously leaks
to CT logs, and to auditors in the case of the auditor extension
(Section~\ref{sec:auditor}), a fraction of certificates of websites
visited using TB to the relatively small number of operators of CT
logs.  This provides to a CT log a partial list of websites visited
via the Tor network over a period of a few days, together with some
indication of distribution based on the number of active CTRs. It does
not, however, provide even pseudonymously any information about which
sites individual users visit, much less with which patterns or timing.
As such it leaks significantly less information than does OCSP
validation by TB or DNS resolution at exit-relays~\cite{TorDNS}, both
of which indicate visit activity in real time.

