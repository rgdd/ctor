\section{Security Analysis} \label{sec:security}
Recall from Section~\ref{sec:adversary} that the adversary's goal can be
broken down into two separate threats that our design must address:
	split-views of CT logs within the Tor network, and
	omission attacks against Tor Browser users.

\subsection{Split-Views}
Given that directory authorities fetch STHs from attacker-controlled logs and
agree upon which ones to use via deterministic rules, the attacker is in full
control of which STHs go into the Tor consensus.  Once an STH is
announced, it follows from Tor's threat model that it is fixed because a
threshold of directory authorities are benign.  As such, CTRs have access to
the same set of STHs and thus they also get the same (in)consistent view of the
CT landscape.  Any inconsistent view that makes it into the Tor consensus is
trivially detected, however, as all STHs are public and can be audited for
consistency by any interested party.  The attacker should therefore be
deterred from creating internal split-views within the Tor network:
	at least one announced auditor would detect it and make the log's
	misbehavior public.

\subsection{Omission Attacks}
To hide an omission the attacker must either play the lucky-game or intervene
with an SFO \emph{somewhere} on the path between Tor Browser and an announced
auditor.  We already discussed attacker risk with regards to both assumptions
and security parameters, e.g., the fraction of malicious CTRs and
\texttt{ct-submit-pr}.  In what follows we shift our focus towards the
intervention timeline in Figure~\ref{fig:sfo-intervention}, analyzing the
attacker's options with regards to what they \emph{know} and how that knowledge
may be \emph{used} to keep an omitted SFO outside of the public domain.

\begin{figure}
	\centering
	TODO: timeline figure on SFO intervention
	\caption{%
		docdoc
	}
	\label{fig:sfo-intervention}
\end{figure}

\subsubsection{Tor Browser $\rightarrow$ CTR} \label{sec:security-analysis:tb}
Recall that an SFO is submitted to a CTR based on a biased coin-flip.  If
the attacker is (not) aware of the outcome results in four scenarios
that can be considered:
\begin{enumerate}
	\item SFO submitted; the attacker does not know it.
	\item SFO not submitted; the attacker does not know it.
	\item SFO submitted; the attacker knows it.
	\item SFO not submitted; the attacker knows it.
\end{enumerate}

The first two outcomes refer to ideal scenarios where the attacker gained no
additional insight regarding the submission status of an SFO.  In contrast,
the latter two outcomes refer to an attacker that gained an advantage that may
be used to intervene with an SFO (our focus).

The third outcome occurs if the attacker controls the sampled CTR or if
submission information can be inferred from traffic analysis.  Fortunately, none
of these options are \emph{reliable enough} for a risk-averse attacker;
	such an attacker controls too much of the network and is therefore not
	within Tor's threat model.
If the attacker cannot rely on its network-related capabilities to infer whether
and where a given SFO is submitted, Tor Browser may be targeted instead.  In
particular, it might be possible to serve a zero-day exploit while engaging in
man-in-the-middle activities that takes control of Tor Browser once the SFO in
question is accepted as valid.  If the attacker's exploit runs reliably after
transmitting the SFO to a CTR but before the submission circuit is closed,
the submission status can be confirmed from circuit metadata and the CTR in
question is identified.

With regards to the attacker's knowledge, the fourth outcome can be viewed as a
stringent version of the third outcome.  In other words, if network-related
capabilities are insufficient to reliably attribute an SFO submission to some
CTR it is also too unreliable to determine the complement event of no submission
at all.  The attacker may have a zero-day exploit that runs reliably before
an SFO is submitted to a CTR, in which case the attacker successfully
intervenes and need not worry any further.

\textbf{Conclusion:}
it is paramount that the time window that the attacker's exploit must run within
is minimized.
%This is the main reason why submission circuits are used at most once and
%closed immediately thereafter.

\subsubsection{CTR Before Auditing}
Now we consider the scenario that Tor Browser audited an SFO further and some
CTR received it successfully, distinguishing between two complementary outcomes:
\begin{enumerate}
	\item Attacker knows where an SFO is located.
	\item Attacker does not know where an SFO is located.
\end{enumerate}

The window that that the attacker can intervene before an SFO is audited by a
CTR is governed by its \texttt{audit\_after} timestamp.  The attacker controls
the log's MMD and the SCT's timestamp, which means that the minimum value of the
\texttt{audit\_after} timestamp is also controlled.  To maximize the attack
window to at least an MMD, the attacker should use a newly issued SCT.

The first outcome occurs if the attacker happened to succeed with the third
outcome in Section~\ref{sec:security-analysis:tb}.
As the attacker either controls the sampled CTR or were able to identify it,
the SFO in question can simply be discarded or intervened with by other means.
For example, it is well within Tor's threat model to DoS a single CTR.  It is
therefore paramount that the attacker \emph{does not know} where an SFO is
located, so that the second outcome must be \emph{assumed} even if an SFO is
not audited.

The second outcome requires the attacker to treat every genuine CTR as if it
holds a problematic SFO.  While it is possible to DoS some CTRs, we are no
longer within Tor's threat model if it is possible to DoS all CTRs.  A less
powerful and thus limited kind of DoS is to pose as a genuine Tor Browser
client, interacting with CTRs through their open SFO endpoints to \emph{flush}
them.  A \emph{network-wide flush} is possible (see Section~\ref{sec:todo}), but
at the same time detectable:
	the number of circuits in the Tor network would explode because SFOs
	are submitted separately, and
	CTRs publish flushing statistics in their extra-info documents.
While it is nontrivial to attribute flushing activities to an identifiable
attacker, being able to confirm the \emph{lack of flushing} is an invaluable
property.

\textbf{Conclusion:}
it is paramount that the attacker does not know where an SFO is located before
it is audited, and the best we can do is to detect network-wide flushes.

\subsubsection{CTR While Auditing}
At this point we can assume that the attacker does not control the CTR in
question (as any SFO could simply be discarded) and any attempt to flush the
network is deemed too visible.  Instead, the attacker may wait for an
SFO to be audited and act thereafter.  In particular, our design makes no
attempt to prevent tagging attacks, which means that once a target SFO is
audited the CTR in question can be identified despite using a full Tor circuit.
This leaves a window that is decided by the \texttt{ct-query-timeout} during
which the CTR is waiting for a response from the attacker-controlled log and
before it is submitted to an announced auditor.  As shown in Section~\ref{todo},
this window should be in the order of seconds and thus it is too narrow for any
flushing activities.

\textbf{Conclusion:}
the \texttt{ct-query-timeout} parameter must be minimized so that the attacker
cannot reliably DoS an arbitrary CTR within the resulting window.

\subsubsection{CTR After Auditing}
The final opportunity to intervene once the \texttt{ct-query-timeout} triggers
is to target the announced auditors.  Our design does nothing to enhance
auditor availability.  As such, we rely on there being a diverse set of
auditors that are difficult to DoS all at once.  Notably it might be the case
that the attacker can win some time by making some auditors unavailable and thus
forcing CTR-to-Auditor resubmissions, but at the same time it is not
\emph{reliable} since CTRs sample their auditors.

\textbf{Conclusion:}
auditor availability is likely the weakest point in practise.  The
announced auditors should therefore employ best practises for DoS prevention.

%%% Local Variables: 
%%% mode: latex 
%%% TeX-master: "../main"
%%% End:          
