%
% emph focus on normal-case behavior, and use distributions that mean auditing
% delay is avg ~10m.  Risk averse attacker would still need to assume shorter
% storage, i.e., success 50\% not enough.
%
\section{Performance} \label{sec:performance}
Our performance evaluation estimates the expected overhead as the base design of
CTor is instantiated.  Mani~\emph{et~al.} showed that up to 140~million websites
are visited over Tor on a daily basis~\cite{mani}.  Our analysis also needs to
know the size of a typical SFO, as well as the distribution of presented SFOs
per website.  To this end we collected a dataset based on the most popular
webpages submitted to Reddit (r/frontpage, all time) as of December 4, 2019.
% \url{https://github.com/pylls/padding-machines-for-tor/commit/353bfa75e9f7d6aa0a1dff9516ff234cbf0f4562}
An average certificate chain is 5440~bytes, and is seldom accompanied by more
than a few SCTs.  As such, we assume that a typical SFO is 6~KiB.  No
certificate chain exceeded 20KiB, and it is likely a conservative value for
\texttt{ct-max-sfo-bytes} that avoids blocking in the TLS handshake.  The
average number of SFOs per website was seven, which might be an overestimate due
to collecting the dataset with fresh Chromium instances and NetLog
\emph{without} filtering SFOs related to the initial call-home on start-up
behavior.

Now we take a closer look at circuit, bandwidth, and memory overhead using a
modest 10\% submission probability and an average 10~minute auditing delay at
CTRs.  The positive effect of CTR caching is disregarded for simplicity, or,
viewed from a different perspective:
	we can already show that the incurred CTor overhead is acceptable and/or
	insignificant from basic sketching.

\textbf{Circuit overhead.}
Equation~\ref{eq:sub-oh} shows the average circuit overhead for Tor Browser
over time, where $p$ is the submit probability and $\mathcal{D}$ a distribution
describing how many SFOs are presented per website visit.

\begin{equation} \label{eq:sub-oh}
	%f(p,\mathcal{D}) =
		\frac{p}{n} \sum_{i=1}^{n} c_i, \textrm{where } c_i\sample\mathcal{D}
\end{equation}

Using our submission probability and approximated $\mathcal{D}$ with $n \gets
8858$ data points, the circuit overhead is $0.70$ on average.  In contrast to
these short-lived circuits, each CTR also maintains a single long-lived circuit
at all times to interact with logs in the CT landscape.

\textbf{Bandwidth overhead.}  An SFO that is audited further gets submitted to
a CTR over a Tor circuit.  Later on, the underlying certificate chain is added
to an independent CT log using another Tor circuit.  This means that the
bandwidth of six relays are involved, one of which must be an exit.  Given the
daily website visits of Mani~\emph{et~al.} as well as our submission probability
and SFO distribution, this amounts to 98 million SFO submissions and thus
560.8~GiB per day.  Converted into bits per second and taking six Tor relays
into account yields $334.5$~Mbps in total.  Such order of overhead is small when
compared to Tor's capacity: 450~Gbps~\cite{tor-bandwidth}.

\textbf{Memory overhead.}
During the average storage time of 10~minutes, there are 680.6~k SFO submissions
that are distributed randomly across more than 4000 CTRs.  This results in an
average buffer size of 170~SFOs, which corresponds to $1.0$~MiB memory.  Such
order of overhead is small when compared to the recommended relay configuration:
	at least 512~MiB~\cite{relay-config}.
Note that the expected memory overhead does not increase much even if the
storage phase is extended for \emph{newly issued} SFOs as in
Section~\ref{sec:auditor}.  For example, suppose that all SFOs had 90~day
lifetimes due to being issued by Let's Encrypt~\cite{le}.  On average, 1.1\% of
the submitted SFOs must then be stored for the order of an MMD rather than
10~minutes.
