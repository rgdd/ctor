%
% emph focus on normal-case behavior, and use distributions that mean auditing
% delay is avg ~10m.  Risk averse attacker would still need to assume shorter
% storage, i.e., success 50\% not enough.
%
\section{Performance} \label{sec:performance}
Our performance evaluation estimates the expected overhead as the base design of
CTor is instantiated.  Mani~\emph{et~al.} showed that up to 140~million websites
are visited over Tor on a daily basis~\cite{mani}.  Our analysis also needs to
know the size of a typical SFO, as well as the distribution of presented SFOs
per website.  To this end we collected a dataset based on the most popular
webpages submitted to Reddit (r/frontpage, all time) as of December 4, 2019.
% \url{https://github.com/pylls/padding-machines-for-tor/commit/353bfa75e9f7d6aa0a1dff9516ff234cbf0f4562}
An average certificate chain is 5440~bytes, and is seldom accompanied by more
than a few SCTs.  As such, we assume that a typical SFO is 6~KiB.  No
certificate chain exceeded 20KiB, and it is likely a conservative value for
\texttt{ct-max-sfo-bytes} that avoids blocking in the TLS handshake.  The
average number of SFOs per website was seven, which is an overestimate due to
collecting the dataset with fresh Chromium instances and NetLog \emph{without}
filtering SFOs related to the initial call-home on start-up behavior.  Now we
take a closer look at circuit, bandwidth, and memory overhead using a modest
submit probability $\texttt{ct-submit-pr} \gets \frac{1}{10}$ and an
average auditing delay of 10~minutes at CTRs.  The effect of CTR caching is
disregarded.  In other words, our focus is on upper bounds.

\textbf{Circuit overhead.}
Equation~\ref{eq:sub-oh} shows the average circuit overhead for Tor Browser
over time, where $p$ is the submit probability and $\mathcal{D}$ a distribution
describing how many SFOs are presented per website visit.

\begin{equation} \label{eq:sub-oh}
	%f(p,\mathcal{D}) =
		\frac{p}{n} \sum_{i=1}^{n} c_i, \textrm{where } c_i\sample\mathcal{D}
\end{equation}

Using $p \gets \frac{1}{10}$ and our approximated $\mathcal{D}$ with $n \gets
8858$ data points, the average circuit overhead is $0.70$.  In contrast to these
short-lived circuits, each and every CTR maintains a long-lived circuit for
log interactions.

\textbf{Bandwidth overhead.}  An SFO that is audited further gets submitted to
a CTR over a Tor circuit.  Later on, the underlying certificate chain is added
to an independent CT log using another Tor circuit.  This means that the
bandwidth of six relays are involved, one of which must be an exit.  Given 140~M
daily website visits and our submit probability as well as SFO distribution,
this amounts to 98~M SFO submissions and thus 560.8~GiB per day.  Viz.,
$334.5$~Mbps in total.  Such order of overhead is negligible when compared to
the advertised capacity of Tor:
	at least 450~Gbps~\cite{tor-bandwidth}.

\textbf{Memory overhead.}
During the average storage time of 10~minutes, there are 972~k website visits
and thus 680~k SFO submissions.  These are distributed randomly across more than
4000 CTRs, resulting in an average buffer size of 170~SFOs and $1.0$~MiB memory.
Such order of overhead is negligible when compared to the recommended relay
requirements:
	at least 512~MiB~\cite{relay-config}.

Note that the expected memory overhead does not increase much even if the
storage phase is extended for \emph{newly issued} SFOs.  For example, suppose
that all SFOs had a 90~day lifetime due to being issued by Let's
Encrypt~\cite{le}.  On average, 1.1\% of the submitted SFOs must be retained for
the order of an MMD rather than 10~m.
