\section{Threat Model} \label{sec:adversary}
We consider a strong attacker who is targeting all or a subset of users visiting
a particular website over Tor. It is generally difficult to perform a targeted
attack on a single particular Tor user because one needs to identify the user's
connection before performing the attack---something that Tor's
anonymity properties frustrate.
However, it is not difficult to perform an attack on all or a subset of unknown
users of a particular service. A network vantage point to perform such an attack
is easily obtained by operating an exit relay (for a subset of Tor users) or by
compromising the network path of multiple exit relays or the final destination.
Once so positioned, the encrypted network traffic can be intercepted using a
fraudulent certificate and associated SCTs.  The subsequent attack on decrypted
network traffic
may be passive (to gather user credentials or other information) or active.
Typical examples of active attacks are to change cryptocurrency addresses to
redirect funds to the attacker or to serve an exploit to the user's browser for
\emph{user deanomymization}. Without the ability to intercept encrypted traffic,
these attacks become more difficult as the web moves towards deprecating
plaintext HTTP.

All of the components of such an attack have been seen in-the-wild
numerous times. Untargeted attacks on visitors of a particular website
include Syria's interception of Facebook traffic using a self-signed
512-bit RSA key in ~2011~\cite{syria-facebook-mitm}, Iran's
interception of Bing and Google traffic using the Diginotar
CA~\cite{diginotar,ct/a}, and the 2018 MyEtherWallet
self-signed certificate that was used as part of a BGP
hijack~\cite{ethereum-hijack-isoc}.  The latter is also an example of
redirecting routing as part of an attack (either suspected or
confirmed). Other examples of this are Iran hijacking prefixes of
Telegram (an encrypted messaging application) in
2018~\cite{iran-telegram-bgp}, another attack on cryptocurrency in
2014 this time targeting unencrypted mining
traffic~\cite{bgp-hijacking-for-crypto},
and hijacks that may have been intelligence-gathering (or honest
mistakes) including hijacks by Russian ISPs in 2017 and China Telecom
in 2018 and 2019~\cite{wiki-bgp}.  Finally, there are several examples of 
law enforcement serving exploits to Tor Browser users to de-anonymize and
subsequently arrest individuals~\cite{forbes-fbi-tor,doj-fbi-tor}.

With
the attacker's profile in mind, we consider someone that controls
	a CA,
	enough CT logs to pass Tor Browser's SCT-centric CT policy, 
	some Tor clients, and
	a fraction of Tor relays.
For example, it is possible to
	issue certificates and SCTs,
	dishonor promises of public logging,
	present split-views at will,
	intercept and delay traffic from controlled exit relays as well as CT logs,
		and
	be partially present in the network.
This includes a weaker attacker that does not \emph{control} CAs and CT logs,
but who \emph{gained access} to the relevant signing keys~\cite{turktrust,%
gdca1-omission}.  A modest fraction of CTor entities can be subject to DoS, but
not everyone at once and all the time.  In other words, we consider the threat
model of Tor and Tor Browser as a starting point~\cite{tor,tor-browser}.  Any
attacker that can reliably disrupt CT and/or Tor well beyond Tor's threat
model is therefore not within ours.

Given that we are in the business of enforcing CT, the attacker needs to hide
mis-issued certificates and SCTs from entities that audit the CT log ecosystem.
As described in Section~\ref{sec:background:ct}, this can either be achieved by
omission or split-view attacks.  Our intended attacker is clearly powerful and
may successfully issue a certificate chain and associated SCTs without detection
some of the time, but a CA caught in mis-issuance or a CT log that violated an
MMD promise will no longer be regarded as trusted.  Therefore, we assume a
\emph{risk-averse} attacker that above a relatively low probability of detection
would be deterred from engaging in such activities. Note that the goal of
\emph{detection} is inherited from CT's threat model, which aims to remedy
certificate mis-issuance \emph{after the fact}; not prevent it~\cite{ct/a}.

We identify and analyze specific attack vectors that follow from our threat
model and design as part of the security analysis in Section~\ref{sec:analysis},
namely, attack vectors related to timing as well as relay flooding and tagging.
