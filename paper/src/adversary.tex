\section{Threat Model} \label{sec:adversary}
%
% Attacker capabilities
%
We consider an attacker that controls
	a CA,
	enough CT logs to pass Tor Browser's SCT-centric CT policy, 
	some Tor clients, and
	a fraction of Tor relays.
For example, it is possible to
	issues certificates and SCTs,
	dishonor promises of public logging,
	present split-views at will,
	intercept and delay traffic from controlled exit relays as well as CT logs,
		and
	be partially present in the network.
This includes a weaker attacker that does not \emph{control} CAs and CT logs,
but, who \emph{gained access} to the relevant signing keys~\cite{turktrust,%
gdca1-omission}.  A modest fraction of CTor entities can be subject to DoS, but
not everyone at once and all the time.  In other words, we consider the threat
model of Tor and Tor Browser as a starting point~\cite{tor,tor-browser}.  Any
attacker that can reliably disrubt CT and/or Tor to such a large extent is
therefore not accounted for in our work.
% TODO: not sure if we need to emph "it assumed that the attacker knows all Tor
% relay configurations, e.g., available memory and setup of torlog". now (?).

%
% Attacker goal and mindset
%
Given that we are in the business of enforcing CT, the attacker needs to hide
mis-issued TLS certificates and SCTs from entities that audit the CT landscape.
As described in Section~\ref{sec:background:ct}, this can either be achieved by
omission or slit-view attacks.  Our intended attacker is clearly powerful and may
successfully issue a certificate chain and associated SCTs without detection
some of the time, but, a CA caught in mis-issuance or a CT log that violated an
MMD promise will no longer be regarded as trusted.  Therefore, we assume a
\emph{risk-averse} attacker that above a relatively low probability of detection
would be deterred from engaging in such activities.

%
% - Our goal: detect TB HTTPS MitM
% - Why: it is a reasonable prerequisite to conduct attacks against TB users
%
We want to minimize the existence of successful man-in-the-middle attacks
against Tor Browser where it cannot be established if
and how they were carried out.  The property of \emph{detection} is inherited
from CT's threat model, which aims to remedy certificate mis-issuance
\emph{after the fact}; not prevent it~\cite{ct/a}.  We
focus on HTTPS traffic that Tor Browser generated because Tor is used to browse
the normal (encrypted) web~\cite{mani}.  Note that it is generally
\emph{difficult} to target a specific Tor Browser user due to Tor's anonymity,
circuit isolation, and HTTPS everywhere.  However, targeting some or all 
users that visit a website is an eminent threat:
	simply intercept traffic from an exit relay or
	upstream of the website.
Once network traffic can be intercepted, it is trivial to serve an exploit to a
subset of Tor Browser users and much easier to target an identifiable user.
Such zero-day exploits are considered in our threat model because \emph{user
exploitation} is the primary reason to attack Tor Browser.

%
% Defer introducing threats that follow from design details
%
We identify additional threats that follow from our adversary model and design
in the security analysis.  Namely, attack vectors related to
	timing (Section~\ref{sec:analysis:pr:phase1}) as well as
	Tor relay tagging and flooding (Section~\ref{sec:auditor:analysis}).
