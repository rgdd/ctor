\section{Adversary Model}
\label{sec:adversary}

The primary goal of our adversary is to undetectably engage in
misbehavior at TLS certificate authorities (CAs) and Certificate
Transparency (CT) logs so that clients are deceived about which
certificates are both valid and included in CT logs.  The two versions
of the primary adversary goal are \emph{omission}, to not add a
certificate to a CT log after promising to include it, and
\emph{inconsistency}, to include two different certificates for a
single domain in the same CT log by creating split versions of the
log, one for each certificate.

An omission attack causes a browser to accept a TLS certificate
fraudulently issued by a misbehaving CA\@. Undetectability stems from
doing so without creating the standard records in the Certificate
Transparency system that could implicate this fraudulent CA behavior
to anyone who checked.  In particular, this means the browser will
receive a signed certificate timestamp (SCT) for the certificate,
which is effectively a promise by a CT log to add that certificate to
the log within a specified time. (CT logs are append-only, designed so
that any deletion will be publicly visible.) The log will, however,
not include the certificate. Thus, any monitor checking up on
certificates issued for the target website will not find in the log
evidence of the fraudulent certificate.

An adversary might use an inconsistency attack to hide the issuance of
a fraudulent certificate.  In principle, an inconsistency attack might
involve no misbehavior on the part of any CA, however. For example, a
site owner might validly obtain multiple certificates for its site from
a single CA\@. A malicious CT log could hide this even while including
each certificate in a different version of its log. We will assume 
the adversary owns a CA, rather than explore whether any practical
attacks could result from a weaker adversary that does not. 

We assume a powerful adversary, one that controls a CA, one or more CT
logs, some Tor clients and a moderate fraction of Tor network
relays. The adversary remains within the threat model of Tor and 
Tor Browser, however. The adversary cannot alter consensus values
produced and authenticated from the directory authorities, including
the current signed tree head (STH) for any CT log in Tor's trust store.
Domain owners and others able to connect to the Tor network cannot
be blocked from accessing the current consensus STH\@.

We use elements of the public Tor network in our design and thus
consider attacks only on Tor Browser clients, but it would be
straighforward to check certificates and SCTs received by other
clients by creating an API to submit these via our design. For
purposes of this paper, we do not consider an attacker that attempts
to offers one consistent version of a CT log via Tor and a different
(consitent) version not via Tor.  Our design uses CT-auditing Tor
relays (CTRs) to periodically query CT logs for inclusion of
certificates for which SCTs have been issued during client
handshakes. The adversary obviously can issue fraudulent certificates
and SCTs. The attacks below describe how the adversary can identify
within seconds the relay responsible for a query to a malicious log,
and can hamper or prevent an identified CTR from reporting to
third-party CT auditors.
The adversary is expected to know the configuration and resources of
all CTRs: their memory and processing capacities, maximum sustained
bandwidth, etc. 

Such a powerful adversary could be expected to succeed some of the
time.  But, a CA caught issuing fraudulent certificates or a CT log
caught failing to add certficates for issued SCTs will no longer be
included as trusted by browsers or network infrastructure. We thus
assume that our adversary is risk-averse.  In particular, above a
relatively low threshold probability of attack detection, the
adversary would prefer to avoid detection than to succeed in sending
the SCT to a client without adding the certificate to the SCT-issuing
CT log or maintaining multiple versions of that CT log. 

For example, the adversary might possess adequate bandwidth to DoS
individual CTRs for at least several hours. But whether a sustained
DoS of all CTRs is feasible for the adversary, it is out of scope
simply because the attack would be too visible. An attack that
attempts to overwhelm the storage capacity for SFOs at all CTRs would
similarly be likely too visible, though see the discussion of flushing
attacks below for details. Likewise an adversary may have access to a
zero-day that could affect Tor Browser clients or a zero-day for some
CTRs. But these should be considered expensive, and relatively quickly
detected and repaired.  So, an adversary will not preemptively spend a
zero-day that affects only one family of operating systems because
CTRs run on multiple OSes and which CTR is relevant to a given attack
cannot be known ahead of time. And a zero-day that effectively allows
the simultaneous takeover of all Tor relays is out of scope. Likewise
the adversary may otherwise have the ability to take over or simply
restart individual CTRs or groups of CTRs, but taking over or
restarting all CTRs simultaneously or within several hours of each
other is also out of scope.

%\subsection{Attacks}
%\label{sec:attacks}

Even if communication between a CTR and a CT log is
via Tor, by timing correlation of traffic exiting and entering the CTR
with queries to and responses from an adversary-controlled CT log, an
adversary might identify the CTR responsible for sending a log
query. Tagging the CTR, however, so that the log can recognize queries
the CTR sends is generally more certain and requires less effort.

\noindent {\bf Tagging:} Since the adversary controls Tor clients and
relays, it can regularly send many different SFOs to different
CTRs. This means that, whatever the interval at which a CTR might
request audit to a CT log, some adversary SCTs will have exceeded
maximum merge delay (MMD) during each interval.  Thus, any SFOs a CTR
sends to a CT log reflecting SCTs received that are now past the
maximum merge delay (MMD) will include some SCTs that are unique to
that CTR\@. (In this paper, we are not describing any of the designs
we have considered that bounce some SCTs to other CTRs before querying
logs. And avoiding tags by sending just one SCT per query connection
will incur too much overhead.)

\noindent {\bf Flooding:} The adversary might also attempt to
eliminate the offending SCT from the CTR by flushing out its cache of
all SCTs that are awaiting audit\@. (Or if flushing is not permitted,
blocking, filling its cache so that the CTR will not accept new ones
until the next audit.) 
The adversary is not so powerful that it can observe all traffic to/from
\emph{every} CTR, whether via owned adjacent Tor relays or local
monitoring of the underlying network.  Thus, before any log has been
queried, which SFO has been sent by an honest client to which honest
CTR may not be known. To have an adequate guarantee of success, the
adversary must thus flush or block all the CTRs in the network.



%%% Local Variables: 
%%% mode: latex 
%%% TeX-master: "../main"
%%% End:          
