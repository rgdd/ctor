\section{Discussion} \label{sec:discussion}

\subsection{Privacy}
At least mention privacy leaks to auditor. For example, related to CT logs
uptime. Caching of already resolved SFOs at CTRs reduce leak to CT logs and
auditors.

\subsection{Design}
A CTR should be stable both to permit long-lived circuits to the CT logs and
to increase the likelihood of them staying online.

The criteria of being a middle relay, as opposed to an exit relay, follows
mainly from resource utilization considerations.  It is also about not making
exit relays ``more attractive targets'' due to also storing SFOs.

\subsection{Extension: Signal to CTRs}
On suspicion of network-wide attack, network health / auditors / DAs could
signal to CTRs through consensus flag to persist buffer of SFOs to disk for
later auditing to prevent loss on flooding or downtime. The reason for doing
this is that it shifts the potential \emph{impact} of detection (see
Section~\ref{sec:risk:impact}) from minor to potentially signfiicant or
catastrophic for the attacker.

There are huge privacy implications here, since the auditor gets a snapshot of
the current state of visited websites over Tor. At the very least, CTRs should
persist data in encrypted form (under provided public key in consensus), such as
not able to decrypt on its own to reduce risk of legal pressure on CTR
operators.

\subsection{CT Logs} \label{sec:discussion:logs}
Current landscape, log independence, accepted roots, available software
