\section{Design}
\label{sec:design}
Figure~\ref{fig:overview} provides an overview of our design idea.  The starting
point is that zero or more certificate chains and SCTs are presented
throughout a website visit (step~1).  We refer to these as SCT Feedback Objects
(SFOs)~\cite{nordberg}.  Tor Browser accepts an SFO if it adheres to an
SCT-centric CT policy similar to Chrome and Safari~%
\cite{chrome-policy,safari-policy}, but it may also be submitted to a CTR
in the background using a fresh independent circuit (step~2).  CTR flags are
assigned by Tor's directory authorities, and among other things, they also
determine Tor's view of the CT landscape by publishing STHs and announcing
third-party CT auditors in the Tor consensus.  CTRs store SFOs temporarily
before challenging the logs to prove inclusion with regards to Tor's published
STHs (step~3).  Should a log fail to provide a valid inclusion proof for an SFO
within a timely manner, it is reported to an auditor that investigates the issue
further (step~4).
\begin{figure}
	% TODO: clarify circuit structure to CTR in the overview figure?
	\centering
	\def\ed{0.8} % distance between entities (cm)
\def\orWidth{0.67} % onion router width (cm)
\def\wsWidth{1} % website width (cm)
\def\tbWidth{0.8} % tor browser width (cm)
\def\logWidth{1} % ct log width (cm)
\def\auditorWidth{1} % ct auditor width (cm)
\def\consensusWidth{1} % Tor consensus width (cm)

\begin{tikzpicture}[
	label/.style = {
		draw=none,
		font=\fontsize{6}{6}\rmfamily,
		text=black,
	},
	edges/.style = {
		draw=black,
		thick,
		-latex,
		rounded corners,
	},
]
	%%%
	% ORs in the centre
	%%%
	\node[draw=none](clor){\includegraphics[width=\orWidth cm]{%
		img/overview/or}
	};
	\node[draw=none, right=\ed cm of clor](cmor){%
		\includegraphics[width=\orWidth cm]{img/overview/or}
	};
	\node[draw=none, right=\ed cm of cmor](cror){%
		\includegraphics[width=\orWidth cm]{img/overview/or}
	};

	\coordinate(lc) at ($ (clor) !.5! (cmor) $);
	\coordinate(rc) at ($ (cmor) !.5! (cror) $);

	%%%
	% ORs on the top
	%%%
	\node[draw=none, above=\ed cm of lc](tlor){%
		\includegraphics[width=\orWidth cm]{img/overview/or}
	};
	\node[draw=none, above=\ed cm of rc](tror){}; % hide unnecessary OR

	%%%
	% ORs on the bottom
	%%%
	\node[draw=none, below=\ed cm of lc](blor){%
		\includegraphics[width=\orWidth cm]{img/overview/or}
	};
	\node[draw=none, below=\ed cm of rc](bror){%
		\includegraphics[width=\orWidth cm]{img/overview/or}
	};

	%%%
	% Website
	%%%
	\node[draw=none, right=\ed cm of cror](ws){%
		\includegraphics[width=\wsWidth cm]{img/overview/ws}
	};
	\node[label, above=0pt of ws]{\textbf{Website}};

	%%%
	% Tor browser
	%%%
	\node[draw=none, left=\ed cm of clor](tb){%
		\includegraphics[width=\tbWidth cm]{img/overview/tb}
	};
	\node[label, below=0pt of tb]{\textbf{Tor browser}};

	%%%
	% CT log
	%%%
	\coordinate(tc) at ($ (tlor) !.5! (tror) $);
	\node[draw=none, above=\ed cm of tc](log) {%
		\includegraphics[width=\logWidth cm]{img/overview/log}
	};
	\node[label, above=0pt of log]{\textbf{CT log}};

	%%%
	% Announced auditor
	%%%
	\node[draw=none, right=\ed cm of log](auditor) {%
		\includegraphics[width=\auditorWidth cm]{img/overview/auditor}
	};
	\node[label, above=0pt of auditor]{\textbf{Auditor}};

	%%%
	% Consensus
	%%%
	\node[draw=none, left=\ed cm of log](consensus) {%
		\includegraphics[width=\consensusWidth cm]{img/overview/consensus}
	};
	\path[edges, densely dotted]
		($ (consensus) !.40! (tlor) $) --
		($ (consensus) !.65! (tlor) $);
	\node[label, above=0pt of consensus]{\textbf{Tor consensus}};
	
	%%%
	% 1) Website -> TB
	%%%
	\path[edges]
		(ws) |-
			node[label,below,pos=.75]{1. receive SFO}
		(bror.350);
	\path[edges] (bror.190) -- (blor.350);
	\path[edges] (blor.190) -| (clor);
	\path[edges] (clor.210) to[out=230, in=310] (tb.330);

	%%%
	% 2) TB -> CTR
	%%%
	\path[edges, dashed]
		(tb.30) to[out=10, in=130]
			node[label, above, pos=0.25]{2. submit SFO}
		(clor.150);
	\path[edges, dashed] (clor) |- (tlor.190);
	\path[edges, dashed] (tlor.350) -| (cmor);
	\path[edges, dashed] (cmor.350) -- (cror.190);

	%%%
	% 3) CTR auditing
	%%%
	\path[edges, latex-latex, dotted]
		(cror.135) --
			node[label, below, sloped]{3. audit SFO}
		(log);

	%%%
	% 4) CTR reporting
	%%%
	\path[edges, dashdotted]
		(cror) --
			node[label, above, sloped]{4. report SFO}
			node[label, below, sloped]{on timeout}
		(auditor);
\end{tikzpicture}

	\caption{%
		Design overview---Tor Browser submits presented SFOs to CTRs that follow
		up on their inclusion statuses.  Entity-to-entity interactions are
		conveyed over full Tor circuits, labeled by onions.
	}
	\label{fig:overview}
\end{figure}

\subsection{Tor Consensus}
Tor's directory authorities produce an hourly consensus document that defines
how the Tor network is composed.  Our proposal extends Tor's consensus document
so that it entails information regarding
	CTRs,
	recognized CT logs,
	announced auditors, and
	security parameters.

\subsubsection{CTR Flag}
The existing \texttt{know-flags} item determines the different flags that a
given consensus document might contain.  For example, documented flags include
\texttt{Authority}, \texttt{Exit}, and \texttt{Running}.  We add another flag
named \texttt{CTR}, which indicates that a relay should support CT-auditing as
described in Section~\ref{sec:design:ctr}.  A relay qualifies as a CTR if it is
flagged as \texttt{stable} and \texttt{middle}.  The \texttt{CTR} flag is
assigned if a majority of directory authorities voted for it.

% - A CTR should be stable both to permit long-lived circuits to the CT logs and
% to increase the likelihood of them staying online.
% - The criteria of being a middle relay, as opposed to an exit relay, follows
% mainly from resource utilization considerations.  It is also about not making
% exit relays "more attractive targets" due to also storing SFOs.

\subsubsection{Recognized CT Log}
The CT landscape is composed of a few dozen logs.  At minimum, the Tor
consensus must recognize those referred by Tor Browser's CT policy.  At some
point during the voting timeline, each directory authority
(i) checks whether there are any CT policy updates, and
(ii) fetches an STH from each log that is recognized.
To agree on a single STH per log, choose the most recent one as determined by
timestamp and resolve ties by appealing to lexicographical order.  A log is
included in the Tor consensus if a majority of directory authorities voted
for it by proposing an instance of \texttt{ct-log-info}:
\begin{description}
	\item[ct-log-info:] Contains a log ID, an MMD and a base URL as found in
		Tor Browser's CT policy, as well as an STH following the log's
		\texttt{get-sth} format~\cite{ct,ct/bis}.
\end{description}

There should be no disagreement regarding a log's basic metadata.  However, if
such an erroneous state occurs, deterministic rules could be used to agree upon
an MMD and a base URL while investigating the issue further.

\subsubsection{Announced Auditor}
Whenever log misbehavior is suspected, the CTR in question reports it to a
CT auditor that investigates the issues further.  Tor's directory authorities
announce CT auditors by proposing a value for the item below, and it is included
in the Tor consensus if a majority of directory authorities voted for it.
\begin{description}
	\item[ct-auditor:] Contains a base URL and a fingerprint that is derived
		from the auditor's public key info.
\end{description}

As built upon in Section~\ref{sec:design:auditor}, the base URL determines the
start of an auditor's submission endpoint.  Moreover, it should be noted that
we pin the auditor's public key because we are dealing with an attacker that
forges TLS certificates.  E.g., a SPKI fingerprint could be used~\cite{hpkp}.

\subsubsection{Security Parameters}
The security of our proposal depends on parameters that Tor's directory
authorities set.  These parameters are first outlined below briefly, then used
throughout the remainder of this section and discussed further later on.
Note that each item occurs exactly once in the Tor consensus:
it is based on the median value of all votes.
\begin{description}
	\item[ct-submit-pr:] A floating-point in $[0,1]$ that determines Tor
		Browser's submission probability, i.e., whether an SFO should be sent to
		a random CTR for further auditing.  E.g., $0$ disables auditing
		while $0.10$ implies every 10$^{\mathsf{th}}$ SFO is audited
		on average.
	\item[ct-submit-timeout:] A natural number that determines how many ms Tor
		Browser waits before concluding that a CTR is unresponsive.  As
		outlined in Section~\ref{sec:design:tb}, a submission timeout results
		in a resubmission to another CTR that is selected at random.
	\item[ct-query-timeout:] A natural number that determines how many ms a CTR
		waits before concluding that a CT log is unresponsive.  As outlined in
		Section~\ref{sec:design:ctr:audit}, a query timeout triggers an
		auditor submission.
	\item[ct-report-timeout:] A natural number that determines how many ms a
		CTR waits before concluding that an announced auditor is unresponsive.
	\item[ct-report-count:] The number of announced auditors that a CTR
		resubmits an SFO to after a report timeout, as well as the number of CT
		logs that the underlying certificate chain is submitted to for public
		logging.  Further details are outlined in
		Section~\ref{sec:design:ctr:audit}.
	\item[ct-backoff:] A natural number that determines how many ms a CTR
		may wait between two auditing instances.  As outlined in
		Section~\ref{sec:design:ctr:audit}, CTRs audit pending SFOs
		in batches at random time intervals.
	\item[ct-sfo-max-bytes:] A natural number that determines how many
		wire-bytes a normal SFO should not exceed.  As outlined in
		Section~\ref{sec:design:tb}, excessively large SFOs are subject to
		stricter verification criteria.
\end{description}

\subsection{Tor Browser} \label{sec:design:tb}
Given a tab $t$ and an incoming SFO $s$:
\begin{enumerate}
	\item Accept $s$ and stop if it was already validated in $t$.
	\item Raise a certificate error and stop if the certificate chain of $s$
		is not rooted in Tor Browser's trust store.
	\item Raise a certificate transparency error and stop if the SCTs of $s$
		fail Tor Browser's SCT-centric CT policy.
	\item Conduct the following steps in the background:
		\begin{enumerate}
			\item Signal that $s$ should be accepted as valid if its byte-size
				is \texttt{ct-max-sfo-bytes} or less.
			\item Flip a coin based on \texttt{ct-submit-pr} and go to
				step~\ref{enm:tb:done} if there should be no further auditing.
			\item\label{enm:tb:submit} Use \texttt{ct-submit-timeout} to set a
				timer and submit $s$ to a sampled CTR's SFO-endpoint on a
				pre-built CT-circuit that starts from the client's guard and
				ends at the CTR: three hops in total.
				\begin{itemize}
					\item On a timely acknowledgment: close the submission
						circuit and go forward to step~\ref{enm:tb:done}.
					\item On any other outcome: close the submission circuit and
						repeat step~\ref{enm:tb:submit}.
				\end{itemize}
			\item\label{enm:tb:done} Signal that $s$ should be accepted as valid
				if its byte-size is larger than \texttt{ct-max-sfo-bytes}.
		\end{enumerate}
	\item Wait for a signal that $s$ should be accepted as valid, then mark it
		as validated in $t$ and stop.
\end{enumerate}

TODO: readd context text

%Similar to Chrome and Safari~\cite{chrome-policy,safari-policy}, we suggest that
%Tor Browser should use an SCT-centric CT policy.  This means that a certificate
%chain is accepted as valid if it is accompanied by a threshold of SCTs.  Such a
%policy would ideally come from Mozilla directly (like many other components),
%but at the time of writing there is no official CT policy that can be inherited
%from Firefox.
%
%If an SFO passes Tor Browser's CT policy, the value of
%\texttt{ct-submit-pr} is used to flip a biased coin.  The outcome determines
%whether further inclusion verification should take place, in which case the SFO
%is submitted to a sampled CTR over a fresh independent circuit
%that is \emph{pre-built} from the client's guard relay (three hops in total) 
%and closed immediately after use.  In other words, Tor Browser should maintain a
%pool of one-time circuits that are used for CT-auditing purposes only.
%Should the same SFO be presented multiple times within the same tab, it is only
%considered for further auditing on a \emph{first-encounter} basis.
%If a submitted SFO is unacknowledged after some maximum time as decided by
%\texttt{ct-submit-timeout}, it is resubmitted as if there was no failure to
%begin with:
%	use another pre-built circuit to a sampled CTR.
%This ensures that Tor Browser does not keep submitting to a CTR that is no
%longer available, while at the same time having limited security impact.

%
% - close circuit asap to make it harder for an attacker to figure out which
% CTR received a submission (should it have access to a zero-day takeover).
% - clearly a submission circuit cannot be reused across tabs, but not doing
% so within tabs may (i) reduce the chance that the receiving CTR knows exactly
% which website was visited, and (ii) make sense because with small submission
% pr (<=1/10) it should be common to submit at most once per tab anyway.
% - resubmission:
%   a) new circuit because the exit might be attacker controlled and
%   intentionally be blocking access to a CTR that is not attacker-controlled
%   b) new ctr because we don't wanna keep on trying indefinitely if there is
%   an actual problem with the selected ctr.  At the same time, it will be
%   difficult for the attacker to rely on Tor Browser never sampling both a
%   benign exit and ctr (in which case the submission should succeed).
%

\subsection{CTR} \label{sec:design:ctr}
A relay that is assigned the CTR flag accepts incoming SFO submissions on a
dedicated endpoint.  Somewhat periodically, an auditing process then runs which
verifies the inclusion status of SFOs that are pending.  CTRs also
publish health metrics in the extra-info document.

\subsubsection{Submission Endpoint} \label{sec:design:api}
We suggest that CTRs accept SFO submissions on an HTTP endpoint.\footnote{%
	Tor's HTTP DirServer codebase can be reused as extension point to interact
	with the tor daemon, i.e., add another listener.
} For example, Nordberg~\emph{et~al.} defined an SCT feedback interface that can
be reused if an array-length of one is enforced by the CTR~\cite{nordberg}.
With regards to some circuit, process an incoming SFO $s$ as follows:
\begin{enumerate}
	\item\label{enm:ctr-api:well-formed} Close the circuit and stop if $s$
		contains no SCT that matches at least one log ID in a
	\texttt{ct-log-info} entry.
	\item\label{enm:ctr-api:ack} Update $s$ by discarding any
		unrecognized SCT, % otherwise can exploit for easier flush
		then acknowledge the reception and close the circuit.
	\item\label{enm:ctr-api:cached}
		Stop if $s$ is cached or pending inclusion verification.
	\item\label{enm:ctr-api:audit-after} Sample an SCT in $s$, noting down the
		outcome and a corresponding \texttt{audit\_after} timestamp
		(Figure~\ref{fig:audit-after}).
	\item\label{enm:ctr-api:store} Store $s$ and its \texttt{audit\_after}
		timestamp in a buffer of pending SFOs that is managed by Tor's OOM.
\end{enumerate}

First we verify that the submitted SFO is well-formed and that it contains at
least one SCT that corresponds to an STH in the Tor consensus.
Possibly preceded by an acknowledgment that the SFO will be audited, the
associated circuit is then closed.  This means that no acknowledgment should be
treated as an error, and at most one SFO can be submitted on a given circuit.
New SFOs, i.e., SFOs that are neither \emph{cached} nor waiting to be resolved
in a \emph{buffer} of pending SFOs, are stored.  We only keep recognized
SCTs around to ensure that no CTR memory is wasted, and one of these SCTs are
sampled so that an \texttt{audit\_after} timestamp can be computed.  The
\texttt{audit\_after} timestamp determines the earliest point in time that an
SFO will be considered for auditing:
	a random delay is added to leak less real-time information regarding visited
		websites to the CT logs, and
	by auditing after an SCT's MMD elapsed the log \emph{must} have an
		inclusion proof available or it misbehaves.

If memory becomes a scarce relay resource, e.g., due to flushing, OOM
should delete SFOs at random~\cite{nordberg}.  The impact of such bulk-deletes
are discussed in Section~\ref{sec:todo}.

\begin{figure}
	\centering
	\pseudocode[linenumbering, syntaxhighlight=auto]{%
		\textrm{t} \gets \mathsf{now}() \\
		\pcif \textrm{SCT.timestamp} + \textrm{MMD} >
				\textrm{t}:\\
			\pcind\textrm{t} \gets \mathsf{min}(
				\textrm{SCT.timestamp}, \textrm{t} +
				\textrm{MMD}
			)\\
			\textrm{t} \gets \textrm{t} + \mathsf{random\_delay}()
	}
	\caption{%
		Algorithm that computes an \texttt{audit\_after} timestamp that is
		bound by the CTR's perception of time and the log's MMD.
	}
	\label{fig:audit-after}
\end{figure}


\subsubsection{Auditing Process} \label{sec:design:ctr:audit}
CTR auditing is initiated at random time intervals to spread out the load
imposed by the Tor network on CT logs.  Each auditing instance is composed of
circuit setup, a core loop of SFO enumeration, and circuit tear-down.

\begin{enumerate}
	\item\label{enm:backoff} Sample a uniform delay from
			$[0, \texttt{ct-backoff}]$,
		then schedule a timer and wait until that time elapsed.
	\item\label{enm:audit-circuit} Establish a new circuit and connect to a
		randomly selected auditor in the Tor consensus.  Test that the
		auditor is available by submitting a dummy-SFO using
		\texttt{ct-report-timeout}.  If not, close the circuit and repeat this
		step at most \texttt{ct-report-count} times.
	\item\label{enm:log-circuit} Establish a new circuit and use it for all
		subsequent log connections.  Connect to the logs when needed.
	\item\label{enm:audit-loop} Enumerate the set of pending SFOs, referring to
		the SCTs in Section~\ref{sec:design:api},
		step~\ref{enm:ctr-api:audit-after}, and their logs' STHs:
		\begin{enumerate}
			\item\label{enm:audit-loop:wait-sct} Continue if
				$\textrm{SFO}.\mathsf{audit\_after} > \mathsf{now}()$.
			\item\label{enm:audit-loop:wait-sth} Continue if
				$\textrm{SFO}.\mathsf{audit\_after} >
				\textrm{STH}.\mathsf{timestamp}$.
			\item\label{enm:audit-loop:challenge}
				Use \texttt{ct-query-timeout} and
				$\textrm{STH}.\mathsf{treesize}$ to set a timer and challenge
				the log to prove inclusion.
				\begin{itemize}
					\item\label{enm:audit-loop:challenge:success} On valid
						proof: cache the SFO in question by storing a hashed
						representation.
					\item\label{enm:audit-loop:challenge:fail} On any other
						outcome: use \texttt{ct-report-timeout} to set a timer
						and send the entire SFO to the auditor in
						step~\ref{enm:audit-circuit}.  On any failure, log $s$
						to torlog and re-report it:
						\begin{itemize}
							\item Sample \texttt{ct-report-count}
								auditors and resubmit $s$ on (new) separated
								circuits.
							\item Sample \texttt{ct-report-count} CT logs not
								referred by $s$ and submit the underlying
								certificate chain via \texttt{add-chain}~\cite{ct}
								or \texttt{submit-entry}~\cite{ct/bis}
								on separated circuits.
						\end{itemize}
						Finally, discard the SFO and break the loop.
				\end{itemize}
		\end{enumerate}
	\item\label{enm:teardown} Close all opened circuits and go back to
		step~\ref{enm:backoff}.
\end{enumerate}

Keeping auditor and log interactions separate is good circuit hygiene in
general.  For example,  sharing these circuits would allow the attacker to
identify which exit relay should be DoS:ed to delay an auditor submission.
In the core loop, we only audit an SFO's sampled SCT if its MMD elapsed
and there is an STH available in the Tor consensus that captures it.  Note that
the full SFO is always reported on failure, which ensures that the auditors can
validate all SCTs further.  If an auditor becomes unavailable, we take several
actions to increase the likelihood that an SFO and/or its underlying certificate
chain makes it into the public domain.  Namely, resubmit to several auditors,
log it locally to torlog, and submit the underlying certificate chain for
merging into independent CT logs that the attacker may not control.

% where sth first after mmd elapsed !  Missing right now
% - not necessary, let auditor bother with that
%
% Remember: audit sampled sct -> less overhead, less info leak to ct logs
%

\subsubsection{Extra-Info Document} \label{sec:design:extra-info}
The extra-info document tells us something about a Tor relay's internal state of
affairs.  For example, \texttt{read-history} and \texttt{write-history} list how
many bytes were consumed throughout different time intervals.  We further
require that the following metrics be added:
\begin{description}
	\item[ct-receive-bytes:] Occurs at most once.  Interval width, followed by a
		list of received SFO bytes per interval.
	\item[ct-delete-bytes:] Occurs at most once.  Interval width, followed by a
		list of deleted SFO bytes per interval.
\end{description}

A CTR should generate a new descriptor and extra-info document if the most
current extra-info document had no deletions while the next one will have at
least one deletion:
	this enables \emph{early detection} of flushing-related activities
	(or suspicion thereof).
There are other health-related metrics that could be added in the
extra-info document, such as
	the number of SFO submissions and deletions,
	the ratio between SFOs that are younger than an MMD, and
	per-log failure rates while querying for inclusion.
The latter, for example, could be used as an indicator that the current
query-timeout is too optimistic.

\subsection{Announced Auditors} \label{sec:design:auditor}
An announced auditor is expected to implement the SCT feedback interface of
Nordberg~\emph{et~al.}~\cite{nordberg}.  If a well-formed SFO is received, the
auditor in question should endeavor to validate the inclusion status of each SCT
with regards to the first STH in the Tor consensus that elapsed the log's MMD.
An announced auditor should also verify that each STH in the Tor consensus is
in fact consistent by fetching consistency proofs from the logs.
While not within our threat model, we do encourage the announced auditors to
verify that STHs in the Tor consensus are consistent with external views of the
CT landscape.  For example, operate an STH pollination~\cite{nordberg} endpoint
and fetch STHs actively from many diverse vantage points using Tor, VPN
services, DoH resolvers, and RIPE Atlas (to mention a few low-cost options).

% TODO: two-component system??
If the log appears to function correctly except for some evidence that cannot be
resolved despite multiple attempts that span a given time period, it is
paramount that the auditor software alerts its operator.  In detail:
\begin{itemize}
	\item If the log fails to provide a valid inclusion proof for an SCT with
		regards to the first applicable STH in the Tor consensus
		(an MMD violation is suspected).
	\item If the log fails to provide a valid consistency proof between any two
		STHs in the Tor consensus
		(a split-view is suspected within the Tor network).
	\item If an STH in the Tor consensus is future-dated or backdated more than
		an MMD (general log misbehavior suspected).
	\item If a log ignores parts of or the entire Tor network (uptime
		misbehavior suspected).
\end{itemize}

An unresponsive log can be suspected by inspecting the report frequency, and
possibly confirmed by querying the log in question over a diverse set of Tor
circuits.  As mentioned in Section~\ref{sec:design:extra-info}, it could be
valuable to let CTRs publish their failure rates while querying the logs.

An announced auditor may choose to replicate possible evidence of log
misbehaviour to the other announced auditors prematurely to ensure that
it is persisted beyond its own durable storage.  At some point, the auditor
software should generate a complete report that can be forwarded by the
auditor's operator to the CT-policy mailing list for further community
investigation.
