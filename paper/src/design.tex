\section{Design}
\label{sec:design}
Figure~\ref{fig:overview} provides an overview of our design idea.  The starting
point is that zero or more certificate chains and SCTs are presented
throughout a website visit (step~1).  We refer to these as SCT Feedback Objects
(SFOs)~\cite{nordberg}.  Tor Browser accepts an SFO if it adheres to an
SCT-centric CT policy similar to Chrome and Safari~%
\cite{chrome-policy,safari-policy}, but it may also be submitted to a CTR
in the background using a fresh independent circuit (step~2).  CTR flags are
assigned by Tor's directory authorities, and among other things, they also
determine Tor's view of the CT landscape by publishing STHs and announcing
third-party CT auditors in the Tor consensus.  CTRs store SFOs temporarily
before challenging the logs to prove inclusion with regards to Tor's published
STHs (step~3).  Should a log fail to provide a valid inclusion proof for an SFO
within a timely manner, it is reported to an auditor that investigates the issue
further (step~4).
\begin{figure}
	\centering
	\def\ed{0.8} % distance between entities (cm)
\def\orWidth{0.67} % onion router width (cm)
\def\wsWidth{1} % website width (cm)
\def\tbWidth{0.8} % tor browser width (cm)
\def\logWidth{1} % ct log width (cm)
\def\auditorWidth{1} % ct auditor width (cm)
\def\consensusWidth{1} % Tor consensus width (cm)

\begin{tikzpicture}[
	label/.style = {
		draw=none,
		font=\fontsize{6}{6}\rmfamily,
		text=black,
	},
	edges/.style = {
		draw=black,
		thick,
		-latex,
		rounded corners,
	},
]
	%%%
	% ORs in the centre
	%%%
	\node[draw=none](clor){\includegraphics[width=\orWidth cm]{%
		img/overview/or}
	};
	\node[draw=none, right=\ed cm of clor](cmor){%
		\includegraphics[width=\orWidth cm]{img/overview/or}
	};
	\node[draw=none, right=\ed cm of cmor](cror){%
		\includegraphics[width=\orWidth cm]{img/overview/or}
	};

	\coordinate(lc) at ($ (clor) !.5! (cmor) $);
	\coordinate(rc) at ($ (cmor) !.5! (cror) $);

	%%%
	% ORs on the top
	%%%
	\node[draw=none, above=\ed cm of lc](tlor){%
		\includegraphics[width=\orWidth cm]{img/overview/or}
	};
	\node[draw=none, above=\ed cm of rc](tror){}; % hide unnecessary OR

	%%%
	% ORs on the bottom
	%%%
	\node[draw=none, below=\ed cm of lc](blor){%
		\includegraphics[width=\orWidth cm]{img/overview/or}
	};
	\node[draw=none, below=\ed cm of rc](bror){%
		\includegraphics[width=\orWidth cm]{img/overview/or}
	};

	%%%
	% Website
	%%%
	\node[draw=none, right=\ed cm of cror](ws){%
		\includegraphics[width=\wsWidth cm]{img/overview/ws}
	};
	\node[label, above=0pt of ws]{\textbf{Website}};

	%%%
	% Tor browser
	%%%
	\node[draw=none, left=\ed cm of clor](tb){%
		\includegraphics[width=\tbWidth cm]{img/overview/tb}
	};
	\node[label, below=0pt of tb]{\textbf{Tor browser}};

	%%%
	% CT log
	%%%
	\coordinate(tc) at ($ (tlor) !.5! (tror) $);
	\node[draw=none, above=\ed cm of tc](log) {%
		\includegraphics[width=\logWidth cm]{img/overview/log}
	};
	\node[label, above=0pt of log]{\textbf{CT log}};

	%%%
	% Announced auditor
	%%%
	\node[draw=none, right=\ed cm of log](auditor) {%
		\includegraphics[width=\auditorWidth cm]{img/overview/auditor}
	};
	\node[label, above=0pt of auditor]{\textbf{Auditor}};

	%%%
	% Consensus
	%%%
	\node[draw=none, left=\ed cm of log](consensus) {%
		\includegraphics[width=\consensusWidth cm]{img/overview/consensus}
	};
	\path[edges, densely dotted]
		($ (consensus) !.40! (tlor) $) --
		($ (consensus) !.65! (tlor) $);
	\node[label, above=0pt of consensus]{\textbf{Tor consensus}};
	
	%%%
	% 1) Website -> TB
	%%%
	\path[edges]
		(ws) |-
			node[label,below,pos=.75]{1. receive SFO}
		(bror.350);
	\path[edges] (bror.190) -- (blor.350);
	\path[edges] (blor.190) -| (clor);
	\path[edges] (clor.210) to[out=230, in=310] (tb.330);

	%%%
	% 2) TB -> CTR
	%%%
	\path[edges, dashed]
		(tb.30) to[out=10, in=130]
			node[label, above, pos=0.25]{2. submit SFO}
		(clor.150);
	\path[edges, dashed] (clor) |- (tlor.190);
	\path[edges, dashed] (tlor.350) -| (cmor);
	\path[edges, dashed] (cmor.350) -- (cror.190);

	%%%
	% 3) CTR auditing
	%%%
	\path[edges, latex-latex, dotted]
		(cror.135) --
			node[label, below, sloped]{3. audit SFO}
		(log);

	%%%
	% 4) CTR reporting
	%%%
	\path[edges, dashdotted]
		(cror) --
			node[label, above, sloped]{4. report SFO}
			node[label, below, sloped]{on timeout}
		(auditor);
\end{tikzpicture}

	\caption{%
		Design overview---Tor Browser submits presented SFOs to CTRs that follow
		up on their inclusion statuses.  Entity-to-entity interactions are
		conveyed over full Tor circuits, labeled by onions.
	}
	\label{fig:overview}
\end{figure}

\subsection{Tor Consensus}
Tor's directory authorities produce an hourly consensus document that defines
how the Tor network is composed.  Our proposal adds the following parameters,
further explained throughout Section~\ref{sec:design}:
\begin{description}
	\item[ct-get-sth:] Occurs multiple times.  Log ID, followed by an STH that is
		formatted as returned by the log's \texttt{get-sth}
		endpoint~\cite{ct,ct/bis}.
	\item[ct-submit-pr:] Occurs exactly once.  A floating-point in $[0,1]$
		that determines the probability that an SFO should be submitted from Tor
		Browser to a CTR.
	\item[ct-query-timeout:] Occurs exactly once.  A natural number that
		determines how many ms a CTR waits before concluding that a CT log is
		unresponsive.
	\item[ct-backoff:] Occurs exactly once.  A natural number that determines
		how many ms a CTR backs-off at most while auditing SFOs for inclusion.
		Note that this is the upper bound of a uniform distribution.
	\item[ct-auditor:] Occurs multiple times.  A submission URI, followed by a
		SPKI fingerprint that must be present in the TLS certificate presented
		by the auditor~\cite{hpkp}.
\end{description}

Throughout any given voting timeline, each directory authority is expected to
obtain the most recent STH from all recognized CT logs.  Next, while computing
the consensus, they must agree to use the most recent STH by inspecting the
timestamp field and resolving ties by appealing to lexicographical
order.\footnote{%
	Note that there is cryptographic proof of log misbehaviour if two STHs with
	the same tree size have different root hashes.
}  A log is recognized and a proposed CT auditor is announced if a majority of
all directory authorities voted on their inclusion, which is implicit by
proposing a value.  The submit probability, query timeout, and CTR back-off
parameters are all computed as the median values of the respective votes.

Other than the new parameters introduced above, we also need to extend the
\texttt{know-flags} parameter by adding a CTR-flag.  The CTR-flag is 
assigned to stable middle relays if a majority of directory authorities voted
for it, and it indicates that a given Tor relay should take on the additional
responsibility of auditing CT logs.

% - A CTR should be stable both to permit long-lived circuits to the CT logs and
% to increase the likelihood of them staying online.
% - The criteria of being a middle relay, as opposed to an exit relay, follows
% mainly from resource utilization considerations.  It is also about not making
% exit relays "more attractive targets" due to also storing SFOs.

\subsection{Tor Browser}
Similar to Chrome and Safari~\cite{chrome-policy,safari-policy}, we suggest that
Tor Browser should use an SCT-centric CT policy.  This means that a certificate
chain is accepted as valid if it is accompanied by a threshold of SCTs.  Such a
policy would ideally come from Mozilla directly (like many other components),
but at the time of writing there is no official CT policy that can be inherited
from Firefox.

For each observed SFO by Tor Browser, if the SFO passes Tor Browser's CT policy,
a biased coin is flipped based on the current value of \texttt{ct-submit-pr}.
Should the outcome indicate auditing, a CTR is sampled uniformly from the set of
available CTRs and then the SFO is submitted on a fresh independent circuit that
is then immediately closed.  In case that an SFO is presented multiple times
within the same tab, it is only considered for auditing once.

%
% - close circuit asap to make it harder for an attacker to figure out which
% CTR received a submission (should it have access to a zero-day takeover).
% - clearly a submission circuit cannot be reused across tabs, but not doing
% so within tabs may (i) reduce the chance that the receiving CTR knows exactly
% which website was visited, and (ii) make sense because with small submission
% pr (<=1/10) it should be common to submit at most once per tab anyway.
%

\subsection{CTR}
A relay that is assigned the CTR flag accepts incoming SFO submissions on a
dedicated HTTP endpoint.  As such, we suggest using Tor's existing HTTP
DirServer as an extension point to interact with the tor daemon.  An auditing
process then runs somewhat periodically, ensuring that the received SFOs are
either merged into the logs or given attention to by the announced auditors.
Following from the threat of flushing, CTRs also publish additional health
metrics in the extra-info document.

\subsubsection{Submission Endpoint} \label{sec:design:api}
As outlined in detail below, a CTR listens for incoming SFO submissions on an
HTTP endpoint.  For example, it is possible to reuse the SCT feedback mechanism
that Nordberg~\emph{et~al.} defined~\cite{nordberg}.  At most one SFO is
expected per circuit.  Therefore, a submission should be rejected if it
contains more than one SFO (step~\ref{enm:ctr-api:incoming}).
A CTR should also reject a submission if it contains no SCT from a recognized CT
log, as there would be no STH available in the Tor consensus to use while
querying for inclusion (step~\ref{enm:ctr-api:well-formed}).
If an SFO corresponds to a frequently visited website, it might already be
resolved or waiting to be resolved in a \emph{cache} or a \emph{buffer} of
pending SFOs (step~\ref{enm:ctr-api:cached}).
Both the cache and the buffer of pending SFOs should be managed by Tor's OOM,
which takes care of similar resources such as DNS.  If an SFO is neither
resolved nor waiting to be resolved, an \texttt{audit\_after} timestamp $t$ is
computed for a randomly selected SCT (step~\ref{enm:ctr-api:audit-after}).
Unless the current time is greater than $t$, the SFO in question will not be
considered for auditing despite being stored in the buffer of pending SFOs
(step~\ref{enm:ctr-api:store}).
The reason why an SFO is not audited immediately is two-fold.  First, adding a
random delay in the order of minutes makes it harder for the log to learn about
real-time website visits.  Secondly, a log is not forced to add an SCT's entry
unless an MMD elapsed (discussed in Section~\ref{sec:todo}).

For each incoming SFO $s$:
\begin{enumerate}
	\item\label{enm:ctr-api:incoming} Return and error and stop if some SFO was
		already received over the current circuit.
	\item\label{enm:ctr-api:well-formed} Return an error and stop if $s$
		contains no SCT from a recognized log.  Unrecognized SCTs are dropped.
		% otherwise can exploit for flush
	\item\label{enm:ctr-api:cached} Check if $s$ is already cached or pending
		inclusion verification.  If so, discard $s$ and stop without error.
	\item\label{enm:ctr-api:audit-after} Sample an SCT in $s$, noting down the
		outcome and a corresponding \texttt{audit\_after} timestamp
		(Figure~\ref{fig:audit-after}).
	\item\label{enm:ctr-api:store} Store $s$ and its \texttt{audit\_after}
		timestamp in a buffer of pending SFOs that is managed by Tor's OOM.
\end{enumerate}

If memory becomes a scarce relay resource, e.g., due to flushing, OOM
should delete SFOs at random~\cite{nordberg}.  The impact of such bulk-deletes
are discussed in Section~\ref{sec:todo}.

\begin{figure}
	\centering
	\pseudocode[linenumbering, syntaxhighlight=auto]{%
		\textrm{t} \gets \mathsf{now}() \\
		\pcif \textrm{SCT.timestamp} + \textrm{MMD} >
				\textrm{t}:\\
			\pcind\textrm{t} \gets \mathsf{min}(
				\textrm{SCT.timestamp}, \textrm{t} +
				\textrm{MMD}
			)\\
			\textrm{t} \gets \textrm{t} + \mathsf{random\_delay}()
	}
	\caption{%
		Algorithm that computes an \texttt{audit\_after} timestamp.
	}
	\label{fig:audit-after}
\end{figure}


\subsubsection{Auditing Process}
CTR auditing is initiated at random time intervals to spread out the load
imposed by the Tor network on CT logs (step~\ref{enm:backoff}).
As described in detail below, each auditing instance is composed of
	initialization (steps~\ref{enm:audit-circuit}--\ref{enm:log-circuit}) and
	tear-down (step~\ref{enm:teardown})
of two dedicated Tor circuits that are used while interacting with the announced
auditors and CT logs, as well as an enumeration phase where the
inclusion statuses of the pending SFOs are checked with regards to their sampled
SCTs (step~\ref{enm:audit-loop}).  We audit a sampled SCT because it is
sufficient to verify that the corresponding certificate chain is available to
the public in \emph{some} log.  Such verification is, however, delayed
until an MMD elapsed (step~\ref{enm:audit-loop:wait-sct}) and an available STH
captures it (step~\ref{enm:audit-loop:wait-sth}).
This assures that the attacker cannot merge a certificate chain into the log
as a result of an SCT being audited by a CTR, i.e., the MMD
violation would already be captured by an STH in the Tor consensus.
Finally, when the log is challenged to prove inclusion in
step~\ref{enm:audit-loop:challenge}, the SFO is cached iff a valid proof is
provided in a timely manner (step~\ref{enm:audit-loop:challenge:success});
otherwise it is submitted to an announced auditor and the remainder of the
auditing instance is canceled (step~\ref{enm:audit-loop:challenge:fail}).
Section~\ref{todo} shows that it is reasonable to cancel the auditing instance.

% TODO: note on separated circuits
% TODO: clarify where timeout form and that it is a security parameter

% Auditor and log circuits are separated to enforce good circuit hygiene
% E.g., the attacker could delay the time it takes for an SFO to reach an
% auditor by DoS:ing the shared exit relay while 4c timesout

% Remember: audit sampled sct -> less overhead, less info leak to ct logs

\begin{enumerate}
	\item\label{enm:backoff} Sample a uniform delay from
			$[0, \texttt{ct-backoff}]$,
		then schedule a timer and wait until that time elapsed.
	\item\label{enm:audit-circuit} Establish a new circuit and connect to a
		randomly selected announced auditor in the Tor consensus.
	\item\label{enm:log-circuit} Establish a new circuit and use it for all
		subsequent log connections.  Connect to the logs when needed.
	\item\label{enm:audit-loop} Enumerate the set of pending SFOs, referring to
		the SCTs in Section~\ref{sec:design:api},
		step~\ref{enm:ctr-api:audit-after}, and their logs' STHs:
		\begin{enumerate}
			\item\label{enm:audit-loop:wait-sct} Continue if
				$\textrm{SFO}.\mathsf{audit\_after} > \mathsf{now}()$.
			\item\label{enm:audit-loop:wait-sth} Continue if
				$\textrm{SFO}.\mathsf{audit\_after} >
				\textrm{STH}.\mathsf{timestamp}$.
			\item\label{enm:audit-loop:challenge} Set a query-timeout and
				challenge the log to prove inclusion with regards to
				$\textrm{STH}.\mathsf{treesize}$.
				% where sth first after mmd elapsed !  Missing right now
				\begin{itemize}
					\item\label{enm:audit-loop:challenge:success} On valid
						proof: cache the SFO in question by storing a hashed
						representation.
					\item\label{enm:audit-loop:challenge:fail} On any other
						outcome: immediately send the entire SFO to the auditor
						from step~\ref{enm:audit-circuit}, then remove it from
						the buffer and break.
				\end{itemize}
		\end{enumerate}
	\item\label{enm:teardown} Close all circuits from
		steps~\ref{enm:audit-circuit}--\ref{enm:log-circuit} and go to
		step~\ref{enm:backoff}.
\end{enumerate}

\subsubsection{Extra-Info Document}
The extra-info document tells us something about a Tor relay's internal state of
affairs.  For example, \texttt{read-history} and \texttt{write-history} list how
many bytes were consumed throughout different time intervals.  We further
require that the following metrics be added:
\begin{description}
	\item[ct-receive-bytes:] Occurs at most once.  Interval width, followed by a
		list of received SFO bytes per interval.
	\item[ct-delete-bytes:] Occurs at most once.  Interval width, followed by a
		list of deleted SFO bytes per interval.
\end{description}

A CTR should generate a new descriptor and extra-info document if the most
current extra-info document had no deletions while the next one will have at
least one deletion:
	this enables \emph{early detection} of flushing-related activities
	(or suspicion thereof).
There are other health-related metrics that could be added in the
extra-info document, such as
	the number of SFO submissions and deletions,
	the ratio between SFOs that are younger than an MMD, and
	per-log failure rates while querying for inclusion.
The latter, for example, could be used as an indicator that the current
query-timeout is too optimistic.

\subsection{Announced Auditors}
An announced auditor is expected to implement the SCT feedback interface of
Nordberg~\emph{et~al.}~\cite{nordberg}.  If a well-formed SFO is received, the
auditor in question should endeavor to validate the inclusion status of each SCT
with regards to the first STH in the Tor consensus that elapsed the log's MMD.
Like any interested party, an announced auditor can and should verify that all
STHs in the Tor consensus are consistent by fetching such proofs.

% TODO: two-component system??
If the log appears to function correctly except for some evidence that cannot be
resolved despite multiple attempts that span a given time period, the auditor
software must alert its operator.  More precisely:
\begin{itemize}
	\item If the log fails to provide a valid inclusion proof for an SCT with
		regards to the first applicable STH in the Tor consensus
		(an MMD violation is suspected).
	\item If the log fails to provide a valid consistency proof between any two
		STHs in the Tor consensus
		(a split-view is suspected within the Tor network).
	\item If an STH in the Tor consensus exceeds the log's MMD (general log
		misbehaviour is suspected).
\end{itemize}

An announced auditor may choose to replicate possible evidence of log
misbehaviour to the other announced auditors prematurely to ensure that
it is persisted beyond its own durable storage.  At some point, the auditor
software should generate a complete report that can be forwarded by the
auditor's operator to the CT-policy mailing list for further community
investigation.
