\section{Design}
Figure~\ref{fig:overview} provides an overview of our design idea.  The starting
point is that zero or more certificate chains and SCTs are presented
throughout a website visit (step~1).  We refer to these as SCT Feedback Objects
(SFOs)~\cite{nordberg}.  Tor Browser accepts an SFO if it adheres to an
SCT-centric CT policy similar to Chrome and Safari~%
\cite{chrome-policy,safari-policy}, but it may also be submitted to a CTR
in the background using a fresh independent circuit (step~2).  CTR flags are
assigned by Tor's directory authorities, and among other things, they also
determine Tor's view of the CT landscape by publishing STHs and announcing
third-party CT auditors in the Tor consensus.  CTRs store SFOs temporarily
before challenging the logs to prove inclusion with regards to Tor's published
STHs (step~3).  Should a log fail to provide a valid inclusion proof for an SFO
within a timely manner, it is reported to an auditor that investigates the issue
further (step~4).
\begin{figure}
	\centering
	\def\ed{0.8} % distance between entities (cm)
\def\orWidth{0.67} % onion router width (cm)
\def\wsWidth{1} % website width (cm)
\def\tbWidth{0.8} % tor browser width (cm)
\def\logWidth{1} % ct log width (cm)
\def\auditorWidth{1} % ct auditor width (cm)
\def\consensusWidth{1} % Tor consensus width (cm)

\begin{tikzpicture}[
	label/.style = {
		draw=none,
		font=\fontsize{6}{6}\rmfamily,
		text=black,
	},
	edges/.style = {
		draw=black,
		thick,
		-latex,
		rounded corners,
	},
]
	%%%
	% ORs in the centre
	%%%
	\node[draw=none](clor){\includegraphics[width=\orWidth cm]{%
		img/overview/or}
	};
	\node[draw=none, right=\ed cm of clor](cmor){%
		\includegraphics[width=\orWidth cm]{img/overview/or}
	};
	\node[draw=none, right=\ed cm of cmor](cror){%
		\includegraphics[width=\orWidth cm]{img/overview/or}
	};

	\coordinate(lc) at ($ (clor) !.5! (cmor) $);
	\coordinate(rc) at ($ (cmor) !.5! (cror) $);

	%%%
	% ORs on the top
	%%%
	\node[draw=none, above=\ed cm of lc](tlor){%
		\includegraphics[width=\orWidth cm]{img/overview/or}
	};
	\node[draw=none, above=\ed cm of rc](tror){}; % hide unnecessary OR

	%%%
	% ORs on the bottom
	%%%
	\node[draw=none, below=\ed cm of lc](blor){%
		\includegraphics[width=\orWidth cm]{img/overview/or}
	};
	\node[draw=none, below=\ed cm of rc](bror){%
		\includegraphics[width=\orWidth cm]{img/overview/or}
	};

	%%%
	% Website
	%%%
	\node[draw=none, right=\ed cm of cror](ws){%
		\includegraphics[width=\wsWidth cm]{img/overview/ws}
	};
	\node[label, above=0pt of ws]{\textbf{Website}};

	%%%
	% Tor browser
	%%%
	\node[draw=none, left=\ed cm of clor](tb){%
		\includegraphics[width=\tbWidth cm]{img/overview/tb}
	};
	\node[label, below=0pt of tb]{\textbf{Tor browser}};

	%%%
	% CT log
	%%%
	\coordinate(tc) at ($ (tlor) !.5! (tror) $);
	\node[draw=none, above=\ed cm of tc](log) {%
		\includegraphics[width=\logWidth cm]{img/overview/log}
	};
	\node[label, above=0pt of log]{\textbf{CT log}};

	%%%
	% Announced auditor
	%%%
	\node[draw=none, right=\ed cm of log](auditor) {%
		\includegraphics[width=\auditorWidth cm]{img/overview/auditor}
	};
	\node[label, above=0pt of auditor]{\textbf{Auditor}};

	%%%
	% Consensus
	%%%
	\node[draw=none, left=\ed cm of log](consensus) {%
		\includegraphics[width=\consensusWidth cm]{img/overview/consensus}
	};
	\path[edges, densely dotted]
		($ (consensus) !.40! (tlor) $) --
		($ (consensus) !.65! (tlor) $);
	\node[label, above=0pt of consensus]{\textbf{Tor consensus}};
	
	%%%
	% 1) Website -> TB
	%%%
	\path[edges]
		(ws) |-
			node[label,below,pos=.75]{1. receive SFO}
		(bror.350);
	\path[edges] (bror.190) -- (blor.350);
	\path[edges] (blor.190) -| (clor);
	\path[edges] (clor.210) to[out=230, in=310] (tb.330);

	%%%
	% 2) TB -> CTR
	%%%
	\path[edges, dashed]
		(tb.30) to[out=10, in=130]
			node[label, above, pos=0.25]{2. submit SFO}
		(clor.150);
	\path[edges, dashed] (clor) |- (tlor.190);
	\path[edges, dashed] (tlor.350) -| (cmor);
	\path[edges, dashed] (cmor.350) -- (cror.190);

	%%%
	% 3) CTR auditing
	%%%
	\path[edges, latex-latex, dotted]
		(cror.135) --
			node[label, below, sloped]{3. audit SFO}
		(log);

	%%%
	% 4) CTR reporting
	%%%
	\path[edges, dashdotted]
		(cror) --
			node[label, above, sloped]{4. report SFO}
			node[label, below, sloped]{on timeout}
		(auditor);
\end{tikzpicture}

	\caption{%
		TODO: docdoc, and label ORs?
	}
	\label{fig:overview}
\end{figure}

\subsection{Tor Consensus}
Tor's directory authorities produce an hourly consensus document that defines
how the Tor network is composed.  Our proposal further requires the following
parameters:
\begin{description}
	\item[ct-get-sth:] Occurs multiple times.  Log ID, followed by an STH that is
		formatted as returned by the log's \texttt{get-sth}
		endpoint~\cite{ct,ct/bis}.
	\item[ct-submit-pr:] Occurs exactly once.  A floating-point in $[0,1]$
		that determines the probability that an SFO should be submitted from Tor
		Browser to a CTR.
	\item[ct-auditor:] Occurs multiple times.  A submission URI, followed by a
		SPKI hash that must be present in the TLS certificate presented by the
		auditor~\cite{hpkp}.
\end{description}

Throughout any given voting timeline, each directory authority is expected to
obtain the most recent STH from all recognized CT logs.  Next, while computing
the consensus, they must agree to use the most recent STH by inspecting the
timestamp field and resolving ties by appealing to lexicographical
order.\footnote{%
	Note that there is cryptographic proof of log misbehaviour if two STHs with
	the same tree size have different root hashes.
}  A log is recognized and a proposed CT auditor is announced if a majority of
all directory authorities voted on their inclusion, which is implicit by
proposing a value.  The submit probability is computed as
the median of all votes.

Other than the new parameters introduced above, we also need to extend the
\texttt{know-flags} parameter as described below.  A relay is assigned the CTR
flag if a majority of directory authorities voted for it.
\begin{description}
	\item[known-flags:] A flag named \texttt{CTR}, which indicates
		that a Tor relay should be used for CT-auditing purposes.  The CTR
		criteria is to be a stable middle relay with some minimum bandwidth.
		\TODO{bw necessary?}
\end{description}
% - A CTR should be stable both to permit long-lived circuits to the CT logs and
% to increase the likelihood of them staying online.
% - The criteria of being a middle relay, as opposed to an exit relay, follows
% mainly from resource utilization considerations.  It is also about not making
% exit relays "more attractive targets" due to also storing SFOs.

\subsection{Tor Browser}
\subsection{CT-Auditing Tor Relays}
\subsection{Announced Auditors}
