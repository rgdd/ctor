\section{Security Analysis} \label{sec:analysis}

\subsection{Impact of Being Detected} \label{sec:analysis:impact}
Upon detection of man-in-the-middle attacks against Tor Browser that used 
mis-issued TLS certificates, we consider four types of impact scenarios for the
attacker:
\begin{description}
	\item[None:] the attack was undetected or detected without knowing any
		details as to how it was carried out.
	\item[Minor:] the attack was detected due to some cover-up that involved
		network-wide actions against Tor.  This is likely hard to attribute to
		the actual attacker, but nevertheless it draws much unwanted attention.
	\item[Significant:] the attack resulted in public cryptographic evidence
		proving CA misbehavior.
	\item[Catastrophic:] the attack resulted in public cryptographic evidence
		proving CA and log misbehavior.
\end{description}

We want to minimize the existence of successful attacks where it cannot be
established if and how they were carried out.  As argued subsequently, our
base design achieves a significant impact level within our threat model.  The
extended designs in Sections~\ref{sec:log}--\ref{sec:auditor} aim higher, but by
making the catastrophic impact level possible, minor impact scenarios follow due
to Tor's threat model.

\subsection{Probability of Being Detected} \label{sec:analysis:pr}
Suppose the attacker mis-issued a certificate chain that is rooted in Tor
Browser's trust store, and that it is accompanied by enough SCTs from
attacker-controlled logs to pass Tor Browser's CT policy.  The attacker's
logs have no intention to keep their promises of public logging.  As such, the
probability that the impact scenario is either none or minor, rather than
significant, is bound by the probability that our design discloses the
attacker's SFO to public scrutiny.  Without any attacker intervention, this
probability is determined by
	the value of \texttt{ct-submit-pr},
	the likelihood that Tor Browser sampled a benign CTR, and
	the likelihood that the CTR sampled a benign independent CT log.
In other words, the risk of exposure can never be higher than the probability
that Tor Browser submits an SFO to a CTR, and the risk decreases proportionally
as the attacker controls larger parts of the Tor network and the CT landscape.

To increase the probability of staying undetected, the attacker must prevent the
mis-issued SFO from reaching an independent CT log that makes it public.  Our
phase-based design forms a weakest-link scenario, in which all phases must
withstand attacker intervention.  At best, we argue that the attacker has a
modest chance of decreasing the probability of detection beyond what could
already be done without any intervention, i.e., by controlling more of the Tor
network or the CT landscape.  Neither of these options are applicable to the
extend that it would shut down auditing entirely though:
	such a powerful attacker would be outside of Tor's threat model or
	violate our premise that independent CT logs exist.
Note that the value of \texttt{ct-submit-pr} is not under adversarial control,
and it should be set to match the risk-averse attacker's capabilities that we
have in mind.

For example, suppose that a risk-averse attacker is deterred from performing a
man-in-the-middle attack if the probability of detection is 5\%, and that
25\% of the Tor network and the CT landscape is adversarial.  Tor Browser's
submission probability should then be in the order of $\frac{1}{10}$.  Clearly,
the risk of exposure increases if the attacker conducts many man-in-the-middle
attacks.

\subsection{Phase Intervention}
