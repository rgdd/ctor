\section{Security Analysis} \label{sec:analysis}

\subsection{Impact of Being Detected} \label{sec:analysis:impact}
We consider four types of impact for an attacker that conducted
HTTPS-based man-in-the-middle attacks on Tor Browser.  Other than \emph{none},
these impact types are:
\begin{description}
	\item[Minor] the attack was detected due to some cover-up that involved
		network-wide actions against CTor.  This is likely hard to attribute to
		the actual attacker, but nevertheless it draws much unwanted attention.
	\item[Significant] the attack generated public cryptographic evidence
		that proves CA misbehavior.
	\item[Catastrophic] the attack generated public cryptographic evidence
		that proves CT log misbehavior.
\end{description}

Our design leads to significant and catastrophic impact events, but does
not preclude minor ones due to Tor's requirement of disk avoidance.  It is
possible to overcome this shortcoming at different trade-offs, e.g., by
tuning CTor parameters reactively (Section~\ref{sec:analysis:pr:phase3}) or by
relying on the scaled-down cross-logging designs in
Section~\ref{sec:incremental}.

\subsection{Probability of Detection} \label{sec:analysis:pr}
Suppose the attacker mis-issued a certificate that Tor Browser trusts, and that
it is considered valid because it is accompanied by enough SCTs from CT logs
that the attacker controls.  The resulting SFO is then used to man-in-the-middle
a single Tor Browser user, i.e., for the purpose of our analysis we consider
\emph{the most risk-averse scenario possible}.  Clearly, none of the attacker's
CT logs plan to keep any promise of public logging:
	that would trivially imply significant impact events.
The risk of exposure is instead bound by the probability that \emph{any} of the
four phases in our design fail to propagate the mis-issued SFO to a pinned CT
auditor that is benign.

\subsubsection{Phase~1: Submission} \label{sec:analysis:pr:phase1}
The probability of detection cannot exceed the value of \texttt{ct-submit-pr}.
We analyze the outcome of submitting the mis-issued SFO from Tor Browser to a
CTR.  There are two cases to consider, namely, the mis-issued SFO is either
larger than \texttt{ct-large-sfo-size} or it is not.

If the SFO is larger than \texttt{ct-large-sfo-size}, Tor Browser blocks until
the SFO is submitted and its CT circuit is closed.  As such, it is impossible to
serve a Tor Browser exploit reactively over the man-in-the-middled connection
that shuts-down the submission procedure before it occurs.  Assuming that
forensic traces in tor and Tor Browser are unreliable,\footnote{%
	``tor'' (aka ``little-t tor'') is the tor process Tor Browser uses to
	interact with the Tor network.  On marking a circuit as closed in tor, tor
	immediately schedules the associated data structures to be freed as soon as
	possible.
} the sampled CTR identity also cannot be revealed with high certainty
afterwards by compromising Tor Browser.  The attacker may know that the SFO is
buffered by \emph{some CTR} based on timing, i.e., blocking-behavior might be 
measurable and distinct.  The important part is not to reveal \emph{which CTR}
received a submission:  a single Tor relay may be subject to DoS.

If the SFO is smaller or equal to \texttt{ct-large-sfo-size} there is a
race between (i) the time it takes for Tor Browser to submit the SFO and close
its CT circuit against (ii) the time it takes for the attacker to compromise Tor
Browser and identify the CTR in question.  It is more advantageous to try and
win this race rather than being in the unfruitful scenario above.  Therefore,
the attacker would maximize the time it takes to perform (i) by sending an SFO
that is \texttt{ct-large-sfo-size}.  Our design reduced the threat of an
attacker that wins this race by using pre-built CT circuits that are closed
immediately after use.  This makes the attack surface \emph{narrow}, limiting
the number of reliable exploits (if any).

Note that the attack surface could, in theory, be eliminated by setting
\texttt{ct-large-sfo-size} to zero.  It is too costly in terms of TLS handshake
latency, however.

\subsubsection{Phase~2: Buffering} \label{sec:analysis:pr:phase2}
The probability of detection cannot exceed $1-(f_{\mathsf{ctr}} +
f_{\mathsf{dos}})$, where $f_{\mathsf{ctr}}$ is the fraction of
malicious CTRs and $f_{\mathsf{dos}}$ the fraction of CTRs that suffer from
DoS.  We analyze the outcome of SFO reception at a genuine CTR.

The time that an SFO is buffered depends on if the log's MMD elapsed or not.
The earliest point in time that a newly issued SCT can be audited (and the log
is expected to respond) is an MMD later, whereas the normal buffer time is
otherwise only governed by smaller randomness in the \texttt{audit\_after}
timestamp.  A rational attacker would therefore maximize the buffer time by
using a newly issued SCT, resulting in an attack window that is \emph{at least}
24~hours for today's CT logs~\cite{google-log-policy}.
%\footnote{%
%	This could be increased further by delaying issuance of an STH that would
%	capture the omission: a log must produce an STH every MMD~\cite{ct,ct/bis}.
%}


Following from Tor's threat model, the mis-issued SFO must be stored in volatile
memory and not to disk.  Two risks emerge due to large buffer times:
	the CTR in question might be restarted by the operator (unrelated to the SFO in question)
	and that given enough time the attacker might find a way to cause the evidence to be deleted.
While a risk-averse attacker cannot rely on the former to avoid detection, we
emphasize that the CTR criteria must include the \texttt{stable} flag to reduce the
probability of this occurring.
There are two ways to induce deletion of the SFO, the first of which is to perform a DoS
attack on the CTR, causing it to shut down or restart. This is not straight forward because
the attacker does not know which CTR
to target.  While a network-wide DoS of all CTRs would be effective, it is not within the
threat model of Tor.  A less intrusive type of DoS would be to \emph{flood} CTRs
by submitting massive amounts of SFOs:
	just enough to make memory a scarce resource, but
	without making Tor unavailable.
This could potentially \emph{flush} a target SFO from the CTR's finite memory
at random (recall Section~\ref{TODO}).  Appendix~\ref{app:flush} shows that the
number of SFO submissions~$k$ that the attacker needs to flush a buffer of $n>2$
entries with some probability~$p\in[0,1)$ is:
\begin{equation} \label{eq:flush}
	k = \frac{\log(1-p)}{\log(1 - \frac{1}{n})}
\end{equation}

It is recommended that a non-exit relay should have at least 512MB of memory.
If the available bandwidth exceeds 40Mbps, it should have at least
1GB~\cite{relay-config}.  Given that these recommendations are lower bounds,
suppose the average memory available to store SFOs is 1GiB (a conservative
assumption).  Further, Section~\ref{sec:performance} shows that the average SFO
size is roughly 6KiB.  This means that the buffer capacity is $n \gets 174763$
SFOs. Plugging it into Equation~\ref{eq:flush} for $p \gets
\frac{9}{10}$, the attacker's flood must involve $k \gets 402406$ submissions.
In other words, 2.3GiB must be transmitted to flush a single CTR with 90\%
success probability,\footnote{%
	As a corner case and implementation detail, it is important that Tor Browser
	and CTRs \emph{reject} SFOs that are bogus in terms of size: it is a trivial
	DoS vector to load data indefinitely. Analysis based on, e.g., 1~MiB SFOs
	also requires 2.3~GiB of data to flush.
} which takes $7.9$--$39.3$~minutes if the relay bandwidth is between 8 and
40~Mbps.  Thus, it is definitely impractical to flush all CTRs within
\emph{minutes}.

Metrics reported by the Tor project show that there are over 4000 relays that
match our CTR criteria set in
Section~\ref{sec:base:consensus:ctr-flag}~\cite{relay-by-flag}.  As such, a
network-wide flush involves the transmission of at least 8.99TiB.  It might
sound daunting at first, but distributed throughout an entire day it only
requires 0.91Gbps.  Such an attack is within our threat model because it does
not make Tor unavailable.  Notably the ballpark of these numbers do not
change to any significant degree by assuming larger success probabilities, e.g.,
$p\gets\frac{99}{100}$ doubles the overhead by a factor of two.  This means
that we cannot avoid early signals to the logs and at the same time rely only on
the finite memory of CTRs to manage network-wide flushes.  The fact that we can
detect this in the extra-info document makes it a minor impact scenario.

TODO: discuss possible \emph{reactions} to network-wide flushes.  For example,
Tor's directory authorities could artificially lower MMDs on-the-fly to, say, 30
minutes if this is an event that keeps occurring in the network.  Most logs
likely included the legit SFOs by this time so there wouldn't be that much of an
open pipe to the auditors (but logs can save face due to early signals).
Another option would be to have a flag that says submit newly issued SCTs
directly to auditors; would affect ~1\% of all web pages if we assume LE world
and that certificates expire on uniform times.

\subsubsection{Phase~3: Auditing} \label{sec:analysis:pr:phase3}
By the time an SFO enters the audit phase the log in question is expected to
respond with a valid inclusion proof.  There is no such proof if the log
violated its MMD, and it is too late to create a split-view that merged the
certificate in time because the CTR's view is already fixed by an STH in the
Tor consensus that captured the log's misbehavior.  In fact, creating any
split-view within Tor is impractical because it requires that the consensus is
forged or that nobody ever checks whether the trusted STHs are consistent.
This leaves two options: the attacker either responds to the query with an
invalid inclusion proof or
	not at all.
The former is detected immediately and starts phase~4, whereas the latter forces
the CTR to first wait for \texttt{ct-watchdog-timeout} to trigger (which is a few seconds
to avoid premature auditor reports).  A rational attacker prefers the second
option to gain more time.

Clearly, the attacker knows that \emph{some} CTR holds evidence of log
misbehavior as it is being audited.  The relevant question is whether the
\emph{exact CTR identity} can be inferred, in which case the attacker could
knock it offline (DoS).  Motivated by the threat of \emph{tagging}, where the
attacker sends unique SFOs to all CTRs so that their identities are revealed
once queried for, we erred on the safe side and built watchdog into our design:
	it is already too late to DoS the querying CTR because the evidence is
	already replicated somewhere else, ready to be reported unless there is a
	timely acknowledgement.
The attacker would have to \emph{break into an arbitrary CTR within seconds} to
cancel the watchdog, which cannot be identified later on (same premise as
the sampled CTR in phase~1).  Such an attacker is not in Tor's threat model.

In other words, the probability that the attacker can intervene by reliable
means in phase~3 is negligible.

\subsubsection{Phase~4: Reporting} \label{sec:analysis:pr:phase4}
As noted in Section~\ref{sec:analysis:pr:phase3}, the watchdog identity is not
known by the attacker and there will be no timely acknowledgment.  This
initiates the process of reporting the SFO in question to a random CT auditor in
the Tor consensus.  Clearly, the probability of detection cannot exceed
$1-f_{\mathsf{auditor}}$, where $f_{\mathsf{auditor}}$ is the fraction of
malicious CT auditors.  The attacker may attempt to (micro)DoS all CT auditors
to win time as the first report takes place at a deterministic point in time
(after the timeout), but besides being outside of our threat model that would
only result in (re)reporting later on \emph{to the same CT auditor}.
While gaining time at this stage is of limited help because the CTR identity is
unknown and network-wide attacks are already more likely to succeed in the
buffer phase, fixating the sampled CT auditor is important to avoid the threat
of eventually succeeding a (re)report only if it is destined to the attacker's
auditor.

Similar to the buffer phase, the detect probability cannot exceed the chance
that a genuine watchdog is sampled to do the reporting on another CTR's behalf.
