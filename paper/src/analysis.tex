\section{Security Analysis} \label{sec:analysis}

\subsection{Impact of Being Detected} \label{sec:analysis:impact}
Given a man-in-the-middle attack against Tor Browser we consider four impact
scenarios for the attacker:
\begin{description}
	\item[None:] the attack was undetected or detected without knowing any
		details as to how it was carried out.
	\item[Minor:] the attack was detected due to some cover-up that involved
		network-wide actions against CTor.  This is likely hard to attribute to
		the actual attacker, but nevertheless it draws much unwanted attention.
	\item[Significant:] the attack generated public cryptographic evidence
		that proves CA misbehavior.
	\item[Catastrophic:] the attack generated public cryptographic evidence
		that proves CT log misbehavior.
\end{description}

We want to minimize the existence of successful attacks where it cannot be
established if and how they were carried out.  As argued subsequently, our
base design achieves a significant impact level within our threat model.  The
extended designs in Sections~\ref{sec:auditor}--\ref{sec:log} aim higher.

\subsection{Probability of Being Detected} \label{sec:analysis:pr}
Suppose the attacker mis-issued a certificate chain that is rooted in Tor
Browser's trust store, and that it is accompanied by enough SCTs from
attacker-controlled CT logs to pass Tor Browser's CT policy.  The resulting SFO
is further used to man-in-the-middle a single Tor Browser user.  Clearly, the
attacker-controlled CT logs have no intention to keep any promise of public
logging as that would trivially imply significant impact.  The attacker's risk
of significant exposure is instead bound by the probability that \emph{any} of
the three phases in our design fails to propagate the mis-issued SFO to a benign
independent CT log.  We analyze each phase separately.

\subsubsection{Phase~1: Submission} \label{sec:analysis:pr:phase1}
The probability of detection can never exceed the value of
\texttt{ct-submit-pr}.  We focus our analysis on the scenario where the
attacker's mis-issued SFO is audited further.  There are two cases to consider,
namely, the mis-issued SFO is either larger than \texttt{ct-max-sfo-bytes} or it
is not.

If the SFO is larger than \texttt{ct-max-sfo-bytes}, Tor Browser blocks until
the SFO is submitted and its CT circuit is closed.  As such, it is
impossible to serve a Tor Browser exploit reactively over the man-in-the-middled
connection that shuts-down the submission procedure before it occurs.  Assuming
that there are no forensic traces in tor and Tor Browser,\footnote{%
	TODO: is there any \emph{reference} based on TB/tor design that makes us
	believe that it is hard to figure out past circuits?
} the sampled CTR identity also cannot be revealed afterwards by compromising
Tor Browser.  The attacker may know that the SFO is being stored by \emph{some
CTR} based on timing, however, i.e., blocking-behavior is measurable and
stands out.  Note that the attacker must not learn a CTR's identity:
	it is well within our threat model to DoS a single Tor relay.

If the SFO is smaller or equal to \texttt{ct-max-sfo-bytes}, then there is a
race between (i) the time it takes for Tor Browser to submit the SFO and close
its CT circuit against (ii) the time it takes for the attacker to compromise Tor
Browser and identify the CTR in question.  It is more advantageous to try and
win this race rather than being in the unfruitful scenario above.  Therefore, we
assume the attacker would maximize the time it takes to perform (i) by sending
an SFO that is \texttt{ct-max-sfo-bytes}.  Section~\ref{sec:todo} analyzes the
timing and work involved while submitting \texttt{ct-max-sfo-bytes} over a CT
circuit that is closed immediately after use.

TODO: wrap-up with how our design reduced the attack surface, i.e., motivate
\texttt{ct-max-sfo-bytes} and pre-built one-time CT circuits.  Explain
why \texttt{ct-submit-pr} is the dominant factor with regards to detection in
phase~1.

\subsubsection{Phase~2: Storage} \label{sec:analysis:pr:phase2}
The probability of detection can never exceed $1-f_{\mathsf{ctr}}$, where
$f_{\mathsf{ctr}}$ is the fraction of attacker-controlled CTRs.  We analyze
the case of submission to a genuine CTR.

The time that an SFO is stored before any auditing takes place is governed by
randomness in the \texttt{audit\_after} timestamp~$t$ as well as the back-off
delay~$d$.  The average storage time is thus in the order of minutes, see
Section~\ref{sec:base:consensus:params}.  The attacker may try to intervene with
a CTR so that the mis-issued SFO is deleted.  However, at this stage it is
generally unknown which CTR should be targeted:
	an attacker that infers the sampled CTR reliably from observing a large
	majority of the Tor network is not within our threat model.
If the target CTR is not known, perhaps all CTRs could be targeted instead.
While a network-wide DoS is also outside of Tor's threat model, the attacker can
interact with all CTRs through their SFO submission endpoints.  This is little
or no threat, however, because a network-wide flush within \emph{minutes} is
impractical.  See section~\ref{sec:todo}.

\subsubsection{Phase~3: Auditing} \label{sec:analysis:pr:phase3}
The probability of detection can never exceed $1-f_{\mathsf{log}}$, where
$f_{\mathsf{log}}$ is the fraction of attacker-controlled CT logs.  We analyze
the case of interacting with genuine CT logs.

While auditing, CTRs attempt to add certificate chains to independent CT logs.
Thus, the attacker does not learn which CTR holds a mis-issued SFO despite
our design reusing a single Tor circuit:
	recall the threat of tagging in Section~\ref{sec:adversary}.
The attacker is aware of all CT logs that are independent, however.  This
introduces the threat of targeting the CT landscape's availability, which has
the adverse effect of increasing the window during which CTRs store SFOs
due to resubmissions.

TODO: note that resubmission is fixed with regards to a log and why this is
important

TODO: wrap-up

\subsubsection{Conclusion}
The probability of detection is determined by 
	the value of \texttt{ct-submit-pr},
	the likelihood that Tor Browser samples a benign CTR, and
	the likelihood that the benign CTR samples a benign CT log.
In other words, the risk of exposure can never be higher than the probability
that Tor Browser submits an SFO to a CTR, and that risk decreases proportionally
as the attacker controls larger parts of the Tor network and the CT landscape.
