\section{Security Analysis} \label{sec:analysis}

\subsection{Impact of Being Detected} \label{sec:analysis:impact}
Upon detection of man-in-the-middle attacks against Tor Browser that used 
mis-issued TLS certificates, we consider four types of impact scenarios for the
attacker:
\begin{description}
	\item[None:] the attack was undetected or detected without knowing any
		details as to how it was carried out.
	\item[Minor:] the attack was detected due to some cover-up that involved
		network-wide actions against Tor.  This is likely hard to attribute to
		the actual attacker, but nevertheless it draws much unwanted attention.
	\item[Significant:] the attack resulted in public cryptographic evidence
		that proves CA misbehavior.
	\item[Catastrophic:] the attack resulted in public cryptographic evidence
		that proves CT log misbehavior.
\end{description}

We want to minimize the existence of successful attacks where it cannot be
established if and how they were carried out.  As argued subsequently, our
base design achieves a significant impact level within our threat model.  The
extended designs in Sections~\ref{sec:auditor}--\ref{sec:log} aim higher.

\subsection{Probability of Being Detected} \label{sec:analysis:pr}
\subsubsection{Phase~1: Submission} \label{sec:analysis:pr:phase1}
\subsubsection{Phase~2: Storage} \label{sec:analysis:pr:phase2}
\subsubsection{Phase~3: Auditing} \label{sec:analysis:pr:phase3}

%Suppose the attacker mis-issued a certificate chain that is rooted in Tor
%Browser's trust store, and that it is accompanied by enough SCTs from
%attacker-controlled logs to pass Tor Browser's CT policy.  The attacker's
%logs have no intention to keep their promises of public logging.  As such, the
%probability that the impact scenario is either none or minor, rather than
%significant, is bound by the probability that our design discloses the
%attacker's SFO to public scrutiny.  Without any attacker intervention, this
%probability is determined by
%	the value of \texttt{ct-submit-pr},
%	the likelihood that Tor Browser sampled a benign CTR, and
%	the likelihood that the CTR sampled a benign independent CT log.
%In other words, the risk of exposure can never be higher than the probability
%that Tor Browser submits an SFO to a CTR, and the risk decreases proportionally
%as the attacker controls larger parts of the Tor network and the CT landscape.
%
%To increase the probability of staying undetected, the attacker must prevent the
%mis-issued SFO from reaching an independent CT log that makes it public.  Our
%phase-based design forms a weakest-link scenario, in which all phases must
%withstand attacker intervention.  At best, we argue that the attacker has a
%modest chance of decreasing the probability of detection beyond what could
%already be done without any intervention, i.e., by controlling more of the Tor
%network or the CT landscape.  Neither of these options are applicable to the
%extent that it would shut down auditing entirely though:
%	such a powerful attacker would be outside of Tor's threat model or
%	violate our premise that independent CT logs exist.
%Note that the value of \texttt{ct-submit-pr} is not under adversarial control,
%and it should be set to match the risk-averse attacker's capabilities that we
%have in mind.
%
%For example, suppose that a risk-averse attacker is deterred from performing a
%man-in-the-middle attack if the probability of detection is 5\%, and that
%25\% of the Tor network and the CT landscape is adversarial.  Tor Browser's
%submission probability should then be in the order of $\frac{1}{10}$.  Clearly,
%the risk of exposure increases if the attacker conducts many man-in-the-middle
%attacks.
%
%\subsection{Intervention}
%A risk-averse attacker needs reliable means to ensure that an SFO is not
%disclosed to public scrutiny.  The starting-point of our phase-based analysis is
%that Tor Browser submitted an SFO to a CTR that intends to relay it to an
%independent CT log.  We discuss how the attacker may (not) intervene for each
%phase separately.
%
%\subsubsection{Phase~1: Tor Browser Submission}
%Suppose that an SFO is selected for submission to a CTR.  There are two possible
%cases to consider, namely, the SFO is larger than \texttt{ct-max-sfo-bytes} or
%it is not.  If the SFO is larger, Tor Browser blocks until the SFO is submitted
%and the circuit is closed.  Assuming that there are no forensic traces in tor
%and Tor Browser, the selected CTR's identity cannot be revealed afterwards by
%compromising Tor Browser (i.e., as a result of loading and running served
%resources).  It should be noted that it is paramount that the attacker
%\emph{does not know} which CTR stores evidence of CA and CT log misbehavior:
%	it is well within Tor's threat model to DoS a single relay.
%
%TODO: reference based on TB/tor design to makes us believe it is hard to figure
%out past circuits?
%
%If the SFO is smaller or equal to \texttt{ct-max-sfo-bytes}, then there is a
%race between (i) the time it takes for Tor Browser to submit the SFO and close
%the circuit against (ii) the time it takes for the attacker to compromise Tor
%Browser and identify the CTR in question based on circuit metadata.  Trying to
%win this race, rather than being in the blocking-mode above, is more
%advantageous as it enables intervention.  Therefore, we assume that
%the attacker would maximize the time it takes to perform (i) by sending an
%SFO that is \texttt{ct-max-sfo-bytes}.
%
%The time and work involved to submit \texttt{ct-max-sfo-bytes} over a circuit
%which is closed as soon as possible is analyzed in Section~\ref{sec:todo}.
%Clearly, we cannot guarantee that there is no attacker that has a zero-day
%exploit that loads, runs, and escalates privileges within \emph{any} non-zero
%time window.
%
%TODO: wrap up with how we minimized attack surface (one-time circuits, close
%asap w/o acknowledgments) without adding latency during normal browsing.
%
%\subsubsection{Phase~2: CTR Mixing}
%
%\subsubsection{Phase~3: CTR Auditing}
