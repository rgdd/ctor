\section{Analysis} \label{sec:analysis}

\subsection{Impact of Being Detected} \label{sec:analysis:impact}
Upon detection of man-in-the-middle attacks against Tor browser that used 
mis-issued TLS certificates, we consider four types of impact scenarios for the
attacker:
\begin{description}
	\item[None:] the attack was undetected or detected without knowing any
		details as to how it was carried out.
	\item[Minor:] the attack was detected due to some cover-up that involved
		network-wide actions against Tor.  This is likely hard to attribute to
		the actual attacker, but nevertheless it draws much unwanted attention.
	\item[Significant:] the attack resulted in public cryptographic evidence
		proving CA misbehavior.
	\item[Catastrophic:] the attack resulted in public cryptographic evidence
		proving CA and log misbehavior.
\end{description}

We want to minimize the existence of successful attacks where it cannot be
established if and how they were carried out.  As argued subsequently, our
base design achieves a significant impact level within our threat model.  The
extended designs in Sections~\ref{sec:log}--\ref{sec:auditor} aim higher, but by
making the catastrophic impact level possible, minor impact scenarios follow due
to Tor's threat model.

\subsection{Sketch of Security}
