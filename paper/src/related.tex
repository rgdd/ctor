%Paul: mixing?
% cite does-ct-break-the-web somewhere as motivation of TB's SCT CT policy
\section{Related Work} \label{sec:related}
Google Chrome and Apple's Safari enforce CT by mandating that every TLS
certificate must be accompanied by two independent
SCTs~\cite{chrome-policy,safari-policy}.  We proposed that TB should
follow suit, but, unlike any other web browser that enforces CT, CTor provides
\emph{concrete next steps} that relax the centralized trust which is otherwise
and evidently misplaced in CT logs~\cite{%
	izenpe-disqualified,%
	venafi-disqualified,%
	gdca1-omission,%
	digicert-log-compromised%
}.  Several proposals surfaced that aim to do better than today's CT
policies, targeting omissions and split-views.

%%%
% Privacy preserving inclusion proofs
%%%
Laurie proposed that inclusion proofs could be fetched over DNS to avoid
additional privacy leaks, i.e., a user's browsing patterns are already exposed
to the DNS resolver but not the logs in the CT landscape~\cite{ct-over-dns}.
CT/bis provides the option of serving stapled inclusion proofs as part of the
TLS handshake in an extension, an OCSP response, or the certificate
itself~\cite{ct/bis}.
Lueks and Goldberg proposed that a separate database of inclusion proofs could
be maintained that supports information-theoretic PIR~\cite{lueks-and-goldberg}.
Kales~\emph{et~al.} improved scalability by reducing the size of each entry
in the PIR database at the cost of transforming logs into multi-tier Merkle
trees, and additionally, showed how the upper tier could be expressed as
a two-server computational PIR database to ensure that any inclusion proof can
be computed privately on-the-fly~\cite{kales}.
Nordberg~\emph{et~al.} avoid inclusion proof fetching by hanging on to presented
SFOs, handing them back to the same origin at a later time~\cite{nordberg}.
In contrast, CTor protects the user's privacy without any persited browser state
by submitting SFOs on independent Tor circuits to CTRs, which in turn add random
noise before there is any log interaction to speak of.  The use of CTRs
enable caching similar to CT-over-DNS, but it does not put the logs in the dark
like PIR could.

%%%
% The same consistent view
%%%
Inclusion proofs are only meaningful if everyone observes the same consistent
STHs. One option is to configure client software with a list of entities that
they should gossip with, e.g., CT monitors~\cite{chase}, or, browser vendors
could push a verified view~\cite{sth-push}. Such trusted auditor relationships
may work for some but not others~\cite{nordberg}. Chuat~\emph{et~al.} proposed
that HTTPS clients and HTTPS servers could pool STHs and consistency proofs
which are gossiped on website visits~\cite{chuat}. Nordberg~\emph{et~al.}
suggested a similar variant, reducing the risk of user tracking by pooling fewer
and recent STHs~\cite{nordberg}. Dahlberg~\emph{et~al.} noted that such
privacy-insensitive STHs need not be encrypted, which could enable network
operators to use programmable data planes to provide gossip
as-a-service~\cite{dahlberg}. Syta~\emph{et~al.} proposed an alternative to
reactive gossip mechanisms by showing how an STH can be cosigned efficiently by
many independent witnesses~\cite{syta}. A scaled-down version of witness
cosigning could be instantiated by cross-logging STHs in other CT
logs~\cite{minimal-gossip}, or, in other append-only ledgers~\cite{catena}.
CTor's full design ensures that anyone connected to the Tor network is on the
same view by making STHs public in the Tor consensus.  In contrast, the first
incremental design is not concerned with catching log misbehavior, while the
second incremental design exposes misbehaving logs \emph{without} inclusion
proofs.

%%%
% Other work that is closely related to our approach
%%%
Nordberg proposed that Tor clients could enforce public logging of consensus
documents and votes~\cite{consensus-transparency}.  Such an initiative is
mostly orthogonal to CTor, as it strengthens the assumption of a secure Tor
consensus by enabling detection of compromised signing keys rather than
mis-issued TLS certificates.  Winter~\emph{et~al.} proposed that TB
could check self-signed TLS certificates for extact matches on independent Tor
circuits~\cite{spoiled-onions}.  Alicherry~\emph{et~al.} proposed that any web browser could
double-check TLS certificates on first encounter using alternative paths and
Tor, again, looking for certificate mismatches and generating warnings of
possible man-in-the-middle attacks~\cite{doublecheck}.  The submission phase in
CTor is similar to such double-checking, expect that there is no normal-case TLS
handshake blocking, browser warnings, or strict assumptions regarding the
attacker's location.
