%Paul: mixing?
% cite does-ct-break-the-web somewhere as motivation of TB's SCT CT policy
\section{Related Work} \label{sec:related}
Google's Chrome and Apple's Safari enforce CT by mandating that every TLS
certificate must be accompanied by two independent
SCTs~\cite{chrome-policy,safari-policy}.  We proposed that Tor Browser should
follow suit, but, unlike any other web browser that enforces CT, CTor provides
\emph{concrete next steps} that relax the centralized trust which is otherwise
and evidently misplaced in CT logs~\cite{%
	izenpe-disqualified,%
	venafi-disqualified,%
	gdca1-omission,%
	digicert-log-compromised%
}.

%%%
% Privacy preserving inclusion proofs
%%%
Laurie proposed that inclusion proofs could be fetched over DNS to avoid
additional privacy leaks, i.e., a user's browsing patterns are already exposed
to the DNS resolver but not the logs in the CT landscape~\cite{ct-over-dns}.
CT/bis provides the option of serving stapled inclusion proofs as part of the
TLS handshake in an extension, an OCSP response, or the certificate
itself~\cite{ct/bis}.
Lueks and Goldberg proposed that a separate database of inclusion proofs could
be maintained that supports information-theoretic PIR~\cite{lueks-and-goldberg}.
Kales~\emph{et~al.} improved scalability by reducing the size of each entry
in the PIR database at the cost of transforming logs into multi-tier Merkle
trees, and additionally, showed how the upper tier could be expressed as
a two-server computational PIR database to ensure that any inclusion proof can
be computed privately on-the-fly~\cite{kales}.
Nordberg~\emph{et~al.} avoid inclusion proof fetching by hanging on to presented
SFOs, handing them back to the same origin at a later time~\cite{nordberg}.

In contrast, CTor assumes a stateless web browser that protects the user's
privacy by submitting each SFO on an independent Tor circuit to a CTR, which in
turn adds random noise before there is any log interaction to speak of.  The
added noise is unlike CT-over-DNS, which otherwise shares some similarities
with regards to client-wide caching.  On the premise that the database servers
do not collude, PIR provides superior privacy properties.  Such a trust
assumption is similar to today's SCT-centric CT policies, which CTor improves
upon by adding \emph{resilience} as a base and \emph{detection} as extension.

TODO: related work on STHs

TODO: related work for Tor Browser
