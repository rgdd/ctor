\section{Conclusion} \label{sec:conclusion} 
We proposed CTor, a privacy-preserving and incrementally-deployable design that
brings CT to Tor.  Tor Browser should start by taking the same proactive
security measures as Google Chrome and Apple's Safari:
	require that a certificate is only valid if accompanied by two
	independent SCTs.
Such CT enforcement narrows down the attack surface from the weakest-link
security of the CA ecosystem to a relatively small number of trusted log
operators \emph{without negatively impacting the user experience to an
unacceptable degree}.  The problem is that a powerful attacker may gain control
of the required logs, trivially circumventing enforcement without significant
risk of exposure.  If deployed incrementally, CTor relaxes the currently
deployed trust assumption by distributing it across all CT logs.  If the full
design is put into operation, such trust is completely eliminated.

CTor repurposes Tor relays to ensure that today's trust in CT logs is not
misplaced:
	Tor Browser probabilistically submits the encountered certificates and SCTs
	to Tor relays, which
		cross-log them into independent CT logs (incremental design)
		or request inclusion proofs with regards to a single fixed view
			(full design).
It turns out that delegating verification to a party that can defer it
is paramount in our setting, both for privacy and security.  Tor and the wider
web would greatly benefit from each design increment.  The full design turns Tor
into a
system for maintaining a probabilistically-verified view of the entire CT log
ecosystem, provided in Tor's consensus for anyone to use as a basis of trust.
The idea to cross-log certificates and SCTs further showcase how certificate
mis-issuance and suspected log misbehavior could be disclosed casually without
any manual intervention by using the log ecosystem against the attacker.

The attacker's best bet to break CTor involves any of the following:
	operating significant parts of the CTor infrastructure,
	spending a reliable Tor Browser zero-day that escalates privileges within a
		tiny time window, or
	targeting all Tor relays in an attempt to delete any evidence of certificate
		mis-issuance and log misbehavior.
The latter---a so-called network-wide flush---brings us to the border of our
threat model, but it cannot be ignored due to the powerful attacker that we
consider.  Therefore, CTor is designed so that Tor can \emph{adapt} in response
to interference.  For example, in Tor Browser the \texttt{ct-large-sfo-size}
could be set reactively such that all SFOs must be sent to a CTR before
accepting any HTTPS application-layer data to counter zero-days, and the submit
probability \texttt{ct-submit-pr} could be increased if ongoing attacks are
suspected.  When it comes to the storage phase, the consensus can minimize or
maximize the storage time by tuning a log's MMD in the \texttt{ct-log-info}
item.  The distribution that adds random buffering delays could also be updated,
as well as log operator relationships during the auditing phase.
