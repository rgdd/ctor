\section{Conclusion} \label{sec:conclusion} 

We have proposed CTor, privacy-preserving and incrementally-deployable support
for CT in Tor.  Our design is composed of a base that discloses mis-issued
certificates to public scrutiny by using the CT landscape against the attacker,
and two orthogonal extensions that additionally catch misbehaving CT logs.  The
use of Tor relays that cache Tor Browser-observed SFOs and add random auditing delays is
central within our setting, for both privacy and security.  Our analysis shows
that the attacker's best bet to break the base design and the following log
extension is to control and/or DoS a large majority of the entire CTor system,
or, access a reliable Tor Browser zero-day that escalates privileges within a tiny time
window.  Our auditor extension does not rely on trust in the CT landscape, but
at the cost of making flooding attacks that flush the network practical.
Network-wide flushes are trivially detectable but not necessarily attributable.

Mitigation of network-wide attacks take us outside of Tor's threat model and
therefore outside of ours too.  That said, we cannot ignore such attacks given
the strong attacker we consider, namely, one that controls a trusted CA and
CT logs.  To this end, CTor is designed so that Tor can \emph{adapt} in response
to interference.  For example, in Tor Browser the \texttt{ct-max-sfo-bytes} could
reactively be set such that all SFOs must be sent to a CTR before accepting any
HTTPS application-layer data to counter zero-days, and the submit probability
\texttt{ct-submit-pr} could be increased if ongoing attacks are suspected.  When
it comes to the storage phase, the consensus can minimize or maximize the
storage time by tuning a log's MMD in the \texttt{ct-log-info} item, as well as
by updating the distributions that add random auditing delays.  
The behavior of the
auditing phase can also be tuned, e.g., by updating log operator relationships.

Our base design lets Tor play a vital role in CT's overall goal of detecting
certificate mis-issuance.  It also exemplifies a new style of auditing, where
certificate chains are cross-logged rather verified for inclusion.  The extended
versions of CTor involves increased deployment burdens, either within Tor or the
CT landscape.  The wider web could greatly benefit from both, however.  The log
extension provides a single coherent place to report suspected log misbehavior,
making it similar to a well-known auditor.  The auditor extension turns Tor into
a system for maintaining a probabilistically-verified view of the entire CT
landscape, provided in Tor's consensus for anyone to use as a basis of trust on
the wider web.
