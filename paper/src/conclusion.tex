\section{Conclusion} \label{sec:conclusion}
CTor consists of a base design and two possible extensions with the goal of
adding incremental and privacy-preserving support for Certificate Transparency
to Tor. The use of relays in the Tor network distributes caching of observed
SFOs and delays interactions with CT logs, central to both overall security and
preserving privacy of Tor users. Our analysis also shows that these relays are
the weakest link: the most promising chance of avoiding exposure of compromising
SFOs appears to be to attempt to perform network-wide DoS on relays, or attack
CT logs directly. 

Mitigation of network-wide attacks take us outside of Tor's---and therefore also
our---threat model. That said, we cannot ignore such attacks given the strong
attacker we consider (to some degree in control of both a trusted CA and CT
logs). Several parameters of our design enables Tor to \emph{adapt} to observed
interference with CTor, such as a network-wide DoS of relays, reported targeted
attacks of CT logs, or relays reporting suspected flooding through the
\texttt{ct-delete-bytes}  extra-info. When it comes to TB, the submission
probability (\texttt{ct-submit-pr}) and SFO size threshold
(\texttt{ct-max-sfo-bytes}) could be set such as to force all SFOs to be sent to
a CTR before accepting any new HTTPS application-layer data. During the storage
phase, the consensus specifies each CT log's MMD and the delay distribution
(\texttt{ct-delay-dist}), which can be set to either minimize or maximize the
delay between TB user and CT log interaction. Similarly, trusted auditors
(auditor extension) and log operator relationships (base design and log
extension) are also defined in the consensus and can be tweaked.

Deploying CTor, in particular with the log extension that requires trusted
auditors, would be a significant operational burden. Overall Tor network health
would have to include considerations of tweaking CTor parameters to adapt to
strong attackers. However, the potential gains are significant. Tor users would
benefit from the significant security improvement provided by CT logs. Perhaps
more significantly, Tor would be a system for maintaining a
probabilistically-audited cryptographically-verifiable view of the entire CT log
ecosystem available from Tor’s consensus. This would benefit the wider web and
Internet, since the view from Tor's consensus could serve as a base of trust,
relaxing the necessary trust that currently has to be placed on CT log
operators. As a starting point, our base design turns Tor Browser into a helpful
participant in addressing the weakest-link issue of the CA ecosystem, in line
with the goals of CT.
