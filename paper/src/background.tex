\section{Background} \label{sec:background}

\subsection{Certificate Transparency} \label{sec:background:ct}
The idea to transparently log TLS certificates emerged at Google in response to
a lack of proposals that could be deployed without drastic ecosystem changes
and/or significant downsides~\cite{ct/a}.  In reality, CT is about logging
certificate \emph{chains}:
	a domain owner's certificate is signed by an intermediate CA, whose
	certificate is in turned signed by a root CA that acts as a trust
	anchor~\cite{ca-ecosystem}.
The resulting certificate chain is composed of the domain owner's leaf
certificate, the intermediate CA certificate, and the root CA certificate.
Trust anchors are shipped in software, such as web browsers and operating
systems.

\subsubsection{Cryptographic Foundation}
The operator of a CT log maintains a tamper-evident append-only Merkle
tree~\cite{ct,ct/bis}.  At any time, a Signed Tree Head (STH) can be produced
which fixes the log's structure and content.  Important attributes of an STH
include
	the tree head (a cryptographic hash),
	the tree size (a number of entries), and
	the current time.
Given two tree sizes, a log can produce a \emph{consistency proof} that proves
the newer tree head entails everything that the older tree head does.  As such,
anyone can verify that the log is append-only without downloading all entries
and recomputing the tree head.  Membership of an entry can also be proven
by producing an \emph{inclusion proof} for an STH\@.  These proof techniques are
formally verified~\cite{secure-logging-and-ct}.

Upon a valid request, a log must add an entry and produce a new STH that covers
it within a time known as the Maximum Merge Delay (MMD), e.g., 24~hours.  This
policy aspect can be verified because in response, a Signed Certificate
Timestamp (SCT) is returned.  An SCT is a signed promise that an entry will
appear in the log within an MMD.  A log that violates its MMD is said to perform
an \emph{omission attack}.  It can be detected by challenging the log to prove
inclusion.  A log that forks, presenting one append-only version
to some entities and another to others is said to perform a \emph{split-view
attack}.  Split-views can be detected by STH
gossip~\cite{chuat,dahlberg,nordberg,syta}.

\subsubsection{Standardization}
The standardized CT protocol defines public HTTP(S) endpoints that allow anyone
to check the log's accepted trust anchors and added certificates, as well as
obtaining the most recent STH and fetching proofs~\cite{ct,ct/bis}.  For
example, the \texttt{add-chain} endpoint returns an SCT if the added
certificate chain ends in a trust anchor returned by the \texttt{get-roots}
endpoint.  CAs mostly add certificate chains because Google's Chrome and
Apple's Safari require at least two SCTs for an otherwise valid certificate
chain to be accepted~\cite{chrome-policy,safari-policy}.  The log's
\texttt{add-chain} endpoint is open for anyone, however.  We use it in
Section~\ref{sec:base}.

We also use the log's \texttt{get-proof-by-hash} and \texttt{get-sth} endpoints
in Section~\ref{sec:auditor}.  The former takes as input a leaf certificate hash
as well an STH's tree size.  The tree size indicates which tree head the log
should base its inclusion proof on.  The proof is valid if it can be used in
combination with the certificate in question to reconstruct the tree head in the
log's original STH.

\subsubsection{Verification}
The CT landscape provides a limited value unless it is verified that the logs
play by the rules.  While browser vendors have a large say in policy aspects
such as a log's uptime requirements and accepted trust anchors, it is evident
that the most fundamental forms of cheating are omission and split-view
attacks:
	it would undermine detection of mis-issued TLS certificates.
A party that follows-up on inclusion and consistency proofs is said to
\emph{audit} the logs.  Wide-spread client-side auditing is a premise for CT
logs to be untrusted, but none of the web browsers that enforce CT
engage in such activities yet.  For example, it is difficult due privacy
concerns~\cite{ct-with-privacy}.  CT also assumes that domain owners
\emph{monitor} the logs for mis-issuance by inspecting the added
certificates~\cite{lwm,ct-monitors}.

\subsection{Tor} \label{sec:background:tor}

Most of the activity of Tor's millions of daily users starts with Tor
Browser (TB) and connects to some ordinary website via a circuit
comprised of three randomly-selected Tor relays. In this way no
identifying information from Internet protocols (such as IP address)
are automatically provided to the destination, and no single entity
can observe both the source and destination of a connection. TB is
also configured and performs some filtering to resist browser
fingerprinting, and first party isolation to resist sharing state or
linking of identifiers across origins. More generally it avoids
storing identifying configuration and behavioral information to disk.

Tor relays in a circuit are selected at random, but not uniformly. A
typical circuit is comprise of a \emph{guard}, a \emph{middle}, and an
\emph{exit}. A guard is selected by a client and used for several
months as the entrance to all Tor circuits. If the guard is not owned
by an adversary, that adversary will not find itself selected to be on
a Tor circuit adjacent to (thus identifying) the client. And because
some relay operators do not wish to act as the apparent Internet
source for connections to arbitrary websites, relay operators can
configure the ports (if any) on which they will permit connections
besides to other Tor relays. Finally, to facilitate load balancing,
relays are assigned a weight based on their apparent capacity to carry
traffic. In keeping with avoiding storing of linkable state, even
circuits that share an origin will only permit new connections over
that circuit for ten minutes. After that, if all connections are
closed, all state associated with the circuit is cleared. Similarly,
at each relay state associated with a given circuit is cleared shortly
after a circuit is closed.

Tor clients use this information when choosing relays with which to
build a circuit. They receive the information via an hourly updated
\emph{consensus}. The consensus assigns weight as well as flags such
as \texttt{guard} or \texttt{exit} along with auxiliary flags such as
\texttt{stable}, which, e.g., is necessary to obtain the
\texttt{guard} flag since guards must have good availability. The
consensus is determined by a set of \emph{directory authorities} that
vote on various components making up the shared view of the state of
the Tor network. Making sure that all clients have a consistent view
of the network prevents epistemic attacks wherein clients can be
separated based on the routes that are consistent with their
understanding~\cite{danezis:pets2008}.  This is only a very rough
sketch of Tor's design and operation.  More details can be found by
following links at Tor's documentation site~\cite{tor-documentation}.

Tor does not prevent end-to-end correlation attacks. An adversary
controlling the guard and exit, or controlling the destination and
observing the client ISP, etc. is assumed able to confirm who is
connected to whom on that particular circuit. The Tor adversary model
assumes an adversary able to control and/or observe a small to
moderate fraction of Tor relays measured by both number of relays and
by consensus weight, assumes a large number of Tor clients able to,
for example, flood individual relays to detect traffic signatures of
honest traffic on a given circuit~\cite{long-paths}. Also, the
adversary can knock any small number of relays offline via either
attacks from clients or direct Internet DDoS\@. Typically, a
Tor adversary at a destination is not assumed capable of completely
taking over a Tor client within moments of being connected. We
assume a stronger adversary, however, who can do so, e.g., via a zero-day.
Similarly, our adversary is stronger in that it can take over
an individual relay within a minute of attempting to do so. We will say
more about our adversary in Section~\ref{FIXME}. 




%%% Local Variables: 
%%% mode: latex 
%%% TeX-master: "../main"
%%% End:          
