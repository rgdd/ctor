\section{Background} \label{sec:background}
The theory and current practise of CT is introduced first, then Tor
and its privacy-preserving Tor Browser.

\subsection{Certificate Transparency} \label{sec:background:ct}
The idea to transparently log TLS certificates emerged at Google in response to
a lack of proposals that could be deployed without drastic ecosystem changes
and/or significant downsides~\cite{ct/a}.  By making the set of issued
certificate chains\footnote{%
	A domain owner's certificate is signed by an intermediate CA, whose
	certificate is in turned signed by a root CA that acts as a trust
	anchor~\cite{ca-ecosystem}.  Such a \emph{certificate chain} is valid if it
	ends in a trusted anchor that is shipped in the user's system software.
} transparent, anyone that inspect the logs can detect certificate
mis-issuance \emph{after the fact}.  It would be somewhat circular to solve
issues in the CA ecosystem by adding trusted CT logs.  Therefore, the
cryptographic foundation of CT is engineered to avoid any such reliance.
Google's \emph{gradual} CT rollout started in 2015, and evolved from merely
downgrading user-interface indicators in Chrome to the current state of hard
failures unless a certificate is accompanied by a signed \emph{promise} that it
will appear in two independent CT logs~\cite{does-ct-break-the-web}.  Unlike
Apple's CT enforcement~\cite{safari-policy}, one of these logs must additionally
be operated by Google~\cite{chrome-policy}.

The lack of mainstream verification, i.e., beyond checking signatures, allows an
attacker to side-step the current CT enforcement with minimal risk of exposure
\emph{if the required logs are controlled by the attacker}.  
CTor integrates into the gradual CT roll-out by starting on the same
premise of pairwise-independently trusted logs, which
avoids the risk of bad user experience~\cite{does-ct-break-the-web}
and significant system complexity.  For example, web pages are unlikely to
break, TLS handshake latency stays about the same, and no robust management of
suspected log misbehavior is needed.  Retaining the latter property as part of
our incremental designs simplifies deployment.

\subsubsection{Cryptographic Foundation}
The operator of a CT log maintains a tamper-evident append-only Merkle
tree~\cite{ct,ct/bis}.  At any time, a Signed Tree Head (STH) can be produced
which fixes the log's structure and content.  Important attributes of an STH
include
	the tree head (a cryptographic hash),
	the tree size (a number of entries), and
	the current time.
Given two tree sizes, a log can produce a \emph{consistency proof} that proves
the newer tree head entails everything that the older tree head does.  As such,
anyone can verify that the log is append-only without downloading all entries
and recomputing the tree head.  Membership of an entry can also be proven
by producing an \emph{inclusion proof} for an STH.  These proof techniques are
formally verified~\cite{secure-logging-and-ct}.

Upon a valid request, a log must add an entry and produce a new STH that covers
it within a time known as the Maximum Merge Delay (MMD), e.g., 24~hours.  This
policy aspect can be verified because in response, a Signed Certificate
Timestamp (SCT) is returned.  An SCT is a signed promise that an entry will
appear in the log within an MMD.  A log that violates its MMD is said to perform
an \emph{omission attack}.  It can be detected by challenging the log to prove
inclusion.  A log that forks, presenting one append-only version
to some entities and another to others, is said to perform a \emph{split-view
attack}.  Split-views can be detected by STH
gossip~\cite{chuat,dahlberg,nordberg,syta}.

\subsubsection{Standardization and Verification}
The standardized CT protocol defines public HTTP(S) endpoints that allow anyone
to check the log's accepted trust anchors and added certificates, as well as
to obtain the most recent STH and to fetch proofs~\cite{ct,ct/bis}.  For
example, the \texttt{add-chain} endpoint returns an SCT if the added certificate
chain ends in a trust anchor returned by the \texttt{get-roots} endpoint.  We
use \texttt{add-chain} in Section~\ref{sec:incremental}, as well as several
other endpoints in Section~\ref{sec:base} to fetch proofs and STHs.  It might be
helpful to know that an inclusion proof is fetched based on two parameters: a
certificte hash and the tree size of an STH.  The former specifies the log entry
of interest, and the latter with regards to which view inclusion should be
proven.  The returned proof is valid if it can be used in combined with the
certificate to reconstruct the STH's tree head.

The CT landscape provides limited value unless it is verified that the logs
play by the rules.  The rules are influenced by the major browser vendors that
define \emph{CT policies}.  For example, how is CT enforced by their user
agents, and what is required to become a recognized CT log in terms of uptime
requirements and accepted trust anchors.  Google Chrome and Apple's Safari
require that a certificate chain must be accompanied by two independent
SCTs~\cite{chrome-policy,safari-policy}, where independence refers to the
associated log operators.  While there are several ways that a log can misbehave
with regards to these policy aspects, the most fundamental forms of cheating are
omission and split-view attacks.  A party that follows-up on inclusion and
consistency proofs is said to \emph{audit} the logs.

Widespread client-side auditing is a premise for CT logs to be untrusted, but
none of the web browsers that enforce CT engage in such activities yet.  For
example, requesting an inclusion proof is privacy-invasive because it leaks
browsing patterns to the logs, and reporting suspected log misbehavior comes
with privacy~\cite{ct-with-privacy} as well as operational challenges.
Found log incidents are mostly reported manually to the CT policy
list~\cite{ct-policy-mailing-list}.  This is in contrast to automated
\emph{CT monitors}, which notify domain owners
of newly issued certificates based on what actually appeared in the public
logs~\cite{lwm,ct-monitors}.

\subsection{Tor} \label{sec:background:tor}

Most of the activity of Tor's millions of daily users starts with Tor Browser
and connects to some ordinary website via a circuit comprised of three
randomly-selected Tor relays. In this way no identifying information from
Internet protocols (such as IP address) are automatically provided to the
destination, and no single entity can observe both the source and destination of
a connection. Tor Browser is also configured and performs some filtering to resist
browser fingerprinting, and first party isolation to resist sharing state or
linking of identifiers across origins. More generally it avoids storing
identifying configuration and behavioral information to disk.

Tor relays in a circuit are selected at random, but not uniformly. A typical
circuit is comprised of a \emph{guard}, a \emph{middle}, and an \emph{exit}. A
guard is selected by a client and used for several months as the entrance to all
Tor circuits. If the guard is not controlled by an adversary, that adversary
will not find itself selected to be on a Tor circuit adjacent to (thus
identifying) the client. And because some relay operators do not wish to act as
the apparent Internet source for connections to arbitrary destinations, relay
operators can configure the ports (if any) on which they will permit connections
besides to other Tor relays. Finally, to facilitate load balancing, relays are
assigned a weight based on their apparent capacity to carry traffic. In keeping
with avoiding storing of linkable state, even circuits that share an origin will
only permit new connections over that circuit for ten minutes. After that, if
all connections are closed, all state associated with the circuit is cleared.
Similarly, at each relay state associated with a given circuit is cleared
shortly after a circuit is closed.

Tor clients use this information when choosing relays with which to build a
circuit. They receive the information via an hourly updated \emph{consensus}.
The consensus assigns weights as well as flags such as \texttt{guard} or
\texttt{exit}. It also assigns auxiliary flags such as
\texttt{stable}, which, e.g.,
is necessary to obtain the \texttt{guard} flag since guards must have good
availability. Self-reported information by relays in their \emph{extra-info
document}, such as statistics on their read and written bytes, are also part of
the consensus and uploaded to \emph{directory authorities}. Directory
authorities determine the consensus by voting on various components making up
the shared view of the state of the Tor network. Making sure that all clients
have a consistent view of the network prevents epistemic attacks wherein clients
can be separated based on the routes that are consistent with their
understanding~\cite{danezis:pets2008}. This is only a very rough sketch of Tor's
design and operation.  More details can be found by following links at Tor's
documentation site~\cite{tor-documentation}.

Tor does not aim to prevent end-to-end correlation attacks. An adversary
controlling the guard and exit, or controlling the destination and observing the
client ISP, etc. is assumed able to confirm who is connected to whom on that
particular circuit. The Tor threat model assumes an adversary able to control
and/or observe a small to moderate fraction of Tor relays measured by both
number of relays and by consensus weight, and it assumes a large
number of Tor clients
able to, for example, flood individual relays to detect traffic signatures of
honest traffic on a given circuit~\cite{long-paths}. Also, the adversary can
knock any small number of relays offline via either attacks from clients or
direct Internet DDoS. 
