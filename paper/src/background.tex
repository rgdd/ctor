\section{Background} \label{sec:background}

\subsection{Certificate Transparency} \label{sec:background:ct}
The idea to transparency log TLS certificates emerged at Google in response to
a lack of proposals that could be deployed without drastic ecosystem changes
and/or significant downsides~\cite{ct/a}.  In reality, CT is about logging
certificate \emph{chains}:
	a domain owner's certificate is signed by an intermediate CA, whose
	certificate is in turned signed by a root CA that acts as a trust
	anchor~\cite{ca-ecosystem}.
The resulting certificate chain is composed of the domain owner's leaf
certificate, the intermediate CA certificate, and the root CA certificate.
Trust anchors are shipped in software, such as web browser and operating
systems.

\subsubsection{Cryptographic Foundation}
The operator of a CT log maintains a tamper-evident append-only Merkle
tree~\cite{ct,ct/bis}.  At any time, a Signed Tree Head (STH) can be produced
which fixes the log's structure and content.  Important attributes of an STH
include
	the tree head (a cryptographic hash),
	the tree size (a number of entries), and
	the current time.
Given two tree sizes, a log can produce a \emph{consistency proof} that proves
the newer tree head entails everything that the older tree head does.  As such,
anyone can verify that the log is append-only without downloading all entries
and recomputing the tree head.  Membership of an entry can also be proven
by producing an \emph{inclusion proof} for an STH.  These proof techniques are
formally verified~\cite{secure-logging-and-ct}.

Upon a valid request, a log must add an entry and produce a new STH that covers
it within a time known as the Maximum Merge Delay (MMD), e.g., 24~hours.  This
policy aspect can be verified because in response, a Signed Certificate
Timestamp (SCT) is returned.  An SCT is a signed promise that an entry will
appear in the log within an MMD.  A log that violates its MMD is said to perform
an \emph{omission attack}.  It can be detected by challenging the log to prove
inclusion.  A log that forks, presenting one append-only version
to some entities and another to others is said to perform a \emph{split-view
attack}.  Split-views can be detected by STH
gossip~\cite{chuat,dahlberg,nordberg,syta}.

\subsubsection{Standardization}
The standardized CT protocol defines public HTTP(S) endpoints that allow anyone
to check the log's accepted trust anchors and added certificates, as well as
obtaining the most recent STH and fetching proofs~\cite{ct,ct/bis}.  For
example, the \texttt{add-chain} endpoint returns an SCT if the added
certificate chain ends in a trust anchor returned by the \texttt{get-roots}
endpoint.  CAs mostly add certificate chains because Google's Chrome and
Apple's Safari require at least two SCTs for an otherwise valid certificate
chain to be accepted~\cite{chrome-policy,safari-policy}.  The log's
\texttt{add-chain} endpoint is open for anyone, however.  We use it in
Section~\ref{sec:base}.

\subsubsection{Status Quo}
The CT ecosystem provides little or no value unless someone verifies that the
logs play by the rules.  At minimum, there should be no omission and split-view
attacks:
	it would undermine detection of mis-issued certificates.
Web browsers specify further policy requirements, such as uptime and trust
anchor constraints~\cite{chrome-policy,safari-policy}.  A party that follows-up
on inclusion and consistency proofs \emph{audits} the logs.  While 
wide-spread existence of client-side auditing is a premise for CT logs to be
untrusted, none of the CT-enforcing web browsers engage in such activities yet.
CT also assumes that domain owners monitor the logs for mis-issuance by
downloading added certificate chains.  Third-party monitors
exist~\cite{ct-monitors}.

\subsection{Tor} \label{sec:background:tor}
Consensus, relays and their roles, Tor's threat model, Tor Browser's key
design goals (disk avoidance and no shared state)
