\section{Attacker Risk} \label{sec:risk}
We assume a risk-averse attacker: we want to make the \emph{probability} of
detection as well as the \emph{impact} of detection as high as possible.

\subsection{Impact of Being Detected}

% TODO Tobias: rework below to make the following split more clear
% - if network-wide attack, significant event but relatively hard to attribute
% - if misbehaving CA (how do we get to this point?), CA burned
% - if misbehaving CT log, log burned (massive)

In general, the impact of the public detecting that some attack is ongoing may
be modest. The key impact is when an inconsistent SFO is found, since it is
signed. An inconsistent SFO indicates one of two things: either the issuing CA
messed up, or it was the CT log. Since a log has a MMD of time since the SFO to
act, it is likely a good idea to wait at least MMD before auditing. This will
maximize the impact, since proving log misbehaviour is significantly more costly
than for a CA. This is a strength of our proposal we need to lean on: it's
devastatingly expensive for an attacker, especially if we want to make dragnet
surveillance harder, not few targeted attacks.

\subsection{Probability of Being Detected}
Given our design and threat model, consider the case of an attacker that
attempts to MitM a single connection by creating an inconsistent SFO. Let $p_a$
be the probability of an attacker being detected. This probability will be bound
by:

\begin{itemize}
    \item $p_c$, the probability that a client (TB) will successfully send the
    SFO to a CTR. Assuming that our design makes it unlikely that an attacker
    has time to compromise the client and prevent the SFO from being sent, this
    probability is dominated by the sending probability of TB set in the
    consensus.
    \item $p_h$, the probability that the CTR that receives the SFO is honest
    (i.e., not controlled by the attacker). This we make assumptions about given
    Tor's threat model.
    \item $p_s$, the probability that the CTR successfully sends the SFO to an
    auditor. The hardest probability to estimate because it includes flushing
    attacks, tagging attacks, and global network DoS.
    \item $p_r$, the probability that the auditor that receives the SFO is
    honest (i.e., not controlled by the attacker). This we make assumptions
    about being low.
\end{itemize}

We can express this as:  

\begin{align}
    p_a \leq p_c p_h p_s p_r
\end{align}

Our design assumes that $p_c$ will be low, in the order of $\approx0.1$. An
attacker that controls a fraction $f$ of all CTRs in the network has $p_h =
1-f$. We may reasonably assume that $p_r \approx 1$ with proper vetting by the
Tor Project (if few trusted auditors are OK). This leaves $p_s$.

Reasoning about $p_s$ is complex. Infrequently causing network-wide DoS or
trivially \emph{detectable} flushing may be acceptable to some attackers
compared to the impact of SFO detection (see following section). In such a case,
$p_s \approx 0$ and the rest is moot. 

We need to focus on reliable ways for an attacker to drop $p_s$, i.e., ways to
prevent SFOs to being sent to auditors without causing any detectable
abnormalities in the network. For this flushing is important, since it's an
attack that has no other impact on the network beyond the CTor functionality.
The same is true for DoSing the auditors. 
