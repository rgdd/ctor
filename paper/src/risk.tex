\section{Attacker Risk}
We assume a risk-averse attacker: we want to make the \emph{probability} of
detection as well as the \emph{impact} of detection as high as possible.

\subsection{Probability of Being Detected}
We briefly reason about the probability of an attacker being detected, $p_a$,
assuming a simple CTR-based design and our threat model. 

The probability that a client (Tor Browser) will send a SFO to a CTR, $p_c$, is
\emph{an upper bound} on the attacker being detected: $p_a \leq p_c$. If a
client does not attempt to get a SFO audited, the attacker won. The probability
of detection can further be reduced by:

\begin{itemize}
    \item The attacker DoS-ing auditors, the Tor-network, and/or CTRs, lowering
    the probability $p_d$ that a SFO will be be successfully delivered to an
    auditor. A special case of this is a SFO flushing attack, where the attacker
    floods a CTR with valid SFOs exhausting its (presumably volatile) storage of
    SFOs to audit, causing it to discard SFOs.
    \item The probability $p_o$ that the CTR with the SFO discards its cache due
    to going offline or restarting by chance.
    \item The probability $p_z$ that the attacker manages to identify and
    isoloate the CTR with the SFO before it sends the SFO to an auditor. This
    can be done by a tagging attack to identify the SFO and then, e.g., using a
    zero-day to compromise/crash the CTR or disconnecting it from the Internet
    by disabling its uplink.
\end{itemize}

Therefore, at least: 

\begin{align}
    p_a \leq p_c p_d(1-p_o)(1-p_z)
\end{align}

For a short time-frame, $p_o$ might be negligible (or we only selected relays
with the stable flag as CTRs). Estimating $p_z$ depends in part on the timeout
parameter used by querying CTRs and CTR diversity in the network.

\subsection{Impact of Being Detected}
if we don't wait at least MMD before auditing, the \emph{impact} of the
inconsistent SFO reaching an auditor can be reduced by the attacker by including
the certificate in the log, damaging the reputation of the issuing CA (or maybe
completely killing it), instead of the involved CT-logs.

\subsection{Comments on Above}
The impact-side is our strength: it's devestatingly expensive for an attacker,
especially if we want to make drag-net surveillance harder, not very few
targetted attacks.

The probability of being detected is more complex. We have full control of
$p_c$, can make $p_o$ negligible, but at the cost of some lack of CTR diversity,
negatively impacting $p_z$. Likely, for $p_z$, the timeout is key. If $p_c$ is
small, then having an aggressive timeout for $p_z$ should be fine from a
privacy-perspective. If $p_c$ is large, then the negative impact of an
aggressive timeout becomes worse. 

The special-case of $p_d$ with SFO flushing is important, since it's an attack
that has no auxillary impact on the network beyond the CTor functionality. The
second such attack, also part of $p_d$, is DoS of the auditors. We can make it
slightly harder to DoS auditors by introducing more of them. Likely, we don't
want auditors running as onions due to the inherent DoS issues? Or doesn't
matter because this vector is unfixable really? Maybe the collateral damage here
as an onion would be good? More likely we want many potential auditors, also
used outside of Tor?