%
% Sketch for introduction: Tor -> anonymity -> deanonymization -> client
% exploitation -> popping endpoints for delivery or mitm -> our focus mitm ->
% TLS and CT -> CT+Tor = CTor -> contributions + outline. 
%
% Make sure contribution also highlights added value of widely audited CT log
% view to larger Internet ecosystem.
%
\section{Introduction} \label{sec:introduction}
%
% 1
% - Web turned encrypted last couple of years
% - HTTPS, TLS, CAs
% - Current mindset -> encrypt as default
%
Metrics reported by Google and Mozilla reveal that encryption on the web
skyrocketed the past couple of years: at least 85\% of all web pages load using
secure HTTP~\cite{google-metrics,mozilla-metrics}.  Secure HTTP, also known as
HTTPS, refers to the use of Transport Layer Security (TLS).  For example, as
part of the TLS handshake web browsers require that a valid certificate chain
is presented that binds the web server's identity to a public key which is used
to bootstrap an encrypted channel.  A certificate chain is considered valid if
it is digitally signed by a trusted Certificate Authority (CA).  It used to be
costly to obtain a certificate chain in the past, but Let's Encrypt
significantly reduced the barrier towards HTTPS everywhere by providing free
and automated certificate issuance since its launch as a non-profit CA in
2015~\cite{le}.
The current mindset of encryption as default is evident by other ongoing
initiatives too, such as
	modern web browsers moving from positive to negative security
		indicators~\cite{chrome-ui,firefox-ui} and
	the rapid deployment of TLS~1.3~\cite{rapid-tls13}.

%
% 2
% - Anonymity
% - Tor Browser to browse the normal web
% - Tor network
% - Threat of deanonymization + some literature on this
%
While HTTPS everywhere raises the bar to inspect and tamper with generated
network traffic, it does not obfuscate who is talking to whom.  A user that
needs anonymity while browsing the web can fortunately use Tor Browser, which
is designed to make surveillance and censorship difficult by encrypting the
user's HTTP(S) traffic three times as it traverse independent relays in the Tor
network~\cite{tor}.  Tor cannot provide anonymity against...
TODO: elaborate on Tor, threat model, how attacker may deanonymize.

%
% 3
% - We consider a different angle towards deanomization
% a) MitM, simply check who logs in to what
% b) Browser exploits, but it is hard to target an anonymous user.  So, such
% an exploit would have to be served over a MitM:ed connection.
% - Thus, the focus of our work is MitM against Tor Browser users
%
We consider another angle that could deanonymize Tor Browser users.
An attacker may man-in-the-middle a user's HTTPS connection from somewhere
between an exit relay and a targeted web server.  If the user logs in to any
service, it trivially links the user's identity and the visited website.
Secondly, the attacker may serve an unknown Tor Browser exploit that loads over
a hijacked HTTPS connection, deanonymizing the user's identity when run.
The premise of both attack scenarios is to man-in-the-middle an HTTPS
connection, which requires a TLS certificate that Tor Browser accepts.  As
such, we focus on certificate verification in Tor Browser.

%
% 4
% - Independent of the threat of deanonymizaion, ref stuff
% - CT one of the main approaches that is deployed
% - Brief intro, status quo w/ Chrome and Safari
Preventing certificate mis-issuance, or, alternatively, detecting and remedying
it, is of independent interest regardless of its impact on anonymity in Tor
Browser.  A wide variety of improvements and replacements to the CA ecosystem
were therefore proposed already, but few are deployed at scale in practise.
TODO: overview, CT.

\subsection{Overview of CTor}

\subsection{Contribution and Structure}
