\section{Introduction} \label{sec:introduction}
Metrics reported by Google and Mozilla reveal that encryption on the web
skyrocketed the past couple of years: at least 85\% of all web pages load using
Transport Layer Security (TLS) as part of
HTTPS~\cite{google-metrics,mozilla-metrics}. A HTTPS connection is initiated by
a TLS handshake where the client web browser requires that the web server
presents a valid certificate to authenticate the identity of the server (e.g.,
to make sure that the client who wants to visit \url{mozilla.org} really is
connecting to Mozilla, and not, say, Google). A certificate is considered valid
if it is digitally signed by a Certificate Authority (CA) that the browser
trusts. Each CA is trusted to certify (sign) that specific cryptographic
key-material as part of a certificate belongs to a particular domain name.

It used to be costly to obtain a widely trusted certificate in the past, but
Let's Encrypt significantly reduced the barrier towards HTTPS everywhere by
providing free and automated certificate issuance since its launch as a
non-profit CA in 2015~\cite{le}. The current mindset of encryption as default is
evident by other ongoing initiatives too, such as modern web browsers moving
from positive to negative security indicators~\cite{chrome-ui,firefox-ui} and
the rapid deployment of TLS version 1.3~\cite{rapid-tls13}.

The CA trust model suffers from \emph{weakest-link} security: web browsers trust
hundreds of CAs, and it is enough to compromise a single CA to get a certificate
mis-issued in the name of a target domain~\cite{ca-ecosystem,https-sok}.
Motivated by this issue manifested by prominent CA compromises---such as the
issuance of a fraudulent certificate for \url{google.com} by DigiNotar in
2011\footnote{\url{https://web.archive.org/web/20200521135444/https://arstechnica.com/information-technology/2011/08/earlier-this-year-an-iranian/}}---several
major browser vendors have mandated that certificates issued by CAs must be
included into trusted Certificate Transparency (CT) logs for browsers to trust
them~\cite{ct/a,ct,ct/bis}. The idea behind CT is that by making all issued
certificates transparent miss-issued certificates can be detected \emph{after}
issuance and appropriate actions taken to keep the wider web safe (e.g.,
revocation of certificates, suspected compromises investigated, or trust in
misbehaving CAs removed from browsers). Browsers that wish to benefit from CT
augment the validation done by the browser during the TLS handshake to also
require cryptographic proof from the server that the presented certificate has
been included into CT logs trusted by the browser. Notable browsers with
mandatory CT support by default are Google Chrome and Apple's Safari
\cite{chrome-policy,safari-policy}, while Mozilla's Firefox lacks support by
default.

Firefox is the basis for Tor Browser (TB) from the Tor Project~\cite{tor}. TB
enables anyone to browse the web anonymously by relaying traffic between browser
and web server through the Tor network, consisting of thousands of voluntary-run
relays across the globe. There are many techniques an attacker can use to
attempt to deanonymize a TB user, such as end-to-end correlation attacks or
other forms of traffic analysis~\cite{tor, FIXME}. 

One common deanonymization technique---known to be used in practice by, e.g., law
enforcement---is to compromise TB instead of directly circumventing the anonymity
provided by the Tor network~\cite{FIXME}. Modern web browsers like TB (Firefox)
are one of the most complex types of software in wide use today. This complexity
and wide use leads to security vulnerabilities and incentives for exploitation.
For example, the exploit acquisition platform Zerodium offers up to \$100,000
for a zero-day exploit against Firefox that leads to remote code execution and
local privilege
escalation\footnote{\url{https://web.archive.org/web/20200521151311/https://zerodium.com/program.html}}
(i.e., full control of the browser for the attacker).

An attacker that wishes to use such an exploit to compromise and then ultimately
deanonymize a TB user has to deliver the exploit to TB. Since the web is mostly
encrypted today, this primarily has to happen over a HTTPS connection where the
attacker controls the content returned by the web server. While there are a
number of possible ways for an attacker to accomplish this (e.g., by
compromising a web server that the target TB user connects to), one option is to
\emph{impersonate} a web server by acquiring a fraudulent certificate. Due to the
Tor network being run by volunteers, getting into a position for performing such
an attack is relatively straightforward (the attacker can volunteer to run
malicious relays~\cite{spoiled-onions}). The same is true for an attacker that
wishes to \emph{man-in-the-middle} connections made by TB users. In such a case,
a TB exploit may not even be needed to deanonymize the user; e.g., if the user
logs in an account at the service associated with its identity that the attacker
can now observe.

\subsection{Overview of CTor}
We propose a design that enforces CT incrementally in Tor Browser: CTor.  The
three increments are as follows:
\begin{enumerate}
	\item \textbf{Basic CT policy:}
		Tor Browser enforces a basic CT policy that requires promises of public
		CT logging.
	\item \textbf{Resilience towards misbehaving CT logs:}
		Tor Browser is part of a Tor-wide auditing process that adds certificate
		chains in independent CT logs.
	\item \textbf{Detection of misbehaving CT logs:}
		Tor Browser is part of a Tor-wide auditing process that exposes evidence
		of CT logs that misbehaved.
\end{enumerate}

Following Google's suit to minimize CT breakage~\cite{does-ct-break-the-web},
the first step is a leap in the right direction which detects mis-issued TLS
certificates that are presented to Tor Browser users \emph{if all CT logs are
honest}.  This is the same flawed assumption that the CA ecosystem relies on,
and we already observed several instances of CT logs that misbehaved~%
	\cite{izenpe-disqualified,venafi-disqualified,gdca1-omission}.
Most recently, a compromised CT log signing key was reported for the first
time~\cite{digicert-log-compromised}.

To add \emph{resilience} against CT logs that misbehave, we propose a Tor-wide
auditing process.  It is composed of a submission phase, a storage phase, and
an auditing phase:
	Tor Browser submits a presented certificate chain probabilistically to
		a random Tor relay,
	which in turn stores it temporarily before adding it to an independent CT
		log that have yet to promise any public logging.
As such, a certificate chain is likely to make it into the public domain
\emph{if some independent CT logs are benign}.  Section~\ref{sec:todo} motivates
why Tor's threat model makes it difficult to perform CT auditing straight-up
from Tor Browser.  Therefore, Tor relays are part of the process.

The basic design can be extended at the cost of different deployment trade-offs
and trust assumptions, such that it is possible to \emph{detect CT logs that
misbehaved} by omitting a certificate chain from the public.  We propose two
orthogonal extensions.  First, a proposal that shifts trust from CT logs to CT
auditors that are easier to diversify due to different operational
requirements.  Second, a trivial extension of the base design that requires
small, but significant, changes to the CT landscape in the form of an
additional CT log endpoint.

\subsection{Contribution and Structure}
