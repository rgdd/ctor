\section{Introduction} \label{sec:introduction}
Metrics reported by Google and Mozilla reveal that encryption on the web
skyrocketed the past couple of years: at least 85\% of all web pages load using
Transport Layer Security (TLS) as part of
HTTPS~\cite{google-metrics,mozilla-metrics}. An HTTPS connection is initiated by
a TLS handshake where the client web browser requires that the web server
presents a valid certificate to authenticate the identity of the server (e.g.,
to make sure that the client who wants to visit \url{mozilla.org} really is
connecting to Mozilla, and not, say, Google). A certificate is considered valid
if it is digitally signed by a Certificate Authority (CA) that the browser
trusts. Each CA is trusted to certify (sign) that specific cryptographic
key-material as part of a certificate belongs to a particular domain name.

The CA trust model suffers from \emph{weakest-link} security: web browsers trust
hundreds of CAs, and it is enough to compromise a single CA to get a certificate
mis-issued in the name of a target domain~\cite{ca-ecosystem,https-sok}.
Motivated in part by prominent CA compromises---such as the issuance of a
fraudulent certificate for \url{*.google.com} and \url{*.torproject.org} by
DigiNotar in 2011~\cite{diginotar}---several major browser vendors have mandated
that certificates issued by CAs must be included into trusted Certificate
Transparency (CT) logs for browsers to trust them~\cite{ct/a,ct,ct/bis}. The
idea behind CT is that, by making all issued certificates transparent,
mis-issued certificates can be detected \emph{after} issuance and appropriate
actions taken to keep the wider web safe (e.g., revocation of certificates,
suspected compromises investigated, or trust in misbehaving CAs removed from
browsers). Browsers that wish to benefit from CT augment the TLS handshake to
also require cryptographic proof from the server that the presented certificate
will be included into CT logs trusted by the browser. Notable browsers with
mandatory CT support are Google Chrome and Apple's Safari
\cite{chrome-policy,safari-policy}. Unfortunately, Mozilla's Firefox lacks
support but has a partial implementation~\cite{ffct}.

Beyond encryption on the web, anonymous access to the web has also matured. In
particular, Tor with its Tor Browser (TB) today has millions of daily users
\cite{tor,mani}, and effort is ongoing into maturing the technology for wider
use~\cite{fftor}. The basis of TB is Firefox, where TB then enables anyone to
browse the web anonymously by relaying traffic between the user's browser and
the web server in question through the Tor network, consisting of thousands of
voluntary-run relays across the globe~\cite{relay-by-flag}. 

Just like attackers may be interested in breaking HTTPS, attackers may also be
interested in breaking the anonymity provided by Tor to deanonymize users. One
common deanonymization technique, known to be used in practice is to compromise
TB instead of directly circumventing the anonymity provided by
Tor~\cite{selfrando,lepop1,lepop2,zerotor}. Modern web browsers like TB (i.e.,
Firefox) are one of the most complex types of software in wide use today. This
complexity and wide use leads to security vulnerabilities and incentives for
exploitation. For example, the exploit acquisition platform Zerodium offers up
to \$100,000 for a zero-day exploit against Firefox that leads to remote code
execution and local privilege escalation~\cite{zeromain} (i.e., full control of
the browser).

An attacker that wishes to use such an exploit to compromise and then ultimately
deanonymize a TB user has to deliver the exploit to TB\@. Since the web is
mostly encrypted today, this primarily has to happen over an HTTPS connection,
where the attacker controls the content returned by the web server. While there
are a number of possible ways for an attacker to accomplish this (e.g., by
compromising a web server that the target TB user connects to), one option is to
\emph{impersonate} a web server by acquiring a fraudulent certificate. Due to
the Tor network being run by volunteers, getting into a position for performing
such an attack is relatively straightforward (the attacker can volunteer to run
malicious exit relays~\cite{spoiled-onions}). The same is true for an attacker that
wishes to \emph{man-in-the-middle} connections made by TB users. In such a case,
a TB exploit may not even be needed to deanonymize the user; e.g., if the user
logs into an account at the service associated with its identity and that the
attacker can now observe.

\subsection{Introducing CTor}
In this paper, we propose an incrementally deployable and privacy-preserving
design which we refer to as CTor. CTor brings CT to Tor by modifying how both TB
and relays in the Tor network work. The goal is to make impersonation and
man-in-the-middle attacks on HTTPS connections \emph{detectable after the fact}
when they are carried out by a powerful attacker that:
\begin{enumerate}
	\item can acquire a forged certificate from a trusted CA,
	\item with the necessary forged cryptographic proofs from CT logs so that TB
	accepts the certificate as valid (with no intent of making the forged
	certificate publicly available in a CT log), and
	\item has the capability to fully gain control of TB shortly after the
	establishment of the HTTPS connection.
\end{enumerate}

Our base CTor design uses the CT log ecosystem against the attacker to ensure a
non-zero (tweakable) probability of public disclosure of the forged certificate
\emph{per use} against a TB user. This is done by probabilistically adding the
certificate to \emph{other CT logs} than those directly inferable as under
attacker control (due to the accompanying proofs). TB uses modified relays to
cache and interact with CT logs for the sake of privacy, ensuring that CT log
operators (e.g., Google and Cloudflare) do not get a real-time feed of all
browsing activities from TB (in addition to what their privileged positions on
the Internet already provides them~\cite{1mtrack,TorDNS}).

In addition to the base CTor design we present two extensions. Both extensions
have in common that their goal is to make transparent the forged cryptographic
proofs from attacker-controlled CT logs used to get TB to accept malicious HTTPS
connections. Such a proof would likely immediately disqualify each offending CT
log and thus come at great cost to the attacker in terms of lost capability.
Even a small probability of detection may entail unacceptable risk at such a
high cost. Note that we have already observed several instances of CT logs that,
intentionally or not,
misbehaved~\cite{izenpe-disqualified,venafi-disqualified,gdca1-omission}. Most
recently, a compromised CT log signing key was reported for the first
time~\cite{digicert-log-compromised}. 

The key change in the first extension is that relays now challenge CT logs to
cryptographically prove the correctness of their prior proofs presented to TB
instead of simply sharing the certificates (the subject of the proofs) with
other CT logs. In case an offending CT log fails to provide such a proof in a
timely manner, the querying relay ensures that the proof arrives at a trusted
auditor. While this adds both more complexity and operational challenges to Tor
(mainly auditors and storage at relays), it removes all trust in the CT
ecosystem. Another benefit is that, as a consequence of CTor's design, Tor would
provide a probabilistically-verified view of the CT log ecosystem from Tor's
consensus.

The second extension involves only a minor tweak to the base CTor design.
Unfortunately, this tweak is primarily not to Tor but to the API of CT logs.
While similar changes have been discussed earlier in the context of gossiping in
CT~\cite{minimal-gossip,nordberg}, it is a significant change. The small tweak
is that CT logs provide an API for receiving cryptographic proofs made by other
CT logs wrt. certificate inclusion. This would then mean that relays would add
to other CT logs proofs observed by TB clients in addition to the certificate.
This would also significantly reduce the trust assumptions the wider web has to
place on CT log operators, from close to full trust as-is today to only trusting
that some operators are honest.

\subsection{Contribution and Structure}
Section~\ref{sec:background} provides necessary background on CT and Tor. Our
threat model is described in detail in Section~\ref{sec:adversary}, taking into
account the threat models of Tor and CT\@.

Section~\ref{sec:base} presents the base design of CTor that enables \emph{TB to
probabilistically provide evidence to CT logs that it has been presented with a
fraudulent certificate while preserving privacy}. This greatly impacts
risk-averse attackers, since part of its fraudulent behavior has been made
transparent through the CT ecosystem: the issuing CA may get revoked from the
trust store of browsers, the domain name in question (as part of the
certificate) is made public, and awareness of the event may draw unwanted
attention. Our security analysis, in Section~\ref{sec:analysis}, shows that one
of the best bets for an attacker would be to attack the entire Tor network or
all CT logs: an act that would likely draw unwanted attention.

Two extensions to the base CTor design are presented in
Sections~\ref{sec:auditor} and \ref{sec:log}. Both extensions aim to make
transparent the accompanying fraudulent proofs issued by CT logs to convince TB
to accept a fraudulent certificate. The deployment of either extension
\emph{would greatly contribute to the open question of how to reduce trust in CT
log operators} currently caused by the lack of an appropriate gossiping
mechanism~\cite{minimal-gossip,nordberg}. In particular, the auditor-based
extension (Section~\ref{sec:auditor}) would result in a
\emph{probabilistically-verified view of the entire CT log ecosystem} available
from Tor's consensus. This view could be used as the basis for trust by other
browsers, \emph{greatly improving the security posture of the entire web}.

By splitting our designs into a base design and two possible extensions we argue 
that CTor is \emph{incrementally deployable}: start with completing Firefox
support for CT, add the base CTor design, then pick one of the two extensions
based on how the landscape evolves.

Section~\ref{sec:performance} estimates performance aspects of our designs,
showing that the estimated (based on Mani et al.'s measurements~\cite{mani})
circuit-, bandwidth- and memory- \emph{overheads are modest even without
caching}. Privacy aspects of our design choices are covered in
Section~\ref{sec:privacy}, with a focus on the essential role of the distributed
nature of the Tor network in both preserving privacy of Tor users as well as the
overall security of our proposed designs. In gist, \emph{a similar approach
would be privacy-invasive without Tor}, e.g., if adopted by Google Chrome.
Section~\ref{sec:related} presents related work and Section~\ref{sec:conclusion}
conclusions.
