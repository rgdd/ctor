%
% Sketch for introduction: Tor -> anonymity -> deanonymization -> client
% exploitation -> popping endpoints for delivery or mitm -> our focus mitm ->
% TLS and CT -> CT+Tor = CTor -> contributions + outline. 
%
% Make sure contribution also highlights added value of widely audited CT log
% view to larger Internet ecosystem.
%
\section{Introduction} \label{sec:introduction}
%
% 1
% - Web turned encrypted last couple of years
% - HTTPS, TLS, CAs
% - Current mindset -> encrypt as default
%
Metrics reported by Google and Mozilla reveal that encryption on the web
skyrocketed the past couple of years: at least 85\% of all web pages load using
secure HTTP~\cite{google-metrics,mozilla-metrics}.  Secure HTTP, also known as
HTTPS, refers to the use of Transport Layer Security (TLS).  For example, as
part of the TLS handshake web browsers require that a valid certificate chain
is presented that binds the web server's identity to a public key which is used
to bootstrap an encrypted channel.  A certificate chain is considered valid if
it is digitally signed by a trusted Certificate Authority (CA).  It used to be
costly to obtain a certificate chain in the past, but Let's Encrypt
significantly reduced the barrier towards HTTPS everywhere by providing free
and automated certificate issuance since its launch as a non-profit CA in
2015~\cite{le}.
The current mindset of encryption as default is evident by other ongoing
initiatives too, such as
	modern web browsers moving from positive to negative security
		indicators~\cite{chrome-ui,firefox-ui} and
	the rapid deployment of TLS~1.3~\cite{rapid-tls13}.

%
% 2
% - Anonymity
% - Tor Browser to browse the normal web
% - Tor network
% - Threat of deanonymization + some literature on this
%
While HTTPS everywhere raises the bar to inspect and tamper with generated
network traffic, it does not obfuscate who is talking to whom.  A user that
needs anonymity while browsing the web can fortunately use Tor Browser, which
is designed to make surveillance and censorship difficult by encrypting the
user's HTTP(S) traffic three times as it traverse independent relays in the Tor
network~\cite{tor}.  Tor cannot provide anonymity against...
TODO: elaborate on Tor, threat model, how attacker may deanonymize.

%
% 3
% - We consider a different angle towards deanomization
% a) MitM, simply check who logs in to what
% b) Browser exploits, but it is hard to target an anonymous user.  So, such
% an exploit would have to be served over a MitM:ed connection.
% - Thus, the focus of our work is MitM against Tor Browser users
%
% Note: "easy" to MitM in Tor, i.e., run an exit relay
%
We consider another angle that could deanonymize Tor Browser users.
An attacker may man-in-the-middle a user's HTTPS connection from somewhere
between an exit relay and a targeted web server.  If the user logs in to any
service, it trivially links the user's identity and the visited website.
Secondly, the attacker may serve an unknown Tor Browser exploit that loads over
a hijacked HTTPS connection, deanonymizing the user's identity when run.
The premise of both attack scenarios is to man-in-the-middle an HTTPS
connection, which requires a TLS certificate that Tor Browser accepts.  As
such, we focus on certificate verification in Tor Browser.  Note that it is
easy to be in a physical position to man-in-the-middle Tor traffic as it enters
the regular Internet:
	exit relays intercepted HTTP(S) traffic in the past~\cite{spoiled-onions}.

%
% 4
% - Independent of the threat of deanonymizaion, ref stuff
% - CT one of the main approaches that is deployed
% - Brief intro, status quo w/ Chrome and Safari
Preventing certificate mis-issuance, or, alternatively, detecting and remedying
it, is of independent interest regardless of its impact on anonymity in Tor
Browser.  This interest stem from long known limitations in the CA ecosystem,
which among other issues suffer from weakest-link security:
	web browsers trust hundreds of CAs, and thus it is suffices to compromise
	a single CA to get a certificate mis-issued in the name of a target
	domain~\cite{ca-ecosystem,https-sok}.
Certificate Transparency (CT) is one of few proposals that help remedy the
threat of certificate mis-issuance in practise~\cite{ct/a,ct,ct/bis}.  The idea
is to disclose certificate chains to public scrutiny in cryptographically
verifiable log servers, which allow anyone to inspect and detect the set of
(mis)issued certificates without trusting the resulting CT ecosystem blindly.
Google's Chrome and Apple's Safari enforce CT partially by requiring
digitally signed \emph{promises} of public certificate logging in CT
logs~\cite{chrome-policy,safari-policy}.  Unless it is verified that these
promises of public logging are honored, CT logs must be trusted.

\subsection{Overview of CTor}
We propose a design that enforces CT incrementally in Tor Browser.  The three
increments are as follows:
\begin{enumerate}
	\item \textbf{Basic CT policy:}
		Tor Browser enforces a basic CT policy that requires promises of public
		CT logging.
	\item \textbf{Resilience towards misbehaving CT logs:}
		Tor Browser is part of a Tor-wide auditing process that adds certificate
		chains in independent CT logs.
	\item \textbf{Detection of misbehaving CT logs:}
		Tor Browser is part of a Tor-wide auditing process that exposes evidence
		of CT logs that misbehaved.
\end{enumerate}

Following Google's suit to minimize CT breakage~\cite{does-ct-break-the-web},
the first step is a leap in the right direction which detects mis-issued TLS
certificates that are presented to Tor Browser users \emph{if all CT logs are
honest}.  This is the same flawed assumption that the CA ecosystem relies on,
and we already observed several instances of CT logs that misbehaved~%
	\cite{izenpe-disqualified,venafi-disqualified}.
Most recently, a compromised CT log signing key was reported for the first
time~\cite{digicert-log-compromised}.

To add \emph{resilience} against CT logs that misbehave, we propose a Tor-wide
auditing process.  It is composed of a submission phase, a storage phase, and
an auditing phase:
	Tor Browser submits a presented certificate chain probabilistically to
		a random Tor relay,
	which in turn stores it temporarily before adding it to an independent CT
		log that have yet to promise any public logging.
As such, a certificate chain is likely to make it into the public domain
\emph{if some independent CT logs are benign}.  Section~\ref{sec:todo} motivates
why Tor's threat model makes it difficult to perform CT auditing straight-up
from Tor Browser.  Therefore, Tor relays are part of the process.

The basic design can be extended at the cost of different deployment trade-offs
and trust assumptions, such that it is possible to \emph{detect CT logs that
misbehaved} by omitting a certificate chain from the public.  We propose two
orthogonal extensions.  First, a proposal that shifts trust from CT logs to CT
auditors that are easier to diversify due to different operational
requirements.  Second, a trivial extension of the base design that requires
small, but significant, changes to the CT landscape in the form of an
additional CT log endpoint.

\subsection{Contribution and Structure}
