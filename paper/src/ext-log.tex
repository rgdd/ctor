\section{Log Extension} \label{sec:log}
If we allow small---yet significant---changes to the CT landscape, it is
possible to avoid inclusion verification and the involved complexities as
described in Section~\ref{sec:auditor}.  The idea is based on returning back to
the premise of some CT logs being honest, and the goal is not only
\emph{resilience towards} but also \emph{detection of} CT logs that misbehaved.
The extended design is nearly identical to the base described in
Section~\ref{sec:base}.
	Tor Browser submits presented SFOs probabilistically to randomly selected
		CTRs in phase 1,
	which store the submitted SFOs in phase 2 before any auditing takes place in
		phase 3.
Here, \emph{auditing} refers to adding a \emph{full SFO} to an independent CT
log; not just the underlying certificate chain.  This is the main difference in
our extension.  By also adding the SCTs, the associated CT logs can be held
accountable.

The prerequisite for such an extension to work is that CT logs support an
additional API endpoint:
	\texttt{add-sfo}.
First we describe how this endpoint could be operated so that \emph{the only
change} is that certificate chains and SCTs are added.  Next, we explain how
some complexity could be moved into Tor but at the cost of bringing back the
threat of flooding (see Section~\ref{sec:auditor:analysis:phase2}).

\textbf{Approach~1.}
Recall from Section~\ref{sec:auditor:design:phase2} that there must not be any
early signals that allow misbehaving CT logs to reactively merge certificate
chains before any MMD promise is violated.  To ensure that this is the case, the
\texttt{add-sfo} endpoint could entail a promise that the added SFO will not be
merged until all MMDs elapsed and the logs (should have) produced STHs that
captured the omission.  For example, the SCT that is returned by the
\texttt{add-sfo} endpoint could have a timestamp that is future-dated by at
least two MMDs, and then it is not considered for merging until that timestamp
is in the present.  The appeal of \emph{delayed merges} is that SFOs need not
be stored any longer than what is necessary for privacy, completely avoiding
the threat of network-wide flushes within our threat model.  Viz., the analysis
in Section~\ref{sec:base} applies without modifications, expect that misbehaving
CAs \emph{and} CT logs are exposed.

\textbf{Approach~2.}  The other option is to make minimal changes to the CT
landscape by operating the \texttt{add-sfo} endpoint without any other
expectation than that it must allow SCTs in addition to a certificate chain.
To avoid early signals, CTRs should employ the same delay strategy as suggested
for CT logs above:
	wait at least two MMDs before adding a (newly issued) SFO.
This is essentially the same as an attacker that maximized the storage phase
by waiting until the last second to produce an STH as discussed in
Section~\ref{sec:auditor:analysis:phase2}, expect that such behavior is assumed
rather than confirmed.  Clearly this brings flooding back to the table, and CTRs
need the extra-info metrics from Section~\ref{sec:auditor:extra-info} to
allow detection.
