\section{Incremental Deployment} \label{sec:incremental}

The design outlined in Section 4 covers an end-to-end design that - if an SFO
is audited - ensures it was included in the log within the MMD and that it will be
resolved to a trusted STH or that it will be provided to a trusted auditor for
human followup and publication. It requires changes to Tor Browser, the Tor
consensus, Tor relays, and the deployment of the new trusted auditor
infrastructure.

It is possible to provide a smaller measure of protection with fewer infrastructure
changes and less complexity.

\subsection{SubDesign 1: Detecting Malicious Certificate Authorities}

After a CTR receives a SFO, it can choose to log the certificate chain to alternate
logs who have not issued SCTs for it (based on the SCTs provided in the SFO.) If
we assume that at least one log it submits the data to is honest, the certificate will
be published and allow detection of the malicious or compromised CA.

A CTR can reuse a single tor circuit for multiple certificate submissions; however
to address the threat of a log performing a Denial of Service attack on the CTR
after learning it has evidence of misbehavior, the CTR must hold circuits to all of
the logs and submit a certificate chain to all of the logs simultaneously. This
ensures that a malicious log will not be able to take the CTR offline before the
evidence can be submitted to an honest log. Alternately, the CTR can submit a single
certificate chain per-circuit, and not perform coordinated submissions.

Submitting the certificate to the logs, one of which is presumed to be honest, and
omitting all subsequent auditing steps will not disclose what log(s) may have issued
SCTs for the certificate in question and thus be malicious; but will allow detection
of the malicious or compromised CA.

This design omits the trusted auditor infrastructure and the necessity for watchdog
CTRs.

\subsection{SubDesign 2: Detecting Malicious Logs}

In a similar vein to SubDesign 1, after a CTR receives a SFO, it could log both the
certificate chain and the SCTs to alternate logs.  By publishing the SCTs in addition
to the certificate, we can see what CA was malicious or compromised, and what logs
misbehaved by not including the SCT within the MMD.

This design also relies on at least one honest log, and the same techniques for avoiding
Denial of Service attacks in SubDesign 1 apply. While this proposal avoids the need for
trusted auditors and watchdogs, submitting SCTs from other logs is not a mechanism
that Certificate Transparency currently supports. Therefore this proposal would require
a protocol improvement and new development on behalf of the logs.
