\section{Design} \label{sec:base}

A complete design that detects misbehavior by CAs and CT Logs, performed by an adversary with the ability to drop or delay network traffic from CT Logs and a subset of Tor relays, control a subset of Tor relays, serve an exploit to targeted users' browsers and perform a Denial of Service attack on a subset of Tor relays necessitates a considerable degree of complexity. A less-complicatd subset of such a system can be deployed incrementally to provide protection against an attacker with a subset of such capabilities.

We build the complete system in phases, demonstrating the need for the complexity, and then present incremental deployments in Section 5. The simplest proposal if one that only validates SCT, which is how CT in Google Chrome and other browsers is currently deployed. This design does not stand up against a malicious CA and logs working in concert, so it is necessary to perform an auditing function.

\subsection{Phase~1---Tor Browser} \label{sec:base:phase1}

The least complicated auditing design would be one where Tor Browser receives a TLS Certificate and accompanying SCTs (we will refer to this bundle as an SCT Feedback Object or SFO) and talk to the corresponding logs, over Tor, requesting an inclusion proof for the SCT. In an ordinary browser, this would be an unacceptable leak of browsing behavior tied to an IP address to the log; performing this request over Tor leaks real-time browsing behavior but not the user's IP address.

The immediate problem with this design is that a primary requirement of Tor Browser is to persist no data about browsing behavior after the application exits. If we assume a user's browser is not left open for long periods of time, the inclusion proof request can be easily circumvented by the attacker by supplying a fresh SCT whose MMD has not completed - thus no inclusion proof can be provided. A second problem is that the STH returned from any inclusion proof exists in a trust vacuum - there is no way to know that it is consistent with other STHs and not part of a split view.

The evolved proposal adds two components: a list of STHs that the browser will receive over a trusted channel and the participation of a third party with the ability to persist data and perform auditing actions at a later time.

A single third party used by all Tor Browser users would receive a considerable aggregation of browsing behavior and would need to scale in-line with the entire Tor network. A small number of auditors presents privacy and single-point-of-failure concerns. A large number would be ideal; but presents difficulties in curation and independent management and still requires scaling independent of the Tor network. These concerns do not entirely preclude the design; but they can be easily avoided by reusing relays in the Tor Network as our trusted third parties: we call the relays so designated Certificate Transparency Relays (or CTRs).

Now; when the browser is completing the TLS handshake they simultaneously either pass the SFO to a CTR (if the MMD of the SCT has not elapsed) or query the log themselves asking for an inclusion proof to a trusted STH.  However, if we presume the attacker can serve an exploit to the browser, the latter behavior is immediately vulnerable. The log, upon receiving an inclusion proof fetch for a SCT that it knows is malicious, will delay responding. The TLS connection in the browser, having succeeded, will progress to the HTTP request and response, at which point the exploit will be served, and the SFO - containing the cryptographic evidence of CA and Log misbehavior - will be deleted by the exploit code. While blocking the TLS connection until the CT Log responds is an option, the failure of OCSP hard-fail indicates this notion is doomed to fail. % TODO: remove two uses of fail in that sentence.

The third evolution of our proposal has the Browser submit the SFO to the CTR immediately upon recipient in all cases. A consequence of this shift is that the trusted STH list no longer needs to be delivered to the browser but rather the CTRs. To mitigate the risk of an exploit finding the CTR and disclosing its identity to the attacker (who could then target it for Denial of Service), we prepare CTR circuits ahead of time and close and discard them as soon as the SFO is sent, allowing the SFO submission to race with the TLS connection completion and HTTP request/response.  An added detail is to block the TLS connection in the situation that an SFO is unusually large, as defined by a parameter \texttt{ct-large-sfo-size}. A large SFO may indicate an attempt to win that race between SFO submission and exploitation, and the parameter can be chosen such that it happens extremely rarely on legitimate connections.

The third evolution of our proposal exhibits all the protections we can reasonably achieve within the browser. We summarize this Phase with the following algorithm that provides more explicit steps and details, and add a parameter \texttt{ct-submit-pr} that indicates a probability a SFO is submitted to a CTR - this provides probabilistic security while providing the ability to adjust submission rates to account for CTR scaling issues. Given an incoming SFO $s$, we follow the below steps: % The bit introducing \texttt{ct-submit-pr} can be better.

\begin{enumerate}
    \item Raise a certificate error and stop if the certificate chain of $s$
        is not rooted in TB's trust store.
    \item Raise a certificate transparency error and stop if the SCTs of $s$
        fail TB's CT policy.
    \item If $\mathsf{len}(s) < \texttt{ct-large-sfo-size}$, accept $s$ and
        conduct the remaining steps in the background while the TLS connection
        and HTTP request/response proceed. If $\mathsf{len}(s) \geq \texttt{ct-large-sfo-size}$ pause the TLS handshake, complete the remaining steps and then
        accept~$s$ as valid and continue the handshake and HTTP request/response.
    \item Flip a biased coin based on \texttt{ct-submit-pr} and stop if the
        outcome indicates no further auditing.
    \item Submit $s$ to a random CTR's SFO-endpoint on a pre-built circuit.
        The circuit used for submission is closed immediately after use without
        waiting for any acknowledgment.
\end{enumerate}

\subsection{Phase 2---Buffering} \label{sec:base:phase2}

Once received, the most straightforward thing for a CTR to do is to contact the
issuing log and request an inclusion proof relative to a trusted STH. (And if the
SCT's MMD has not elapsed, hold the SFO until it has.) However, this proposal has
two flaws, the first of which leads us to the final design of Phase 2.

Immediately contacting the log about a SFO (a) allows the log to predict when exactly
it will recive a request about a SFO and (b) discloses real-time browsing behavior
to the log. The former problem means that an attacker can position resources for
perpetuating an attack ahead-of-time, as well as let it know with certainty whether a
connection was audited (based on \texttt{ct-submit-pr}.) The latter is some amount
of information leakage that can help with real-time traffic analysis. 

Because a CTR must support storing SCTs regardless, we could schedule an event in the
future for when each SFO would be sent. This is bad because.... TODO

% TODO Here

opt to effectively create a timed mix~\cite{trickle02}, though one with a
randomized firing interval. The randomness in whether a client submits an SFO
at all combined with the timed mixing potentially complicates leaking client
behavior to a CT log. But we cannot reliably assume that other honest
clients submit to the CTR between mix firings (submissions to CT
logs). Adding a per-SFO value sampled from \texttt{ct-delay-dist}
effectively adds stop-and-go mixing~\cite{kesdogan:ih1998} to the
privacy protection, but where there is only one mix (CTR) between
sender (client) and receiver (CT log). So there is no point in a
client-specified interval-start-time such that the mix drops messages
arriving before then, and there is no additional risk in having the
interval end time set by the mix rather than the sender. This means
both that some SFOs a client sends to a CTR at roughly the same time
might be in different batches of SFOs sent to a CT log and that SFOs
submitted to that CTR by other honest clients are more likely to be
mixed with these.


In addition to storing SFOs for mixing effects, we also add a layer of caching to
alleviate stored data and unnecessary log connections and data disclosure. So with
regards to some CT circuit, process an incoming SFO $s$ as follows:
\begin{enumerate}
    \item\label{enm:storage:close} Close the current circuit to enforce one-time
        usage.
    \item\label{enm:storage:unrecognized} Discard all SCTs in the SFO for logs the
        CTR is not aware of; if no SCTs remain then discard the entire SFO.
    \item\label{enm:storage:cached}
        Stop if $s$ is cached (see Section~\ref{sec:base:phase3}) or already
        pending to be audited.
    \item\label{enm:storage:fix-log} Sample a CT log $l$ that issued a
        remaining SCT in~$s$.
    \item\label{enm:storage:audit-after} Compute an \texttt{audit\_after}
		time~$t$, see Figure~\ref{fig:audit-after}.
    \item\label{enm:storage:store} Add $(l,t,s)$ to a buffer of pending SCTs to audit.
\end{enumerate}

Recall from Section~\ref{sec:background:ct} that an inclusion proof is fetched
with regards to an STH.  As such, we discard SCTs that cannot be verified due to
lack of a trusted STH.  The sampled CT log $l$ now refers to an entity that issued an
SCT in the submitted SFO, and it will be challenged to prove inclusion in phase~3
sometime after the \texttt{audit\_after} timestamp $t$ elapsed. We can choose one
at SCT (and therefore log) at random from the SFO, because it is sufficient to catch
only one misbehaving log so long as we report the entire SFO, allowing for other
malicious logs to be identified. (And if one log is malicious and others are honest; the
honest logs ahve ensured publication of the malicious certificate.)

The \texttt{audit\_after} timestamp specifies the earliest point in time that a SCT
from a SFO will be audited in phase~3, which adds random noise that obfuscate
real-time browsing patterns in the Tor network.  If memory becomes a scarce
resource, delete pending triplets at random~\cite{nordberg}. Figure~\ref{fig:audit-after}
shows that $t$ takes the log's MMD into account.  This is one of two parts that prevent
\emph{early signals} to the issuing CT logs that an SFO is being audited.  For example,
if an SFO is audited before the MMD elapsed, the issuing CT log could simply merge
the underlying certificate chain to avoid an MMD violation.

% TODO: If we're going to say 'one of two early signals' we need to clearly identify the other one. And it's not clear to me what else specifically is considered such an early signal.

\begin{figure}
	\centering
	\pseudocode[linenumbering, syntaxhighlight=auto]{%
		\textrm{t} \gets \mathsf{now}() +
			\mathsf{MMD} +
			\mathsf{random}(\texttt{ct-delay-dist}) \\
		\pcif \textrm{SCT.timestamp} + \textrm{MMD} <
				\mathsf{now}():\\
			\pcind\textrm{t} \gets \mathsf{now}() +
				\mathsf{random}(\texttt{ct-delay-dist})
	}
	\caption{%
		Algorithm that computes an \texttt{audit\_after} timestamp $t$.
	}
	\label{fig:audit-after}
\end{figure}

\subsection{Phase 3---Auditing} \label{sec:base:phase3}

As alluded to in Phase 2, there is a second problem why the simple behavior of
`contact the log and request an inclusion proof' is unacceptable. We include
the ability to DoS an individual Tor relay in our threat model - if the log
knows which CTR holds the evidence of its misbehavior, it can take the CTR
offline, wiping the evidence of the log's misbehavior from its memory. 

We can address this concern in a few ways. The simple proposal of contacting
the log over a Tor circuit will not suffice - a log can tag each CTR by
submitting unique SFOs to them all, and recognize the CTR when they are
submitted. Unique Tor circuits per-SFO can work, however the connections
cannot be correlated. Were they correlated, for example by supporting
exponential backoff, a malicious log could fill all CTRs with tags A1, A2, ... An;
B1, B2, ... Bn; etc and only answer non-tagged SFOs and tags from a target
CTR. With high probability all other CTRs will go into backoff, the malicious log
allows the target CTR to empty itself and if the SFO indicated evidence of
misbehavior appears, the target CTR can be taken offline. If not; move to the
next CTR. 

While there are ways to detect this attack after-the-fact, and may be ways to mitigate
it, a more robust design would tolerate the disclosure of a CTRs identity to the log
without allowing a Denial-of-Service attack to erase evidence of misbehavior.  The
simple approach is to write the data to disk prior to contacting the log; however, Tor relays
are explicitly designed not to write data about user behavior to disk unless
debug-level logging is enabled. Relay operators have expressed an explicit desire
to never have any user data persisted to disk, as it changes the risk profile of
their servers with regards to search, seizure, and forensic analysis.

The evolved design is to have the CTR work with a partner CTR - we call it
a watchdog - who they choose at random and contact over a Tor circuit. Prior to talking
to a log, the CTR provides the watchdog with the SFO it is about to submit. After
an appropriate response from the log, the CTR tells the watchdog that SFO has been
adequately addressed.

Each CTR maintains a single shared circuit that is used to interact with all CT
logs known to the CTR. For \emph{each} such CT log $l$, the
CTR runs the following steps indefinitely:
\begin{enumerate}
    \item\label{enm:auditing:backoff} Sample a delay $d \gets
        \mathsf{random}(\texttt{ct-backoff-dist})$ and wait until $d$ time units
        elapsed.
    \item Connect to a random watchdog CTR.
    \item\label{enm:auditing:loop} For each pending buffer entry $(l',s,t)$,
    where $l' = l$ and $t <= \mathsf{now}()$:
		\begin{enumerate}
			\item\label{enm:ext:auditing:watchdog} Share $s$ with the current
				watchdog.
			\item\label{enm:ext:auditing:challenge} Challenge the log to prove
                                  inclusion to a known STH, waiting for \texttt{ct-log-timeout} before
                                  timing out.
				\begin{itemize}
					\item\label{enm:ext:auditing:challenge:success} On valid
						proof: send an acknowledgment to the watchdog, then
						cache $s$ and discard it.
					\item\label{enm:ext:auditing:challenge:fail} On any other
						outcome: discard $s$, close circuit to the watchdog CTR,
						and go to step~1.
				\end{itemize}
		\end{enumerate}
\end{enumerate}

\subsection{Phase 4---Reporting}

At any given time, a CTR may be requesting inclusion proofs from logs (we will refer to this role as the log-challenger) and may also be a watchdog for one or more other CTRs. A CTR acting as a watchdog will have at most one SFO held temporarily for each log-challenger it is interacting with. If a response from the log-challenger is not received within \texttt{ct-watchdog-timeout}, it becomes the watchdog's responsibility to raise it for human intervention. This end-stage process that begins with the watchdog's receipt of a suspicious SFO and culminates in human review is referred to as auditing, and left mostly-unspecified.

Because human review and publication\footnote{Most likely on the ct-policy group
at \url{https://groups.google.com/a/chromium.org/forum/\#!forum/ct-policy}} is critical, we envision that the watchdog (which is a tor relay that may not be closely monitored) provides the SFO to an independent auditor run by a human closely monitoring its operation. When arriving at the design of the CTR being a role played by a Tor relay, we eschewed separate auditors because of the lack of automatic scaling with the Tor network, the considerable aggregation of browsing behavior across the Tor network, and the difficulties of curation and validation of trustworthy individuals. SFOs submitted to auditors at this stage have been filtered through the CTR layer, resulting in an exponentially smaller load (and data exposure) for auditors and allowing a smaller number of them to operate without needing to scale with the network.

While we consider all auditors trusted - the watchdog needs to take precautions talking to them because the network is not trusted\footnote{While our threat model, and Tor's, precludes a global network adversary, we both include partial network control within the threat model.}. If the watchdog contacted the auditor without a tor circuit, an adversary watching the auditors' network connections could induce a watchdog to contact the auditor, learn the watchdog's identity, pause their network connection, and perform a Denial of Service attack erasing the evidence of misbehavior. To mitigate this, the watchdog can contact an auditor as soon as it receives a SFO it should report; however it must contact the auditor over a Tor circuit. If a successful acknowledgement from the auditor is not received within \texttt{ct-auditor-timeout}, the SFO is buffered and should be re-presented to the same auditor after a random delay.

When an auditor receives a SFO, it should persist it to durable storage until it can be successfully resolved to a trusted STH. Once so persisted, the auditor can begin querying the log itself asking for an inclusion proof. If no inclusion proof can be provided after some threshold of time, or the inclusion proof shows evidence of a MMD violation, the auditor software should raise the details to a human operator for investigation.

Separately, the auditor should be retrieving the current Tor consensus and ensuring that a consistency proof can be provided between STHs from the older consensus and the newer. If consistency can be established, the older STH can be discarded. If consistency cannot be established after some threshold of time, the auditor software should raise the details to a human operator for investigation. It would also be possible for an auditor to monitor a log's uptime and report on excessive downtime.

\subsection{Additional Details} \label{sec:base:consensus}

\subsubsection{CTR Flag} \label{sec:base:consensus:ctr-flag}
Within the Tor consensus, the existing \texttt{known-flags} item determines the
different flags that the consensus might contain.  We add another flag named
\texttt{CTR}, which indicates that a Tor relay should support CT-auditing as
described here. For now, assume that a relay qualifies as a CTR if it is flagged
as \texttt{stable} and not \texttt{exit}, to spare the relatively sparse exit
bandwidth and only use relays that can be expected to stay online.
Section~\ref{sec:privacy} discusses trade-offs in the assignment of the
\texttt{CTR} flag.

\subsubsection{Trusted STHs}
Tor's consensus should capture a fixed view of the CT landscape by publishing
STHs from all recognized logs.  A CT log is recognized if a majority of directory
authorities proposed a \texttt{ct-log-info} item, which contains a log's ID,
public key, base URL, MMD, and most recent STH.  Each directory authority
proposes its own STH, and agrees to use the most recent STH as determined by
timestamp and lexicographical order.  Since CTRs verify inclusion with regards
to SCTs that TB accepts, the CT logs recognized by TB must be
in Tor's consensus.

\subsubsection{Trusted Auditors}
Tor's directory authorities also majority-vote on \texttt{ct-auditor} items,
which pin base URLs and public keys of CT auditors that watchdogs contact in
case that any log misbehavior is suspected.  A watchdog triggers if the time
specified by \texttt{ct-watchdog-timeout} elapses without receiving any
acknowledgment.  The following auditor submission is governed by a
\texttt{ct-auditor-timeout}, which, if triggered, results in a resubmission
later on.

\subsubsection{Other Parameters} \label{sec:base:consensus:params}
Directory authorities influence the way in which TB and CTRs behave by
voting on other necessary parameters. Below, the value of an item is computed as
the median of all votes.
\begin{description}
    \item[ct-submit-pr:] A floating-point in $[0,1]$ that determines Tor
        Browser's submission probability.  For example, $0$ disables submissions
        while $0.10$ means that every 10$^{\mathsf{th}}$ SFO is sent to a
        random CTR on average.
    \item[ct-large-sfo-size:] A natural number that determines how many
        wire-bytes a normal SFO should not exceed.  As outlined in
        Section~\ref{sec:base:phase1}, excessively large SFOs are subject
        to stricter verification criteria.
    \item[ct-log-timeout:] A natural number that determines how long a CTR
        waits before concluding that a CT log is unresponsive, e.g., 10~seconds.
        As outlined in Section~\ref{sec:base:phase3}, timeouts trigger implicit
        resubmissions.
    \item[ct-delay-dist:] A distribution that determines how long a CTR should
        wait at minimum before auditing a submitted SFO\@.  As outlined in
        Section~\ref{sec:base:phase2}, random noise is added, e.g., on the order of minutes to an hour.
    \item[ct-backoff-dist:]
        A distribution that determines how long a CTR should wait between two
        auditing instances, e.g., a few minutes on average.  As outlined in
        Section~\ref{sec:base:phase3}, CTRs audit pending SFOs in batches at
        random time intervals to spread out log overhead.
    \item[ct-watchdog-timeout]
    \item[ct-auditor-timeout]
\end{description}
