\section{Design} \label{sec:base}

A complete design that detects misbehavior by CAs and CT Logs, performed by an adversary with the ability to drop or delay network traffic from CT Logs and a subset of Tor relays, control a subset of Tor relays, serve an exploit to targeted users' browsers and perform a Denial of Service attack on a subset of Tor relays necessitates a considerable degree of complexity. A less-complicatd subset of such a system can be deployed incrementally to provide protection against an attacker with a subset of such capabilities.

\subsection{Phase~1---Tor Browser} \label{sec:base:phase1}

The simplest proposal would be one where Tor Browser receives a TLS Certificate and accompanying SCTs (we will refer to this bundle as an SCT Feedback Object or SFO) and talk to the corresponding logs, over Tor, requesting an inclusion proof on the SCT. In an ordinary browser, this would be an unacceptable leak of browsing behavior tied to an IP address to the log; but performing this request over Tor leaks real-time browsing behavior but not the user's IP address.

The immediate problem with this design is that a primary requirement of Tor Browser is to persist no data about browsing behavior after the application exits. If we assume a user's browser is not left open for long periods of time, the inclusion proof request can be easily circumvented by the attacker by supplying a fresh SCT whose MMD has not completed - thus no inclusion proof can be provided. A second problem is that the SCT returned from any inclusion proof exists in a trust vaccuum - there is no way to know that it is consistent with other STHs and not part of a split view.

The evolved proposal adds two components: a list of STHs that the browser will receive over a trusted channel that it can request inclusion or consistency proofs to and the participation of a third party with the ability to persist data and perform auditing actions at a later time.

A single third party used by all Tor Browser users would receive a considerable aggregation of browsing behavior and present both  single-point-of-failure concerns as well as the deployment of new infrastructure. These concerns do not entirely preclude the design; but they can be easily avoided by reusing relays in the Tor Network as our trusted third parties: we call the relays so designated Certificate Transparency Relays (or CTRs).

Now; when the browser is completing the TLS handshake they simultaneously either pass the SFO to a CTR (if the MMD of the SCT has not elapsed) or query the log themselves asking an inclusion proof to a trusted STH. If no inclusion proof can be provided, the browser will report the SFO to a CTR or other auditor. However, if we presume the attacker can serve an exploit to the browser, the latter behavior is immediately vulnerable. The log, upon receiving an inclusion proof fetch for a SCT that it knows is malicious, will delay responding. The TLS connection, having succeeded will progress to the HTTP request and response, at which point the exploit will be served, and the SFO - containing the cryptographic evidence of CA and Log misbheavior - will be deleted by the exploit code. While blocking the TLS connection until the Log responds is an option, the failure of OCSP hard-fail indicates this notion is doomed to fail. % TODO: remove two uses of fail in that sentence.

The third evolution of our proposal has the Browser submit the SFO to the CTR immediately upon recipient in all cases. To mitigate the risk of an exploit finding the CTR and disclosing its identity to the attacker (who could then target it for Denial of Service), we prepare CTR circuits ahead of time and close and discard them as soon as the SFO is sent, allowing the SFO submission to race with the TLS connection completion and HTTP request/response.  An added detail is to block the TLS connection in the situation that an SFO is unusually large, as defined by a parameter ct-large-sfo-size. A large SFO may indicate an attempt to win that race between SFO submission and exploitation, and the parameter can be chosen such that it happens extremely rarely on legitimate connections.

The third evolution of our proposal exhibits all the protections we can reasonably achieve within the browser. We summarize this Phase with the following algorithm that provides more explicit steps and details, and add a parameter \texttt{ct-submit-pr} that indicates a probability a SFO is submitted to a CTR - this provides probablistic security while providing the ability to adjust submission rates to account for CTR scaling issues. Given an incoming SFO $s$, we follow the below steps: % The bit introducing \texttt{ct-submit-pr} can be better.

\begin{enumerate}
    \item Raise a certificate error and stop if the certificate chain of $s$
        is not rooted in TB's trust store.
    \item Raise a certificate transparency error and stop if the SCTs of $s$
        fail TB's CT policy.
    \item If $\mathsf{len}(s) \le \texttt{ct-large-sfo-size}$, accept $s$ and
        conduct the remaining steps in the background while the TLS connection
        and HTTP request/response proceed. If $\mathsf{len}(s) \gte \texttt{ct-large-sfo-size}$ pause the TLS handshake, complete the remaining steps and then
        accept~$s$ as valid and continue the handshake and HTTP request/response.
    \item Flip a biased coin based on \texttt{ct-submit-pr} and stop if the
        outcome indicates no further auditing.
    \item Submit $s$ to a random CTR's SFO-endpoint on a pre-built circuit.
        The circuit used for submission is closed immediately after use without
        waiting for any acknowledgment.
\end{enumerate}

\subsection{Phase 2---Storage} \label{sec:base:phase2}

We suggest that CTRs accept SFO submissions on an HTTP endpoint.\footnote{%
    Tor's HTTP DirServer codebase can be reused as extension point to interact
    with the tor daemon, i.e., add another listener.
} For example, Nordberg~\emph{et~al.} defined an SCT feedback interface that can
be reused if an array-length of one is enforced by the CTR~\cite{nordberg}.

Once received, the most straightforward thing for a CTR to do is to contact the
issuing log and request an inclusion proof relative to a trusted STH. (And if the
SCT's MMD has not elapsed, hold the SFO until it has.) However, this proposal has
two flaws, the first of which leads us to the final design of Phase 2.

Immediately contacting the log about a SFO discloses real-time browsing behavior
to the log. This is some amount of information leakage that can help with real-time
traffic analysis, and because a CTR must support storing SCTs regardless, we can
opt to effectively create a timed mix~\cite{trickle02}, though one with a
randomized firing interval. The randomness in whether a client submits an SFO
at all combined with the timed mixing potentially complicates leaking client
behavior to a CT log. But we cannot reliably assume that other honest
clients submit to the CTR between mix firings (submissions to CT
logs). Adding a per-SFO value sampled from \texttt{ct-delay-dist}
effectively adds stop-and-go mixing~\cite{kesdogan:ih1998} to the
privacy protection, but where there is only one mix (CTR) between
sender (client) and receiver (CT log). So there is no point in a
client-specified interval-start-time such that the mix drops messages
arriving before then, and there is no additional risk in having the
interval end time set by the mix rather than the sender. This means
both that some SFOs a client sends to a CTR at roughly the same time
might be in different batches of SFOs sent to a CT log and that SFOs
submitted to that CTR by other honest clients are more likely to be
mixed with these.


In addition to storing SFOs for mixing effects, we also add a layer of caching to
alleviate stored data and unnecessary log connections and data disclosure. So with
regards to some CT circuit, process an incoming SFO $s$ as follows:
\begin{enumerate}
    \item\label{enm:storage:close} Close the current circuit to enforce one-time
        usage.
    \item\label{enm:storage:unrecognized} Stop if no CT log in the Tor consensus
        accepts the trust anchor of the underlying certificate chain in $s$. % I think we should remove this
    \item\label{enm:storage:cached}
        Stop if $s$ is cached (see Section~\ref{sec:base:phase3}) or already pending to be audited.
    \item\label{enm:storage:fix-log} Sample an independent CT log $l$ that
        issued no SCT in $s$.  If there are no independent CT logs listed in the
        Tor consensus, sample a dependent CT log instead.
    \item\label{enm:storage:audit-after} Compute an \texttt{audit\_after}
        timestamp $\textrm{t} \gets \mathsf{now()} +
            \mathsf{random}(\texttt{ct-delay-dist})$.
        The former returns the current time and the latter a random delay.
    \item\label{enm:storage:store} Add $(l,t,s)$ to a buffer of pending SFOs.
\end{enumerate}

An SFO that
    (i) cannot be audited with regards to a CT log that the Tor consensus
        recognizes,
    (ii) is already audited as indicated by a \emph{cache}, or
    (iii) is pending to be audited in a \emph{buffer} of pending SFOs,
is discarded.  In contrast, a new SFO is stored in the CTR's buffer
alongside an \texttt{audit\_after} timestamp and a sampled CT log.  The
\texttt{audit\_after} timestamp specifies the earliest point in time that an SFO
will be audited in phase~3, which adds random noise that obfuscate real-time
browsing patterns in the Tor network.  Auditing is also fixed at this stage with
regards to some CT log.  If memory becomes a scarce resource, delete SFOs at
random~\cite{nordberg}.



\subsection{Phase 3---Auditing} \label{sec:base:phase3}

As alluded to in Phase 2, there is a second problem why the simple behavior of
'contact the log and request an inclusion proof' is unaccpetable. We include
the ability to DoS an individual Tor relay in our threat model - if the log
knows which CTR holds the evidence of its misbehavior, it can take the CTR
offline, wiping the evidence of the log's misbheavior from its memory. 

We can address this concern in a few ways. The simple proposal of contacting
the log over a Tor cicruit will not suffice, a log can tag each CTR by
submitting unique SFOs to them all, and recognize the CTR when they are
submitted. Unique Tor circuits per-SFO can work, however the connections
cannot be correlated; as that could lead to CTR identification as well.

% TODO HERE --------
Because connections cannot be correlated, it is not possible for a CTR to
temporarily pause connections to a CT log that is down or experiencing load, this
%TODO

If we want to build a more robust system, one that can handle disclosure of
the CTR's identity to the log, there are at least three solutions. A simple one
is to write the data to disk prior to contacting the log; however, Tor relays
are explicitly designed not to write data about user behavior to disk unless
debug-level logging is enabled. Relay operators have expressed an explicit desire
to never have any user data persisted to disk, as it changes the risk profile of
their servers with regards to search, seizure, and forensic analysis. A second
simple architecture is to create a new Tor circuit for each query to the log; 
however doing so is increasing the load on the Tor network......

The more complicated mitigation is to have the CTR work with a partner CTR - we call it
a watchdog - who they choose at random and contact over a Tor circuit. Prior to talking
to a log, the CTR provides the watchdog with the SFO it is about to submit. After
an appropriate response from the log, the CTR tells the watchdog that SFO has been
adequately addressed.  

Each CTR maintains a single shared circuit that is used to interact with all CT
logs that have \texttt{ct-log-info} items. For \emph{each} such CT log $l$, the
CTR runs the following steps indefinitely:
\begin{enumerate}
    \item\label{enm:auditing:backoff} Sample a delay $d \gets
        \mathsf{random}(\texttt{ct-backoff-dist})$ and wait until $d$ time units
        elapsed.
    \item\label{enm:auditing:loop} For each pending buffer entry $(l',s,t)$,
    where $l' = l$ and $t <= \mathsf{now}()$:
        \begin{enumerate}
            \item\label{enm:auditing:add-chain} Using \texttt{ct-log-timeout} as
                the timeout, attempt to add the certificate chain in $s$ to $l$
                with the \texttt{add-chain}~\cite{ct} or
                \texttt{submit-entry}~\cite{ct/bis} endpoints.
                \begin{itemize}
                    \item\label{enm:auditing:add-chain:success} On valid
                        SCT: cache the SFO, then discard it from the buffer of
                        pending SFOs.
                    \item\label{enm:auditing:add-chain:fail} On any other
                        outcome: go to step 1.
                \end{itemize}
        \end{enumerate}
\end{enumerate}

\subsection{Phase 4---Watchdog and Auditor Behavior}

At any given time, a CTR will be reuqesting inclusion proofs from logs (we will refer to this role as the log-challenger) and may also be a watchdog for one or more other CTRs. A CTR acting as a watchdog will have at most one SFO held temporarily for each log-challenger it is interacting with. If a response from the log-challenger is not received within \texttt{ct-watchdog-timeout}, it becomes the watchdog's responsibility to submit it to an auditor.

While we consider all auditors trusted - the watchdog needs to take precautions talking to them because the network is not trusted. If the watchdog contacted the auditor without a tor circuit, an adversary watching the auditors' network connections could induce a watchdog to contact the auditor, learn the watchdog's identity, and perform a Denial of Service attack. To mitigate this, the watchdog can contact an auditor as soon as it receives a SFO it should report; however it must contact the auditor over a Tor circuit. If a successful acknowlegement from the auditor is not received within \texttt{ct-auditor-timeout}, the SFO is placed into the general SFO buffer, where it will ultimately be presented to the log again, and potentially given to another watchdog. This strategy is done for simplicity, to avoid maintaining two separate buffers and handling evictions from them separately.

\subsection{Additional Details}

\subsubsection{CTR Flag} \label{sec:base:consensus:ctr-flag}
Within the Tor consensus, the existing \texttt{known-flags} item determines the
different flags that the consensus might contain.  We add another flag named
\texttt{CTR}, which indicates that a Tor relay should support CT-auditing as
described here. For now, assume that a relay qualifies as a CTR if it is flagged
as \texttt{stable} and not \texttt{exit}, to spare the relatively sparse exit
bandwidth and only use relays that can be expected to stay online.
Section~\ref{sec:privacy} discusses trade-offs in the assignment of the
\texttt{CTR} flag.

\subsubsection{Trusted STHs}
Tor's consensus should capture a fixed view of the CT landscape by publishing
STHs from all recognized logs.  A CT log is recognized if a majority of directory
authorities proposed a \texttt{ct-log-info} item, which contains a log's ID,
public key, base URL, MMD, and most recent STH.  Each directory authority
proposes its own STH, and agrees to use the most recent STH as determined by
timestamp and lexicographical order.  Since CTRs verify inclusion with regards
to SCTs that TB accepts, the CT logs recognized by TB must be
in Tor's consensus.

\subsubsection{Trusted Auditors}
Tor's directory authorities also majority-vote on \texttt{ct-auditor} items,
which pin base URLs and public keys of CT auditors that watchdogs contact in
case that any log misbehavior is suspected.  A watchdog triggers if the time
specified by \texttt{ct-watchdog-timeout} elapses without receiving any
acknowledgment.  The following auditor submission is governed by a
\texttt{ct-auditor-timeout}, which, if triggered, results in a resubmission
later on.

\subsubsection{Other Parameters} \label{sec:base:consensus:params}
Directory authorities influence the way in which TB and CTRs behave by
voting on other necessary parameters. Below, the value of an item is computed as
the median of all votes.
\begin{description}
    \item[ct-submit-pr:] A floating-point in $[0,1]$ that determines Tor
        Browser's submission probability.  For example, $0$ disables submissions
        while $0.10$ means that every 10$^{\mathsf{th}}$ SFO is sent to a
        random CTR on average.
    \item[ct-large-sfo-size:] A natural number that determines how many
        wire-bytes a normal SFO should not exceed.  As outlined in
        Section~\ref{sec:base:phase1}, excessively large SFOs are subject
        to stricter verification criteria.
    \item[ct-log-timeout:] A natural number that determines how long a CTR
        waits before concluding that a CT log is unresponsive, e.g., 10~seconds.
        As outlined in Section~\ref{sec:base:phase3}, timeouts trigger implicit
        resubmissions.
    \item[ct-delay-dist:] A distribution that determines how long a CTR should
        wait at minimum before auditing a submitted SFO\@.  As outlined in
        Section~\ref{sec:base:phase2}, random noise is added, e.g., on the order of minutes to an hour.
    \item[ct-backoff-dist:]
        A distribution that determines how long a CTR should wait between two
        auditing instances, e.g., a few minutes on average.  As outlined in
        Section~\ref{sec:base:phase3}, CTRs audit pending SFOs in batches at
        random time intervals to spread out log overhead.
    \item[ct-watchdog-timeout]
    \item[ct-auditor-timeout]
\end{description}


\section{Incremental Deployment} \label{sec:incremental}

The design outlined in Section 4 covers an end-to-end design that - if an SFO
is audited - ensures it will be resolved to a trusted STH and ensure it was
included within the MMD or that it will be provided to a trusted auditor for
human followup and publication. It requires changes to Tor Browser, the Tor
consensus, Tor relays, and the deployment of the new trusted auditor
infrastructure.

It is possible to provide a smaller measure of protection with fewer infrastructure
changes and less complexity.

\subsection{SubDesign 1: Detecting Malicious Certificate Authorities}

After a CTR receives a SFO, it can choose to log the certificate chain to alternate
logs who have not issued SCTs for it (based on the SCTs provided in the SFO.) If
we assume that at least one log it submits the data to is honest, the certificate will
be published and allow detection of the malicious or compromised CA.

A CTR can reuse a single tor circuit for multiple certificate submissions; however
to address the threat of a log performing a Denial of Service attack on the CTR
after learning it has evidence of misbehavior, the CTR must hold circuits to all of
the logs and submit a certificate chain to all of the logs simultaneously. This
ensures that a malicious log will not be able to take the CTR offline before the
evidence can be submitted to an honest log. Alternately, the CTR can submit a single
certificate chain per-circuit, and not perform coordinated submissions.

Submitting the certificate to the logs, one of which is presumed to be honest, and
omitting all subsequent auditing steps will not disclose what log(s) may have issued
SCTs for the certificate in question and thus be malicious; but will allow detection
of the malicious or compromised CA.

\subsection{SecDesign 2: Detecting Malicious Logs}

In a similar vein to SubDesign 1, after a CTR receives a SFO, it could log both the
certificate chain and the SCTs to alternate logs.  By publishing the SCTs in addition
to the certificate, we can see what CA was malicious or compromised, and what logs
misbehaved by not including the SCT within the MMD.

This design also relies on at least one honest log, and the same techniques for avoiding
Denial of Service attacks in SubDesign 1 apply. While this proposal avoids the need for
trusted auditors, submitting SCTs from other logs is not a mechanism that Certificate
Transparency currently supports. Therefore this proposal would require a protocol
improvement and new development on behalf of the logs.